\documentclass[a4paper,12pt]{book}
\usepackage{fontspec}
\setmainfont{Charis SIL}

\usepackage{tocbasic}
%\usetocstyle{standard}
\usepackage{longtable}
\usepackage{gb4e}
\usepackage{fullpage}
\usepackage{hyperref}
\hypersetup{
    colorlinks = true,
    linkcolor = blue
}

\usepackage{nameref}

\usepackage{multirow}% http://ctan.org/pkg/multirow
\usepackage{multicol}
\usepackage{hhline}% http://ctan.org/pkg/hhline
\newsavebox\ltmcbox

\pagestyle{headings}
\pagestyle{plain}

\newcommand{\1}[1]{\textbf{\emph{#1}}} %at'ik words
\newcommand{\2}[1]{\textbf{[#1]}} %at'ik phonetics
\newcommand{\3}[1]{\fontsize{11pt}{0cm}\textbf{\emph{#1}}} %at'ik morphemes
\newcommand{\4}[1]{\fontsize{10pt}{0cm}\emph{#1}}	%names of morphemes
\newcommand{\5}[1]{\textbf{/#1/}} %at'ik phoneMics //
\newcommand{\6}[1]{\textbf{[#1]}} %at'ik phoneTics []
\newcommand{\7}[1]{\fontsize{12pt}{0cm}\emph{#1}} %emphasis
\newcommand{\8}[1]{\fontsize{12pt}{0cm}`#1'} %English words
\newcommand{\9}[1]{\fontsize{12pt}{0cm}(lit. `#1')} %English words

\title{Kabardian Grammar in two volumes (translation)}

\begin{document}

\chapter{Creation of writing}
\section{From the History of Kabardino-Circassian Writing}
In the history of language development, the appearance of written monuments is not always connected with the creation of writing in that language. Some written monuments of the Kabardino-Circassian language belong to travelers, state officials and scholars who had no relation to the development of writing. However, there are also monuments of writing that were made in connection with the attempts to create writing in the Kabardino-Circassian language. But irrespective of the reasons and conditions of their emergence, the existing written monuments are an important material for studying the history of the Adyg languages. These monuments greatly help clarify the linguistic processes in the modern Kabardian-Circassian literary language. In this respect, the earliest materials are also of certain interest, despite their scantiness and extreme imperfection of the techniques of their graphic fixation. Thus, the lexical and textual materials of Evlius Chelebi, N. Witsen, F. I. Strallenberg, I. A. Guldenstedt, P. S. Pallas, G. Yu. Klaproth1 are not only fragmentary, but also suffer from significant flaws, which has caused skeptical attitude to them by specialists. However, it would be wrong to completely ignore them when studying the history of the Kabardino-Circassian literary language. For example, the small lexical material of N. Vitsen reflects a number of archaic features, lost by the modern Kabardino-Circassian language. The records of I. A. Guldenstedt, inaccurately transcribing the Adyghe material with signs of different graphic systems, despite their flaws, shed light on the history of the development of some classes of words. In particular, this material is valuable in reconstructing the history of the development of numerals in the Adyghe languages. The fact is that numerals in the Adyghe languages are characterized by extreme diversity in their word-formation structure. In the living dialects there are different systems of counting - decimal, twentieth and mixed, and there are great differences between the Adyghe dialects and languages in the formation of quantitative numerals. The records of I. A. Guldenstedt helps to identify later innovations and reconstructions of the common Adyghe counting system, which is important for studying the history of the development and formation of the Adyghe literary languages. Therefore, one can hardly agree with P.K. Uslar, who considered completely fruitless the activity of G.Y. Klaproth to collect the linguistic material2. With all their shortcomings, the materials of the above authors are still important for the history of the Adyg languages, that do not have a long written tradition.\\
As for the works of L. Lopatinsky and Sh. Nogmov, they made a significant contribution to the study of the Kabardian-Circassian language in the pre-revolutionary period (see below). Especially valuable for the history of the Kabardino-Circassian literary language are the lexicographical studies of the mentioned authors.\\
Together with the study of the Kabardino-Circassian language, attempts to create a written language were made in the pre-revolutionary period. In the beginning of the 19th century, Russian educators of Adygea people worked on compiling the Kabardian-Circassian alphabet and alphabet.\\
In 1829 I. Gratsilevsky, a teacher at St. Petersburg University, created the Circassian alphabet based on Russian graphics. As S. Petin notes, with the help of Grazilevsky alphabet "students even used to correspond with each other later "1. 1 The alphabet of I. Grazilevsky compiled for soldiers of the Caucasus semi-squadron was not distributed.\\

\chapter{Phonetics and Phonology}
\chapter{Morphology}
\section{General characteristics of the morphological system}
In terms of the complexity of the systems of word-formation and word-formation, the name and the verb form two poles. The name, as compared to the verb, has a relatively simple structure. Nouns have grammatical forms of number, definiteness-undefiniteness, case, possessive and conjunctive. All these grammatical forms, except possessive, are expressed by means of suffixal morphemes. Prefixation is used to express possessiveness. Cf.: \textbf{унэ} "house", \textbf{унэхэр} "house", \textbf{унэр} "house" (definite, known), \textbf{уыни} "and house" (union form); \textbf{шы} "horse", \textbf{сиш} "my horse", \textbf{Iэ} "hand", \textbf{сиIэ} "my hand".\\
The distinction between the main two cases, the Nominative and the Ergative, is related to the nature of the verb base. A transitive verb creates an ergative construction and a non-transitive one creates a nominative construction, for example: \textbf{абы тхылъыр къищэхуащ} "he bought a book" (\textbf{абы} "he" is ergative, \textbf{тхылъыр} "book" is neuter, \textbf{къищэхуащ} is infinitive); \textbf{ар мэлажьэ} "he works" (\textbf{ар} "he" is neuter, \textbf{мэлажьэ} is a non-receptive verb).\\
From the point of view of morphology, adjectives are not very clearly distinguished from nouns. The latter have degrees of comparison - positive, comparative and superlative. Degrees of comparison are expressed in two ways: synthetic (or rather suffixal) and analytic, for example: \textbf{IэфIыщэ} "too sweet", \textbf{нэхъ IэфI} "more sweet", \textbf{нэхъ IэфI дыдэ} "the sweetest".\\
Derivative names are formed by means of affixation, word formation and conversion, for example: \textbf{цIыху-гъэ} "humanity" (suffixation); \textbf{джэдкъаз} "poultry" (\textbf{джэд} "chicken", \textbf{къаз} "goose"), \textbf{псчэ} "cough" from \textbf{псчэ-н} "to cough" (conversion). The most productive of these methods of nominal word formation is word formation. Of the affixal types of word formation in names, only the suffixal one is active. It should be noted that suffixal word formation is mostly characteristic of nouns. As for the word-formation suffixes of adjectives, they are very few, and the most productive of them have not been completely grammatized.\\
In combinations 'noun + noun' and 'noun + adjective', the word-change suffixes are attached only to the postpositional member, e.g.: \textbf{пхъэ унэ-хэ-р} 'wooden houses', \textbf{унэ дахэ-хэ-р} 'beautiful houses', (\textbf{-хэ} is a plural suffix, \textbf{-р} is a neuter suffix).\\
The verb has an exceptionally rich and complex system of word-formation and word-formation. The verb is the clearest of all typological features of the Kabardino-Circassian language.\\
Paradigmatically, we distinguish between transitive and non-transitive verbs, dynamic and static verbs, finitic and infinitic verbs. The grammatical meaning of the verb base determines the paradigms of conjugation and the sentence structure as a whole. Primary Static verb bases, in contrast to Dynamic verb bases, are in the closed list, i.e., they are quantitatively restricted. The paradigm of the Static verbs may include noun bases without any stemming morpheme. Cf. e.g.: \textbf{у-щIалэ-тэ-мэ} 'if you were a young man' (y is the subject prefix of the 2nd person singular, shIale is the stem of 'young man', -tae is the temporal suffix, -me is the inflectional suffix). This indicates a weak morphological differentiation of nominal and verbal bases. In the synchronic plane, there are bases that are neutral with respect to their belonging to nominal and verbal bases. From such bases are formed forms of nouns and verbs (static and dynamic). Cf. jaegu "game", so-jaegu "I play", ar jaegu-ch "this is a game".\\
Verbs are characterized by affixes of person (subject, direct object, indirect object), number, union, negation, time, inclination, causative, version, possibility, complicity, reciprocity, jointness, reversibility, locative and directive preverbs, modal particles, etc. A distinctive feature of the verb is also the circumstantial forms, functionally equivalent to adjective sentences. These are morphological forms such as \textbf{сы-зэрыкIуэр} "as I go", \textbf{сы-зэрыкIуэ-у} "as soon as I go", \textbf{сы-зэры-кIуэ-рэ} "since I went", \textbf{сы-здэ-кIуэ-р} "where I go", \textbf{сы-щIэ-кIуэ-р} "why I go", \textbf{сы-щы-кIуэ-р} "when I go", \textbf{сы-кIуэ-ху} "while I go".\\
The forms of person, version, dynamism, complicity, reciprocity, jointness, and place of action are expressed by prefixes, e.g.: \textbf{с-о-кIуэ} "I go", \textbf{у-о-кIуэ} "you go", \textbf{сы-ху-о-кIуэ} "I go for him", \textbf{сы-до-кIуэ} "I go together with him", \textbf{ды-зд-о-кIуэ} "we go together", \textbf{ды-зэр-о-лъагъу} "we see each other", \textbf{сы-те-с-щ} "I sit on something".\\
The forms of tense, inclination and conjunction are expressed by suffixes, e.g., sy-kIu-a-sh "I went", sy-kIua-me "if I went", sy-kIue-ri "I went and".\\
Forms of negation and possibility can be expressed in prefixal and suffixal ways, cf. sy-my-kIue "I am not going" (we- is the prefix of negation), sy-kIuer-kyym (-kyym is the suffix of negation), s-hue-hyinsh "I can carry" (hue- is the prefix of possibility), sy-kIue-fyn-sh "I can go" (-fa is the suffix of possibility).\\
The direction of action is expressed in prefixal and prefix-suffixal ways, e.g.: sy-ky-o-kIue "I go here" (kъ(e)- direction prefix), sy-d-o-kIue-y "I go up" (d(e)- direction prefix, -y(s) - direction suffix).\\
Different methods are used to express circumstantial forms - prefixal, suffixal and prefix-suffixal, for example: zer-i-thyr "as he writes" [zer- is a prefix meaning "as"], i-thy-hu "while he writes" (-hu is a suffix meaning "until"), zer-i-thy-u "as soon as he writes" [prefix zer- and suffix -u meaning "as soon as "*].\\
The verb form may be single-morphemic, cf. zhe 'run! "write!" But the verb as a whole is characterized by a high degree of synthesis.\\
The process of fattening a root morpheme with affixes (prefixes and suffixes) is illustrated by the following example:
у-а-къы-д-е-з-гъэ-шэ-жы-ф-а-тэ-къым "I couldn't get him to lead you with them here then."\\
\chapter{Syntax}
\end{document}

\documentclass[a4paper,12pt]{book}
\usepackage{fontspec}
\setmainfont{Charis SIL}

\usepackage{tocbasic}
%\usetocstyle{standard}
\usepackage{longtable}
\usepackage{gb4e}
\usepackage{fullpage}
\usepackage{hyperref}
\hypersetup{
    colorlinks = true,
    linkcolor = blue
}

\usepackage{nameref}

\usepackage{multirow}% http://ctan.org/pkg/multirow
\usepackage{multicol}
\usepackage{hhline}% http://ctan.org/pkg/hhline
\newsavebox\ltmcbox

\pagestyle{headings}
\pagestyle{plain}



\newcommand{\1}[1]{\textbf{\emph{#1}}} %at'ik words
\newcommand{\2}[1]{\textbf{[#1]}} %at'ik phonetics
\newcommand{\3}[1]{\fontsize{11pt}{0cm}\textbf{\emph{#1}}} %at'ik morphemes
\newcommand{\4}[1]{\fontsize{10pt}{0cm}\emph{#1}}	%names of morphemes
\newcommand{\5}[1]{\textbf{/#1/}} %at'ik phoneMics //
\newcommand{\6}[1]{\textbf{[#1]}} %at'ik phoneTics []
\newcommand{\7}[1]{\fontsize{12pt}{0cm}\emph{#1}} %emphasis
\newcommand{\8}[1]{\fontsize{12pt}{0cm}`#1'} %English words
\newcommand{\9}[1]{\fontsize{12pt}{0cm}(lit. `#1')} %English words

\newcommand{\glossphonemics}[1]{\textbf{/#1/}} %at'ik phoneMics //

\title{Kabardian Grammar in two volumes (translation)}

\begin{document}
\frontmatter
\maketitle\newpage
\setcounter{tocdepth}{7}\tableofcontents

\newpage
\mainmatter
\chapter{Creation of writing}
\section{From the History of Kabardino-Circassian Writing}
In the history of language development, the appearance of written monuments is not always connected with the creation of writing in that language. Some written monuments of the Kabardino-Circassian language belong to travelers, state officials and scholars who had no relation to the development of writing. However, there are also monuments of writing that were made in connection with the attempts to create writing in the Kabardino-Circassian language. But irrespective of the reasons and conditions of their emergence, the existing written monuments are an important material for studying the history of the Adyg languages. These monuments greatly help clarify the linguistic processes in the modern Kabardian-Circassian literary language. In this respect, the earliest materials are also of certain interest, despite their scantiness and extreme imperfection of the techniques of their graphic fixation. Thus, the lexical and textual materials of Evlius Chelebi, N. Witsen, F. I. Strallenberg, I. A. Guldenstedt, P. S. Pallas, G. Yu. Klaproth1 are not only fragmentary, but also suffer from significant flaws, which has caused skeptical attitude to them by specialists. However, it would be wrong to completely ignore them when studying the history of the Kabardino-Circassian literary language. For example, the small lexical material of N. Vitsen reflects a number of archaic features, lost by the modern Kabardino-Circassian language. The records of I. A. Guldenstedt, inaccurately transcribing the Adyghe material with signs of different graphic systems, despite their flaws, shed light on the history of the development of some classes of words. In particular, this material is valuable in reconstructing the history of the development of numerals in the Adyghe languages. The fact is that numerals in the Adyghe languages are characterized by extreme diversity in their word-formation structure. In the living dialects there are different systems of counting - decimal, twentieth and mixed, and there are great differences between the Adyghe dialects and languages in the formation of quantitative numerals. The records of I. A. Guldenstedt helps to identify later innovations and reconstructions of the common Adyghe counting system, which is important for studying the history of the development and formation of the Adyghe literary languages. Therefore, one can hardly agree with P.K. Uslar, who considered completely fruitless the activity of G.Y. Klaproth to collect the linguistic material2. With all their shortcomings, the materials of the above authors are still important for the history of the Adyg languages, that do not have a long written tradition.\\
As for the works of L. Lopatinsky and Sh. Nogmov, they made a significant contribution to the study of the Kabardian-Circassian language in the pre-revolutionary period (see below). Especially valuable for the history of the Kabardino-Circassian literary language are the lexicographical studies of the mentioned authors.\\
Together with the study of the Kabardino-Circassian language, attempts to create a written language were made in the pre-revolutionary period. In the beginning of the 19th century, Russian educators of Adygea people worked on compiling the Kabardian-Circassian alphabet and alphabet.\\
In 1829 I. Gratsilevsky, a teacher at St. Petersburg University, created the Circassian alphabet based on Russian graphics. As S. Petin notes, with the help of Grazilevsky alphabet "students even used to correspond with each other later "1. 1 The alphabet of I. Grazilevsky compiled for soldiers of the Caucasus semi-squadron was not distributed.\\

\chapter{Phonetics and Phonology}
\chapter{Morphology}
\section{General characteristics of the morphological system}
In terms of the complexity of the systems of word-formation and word-formation, the name and the verb form two poles. The name, as compared to the verb, has a relatively simple structure. Nouns have grammatical forms of number, definiteness-undefiniteness, case, possessive and conjunctive. All these grammatical forms, except possessive, are expressed by means of suffixal morphemes. Prefixation is used to express possessiveness. Cf.: \textbf{унэ} "house", \textbf{унэхэр} "house", \textbf{унэр} "house" (definite, known), \textbf{уыни} "and house" (union form); \textbf{шы} "horse", \textbf{сиш} "my horse", \textbf{Iэ} "hand", \textbf{сиIэ} "my hand".\\
The distinction between the main two cases, the Nominative and the Ergative, is related to the nature of the verb base. A transitive verb creates an ergative construction and a non-transitive one creates a nominative construction, for example: \textbf{абы тхылъыр къищэхуащ} "he bought a book" (\textbf{абы} "he" is ergative, \textbf{тхылъыр} "book" is neuter, \textbf{къищэхуащ} is infinitive); \textbf{ар мэлажьэ} "he works" (\textbf{ар} "he" is neuter, \textbf{мэлажьэ} is a non-receptive verb).\\
From the point of view of morphology, adjectives are not very clearly distinguished from nouns. The latter have degrees of comparison - positive, comparative and superlative. Degrees of comparison are expressed in two ways: synthetic (or rather suffixal) and analytic, for example: \textbf{IэфIыщэ} "too sweet", \textbf{нэхъ IэфI} "more sweet", \textbf{нэхъ IэфI дыдэ} "the sweetest".\\
Derivative names are formed by means of affixation, word formation and conversion, for example: \textbf{цIыху-гъэ} "humanity" (suffixation); \textbf{джэдкъаз} "poultry" (\textbf{джэд} "chicken", \textbf{къаз} "goose"), \textbf{псчэ} "cough" from \textbf{псчэ-н} "to cough" (conversion). The most productive of these methods of nominal word formation is word formation. Of the affixal types of word formation in names, only the suffixal one is active. It should be noted that suffixal word formation is mostly characteristic of nouns. As for the word-formation suffixes of adjectives, they are very few, and the most productive of them have not been completely grammatized.\\
In combinations 'noun + noun' and 'noun + adjective', the word-change suffixes are attached only to the postpositional member, e.g.: \textbf{пхъэ унэ-хэ-р} 'wooden houses', \textbf{унэ дахэ-хэ-р} 'beautiful houses', (\textbf{-хэ} is a plural suffix, \textbf{-р} is a neuter suffix).\\
The verb has an exceptionally rich and complex system of word-formation and word-formation. The verb is the clearest of all typological features of the Kabardino-Circassian language.\\
Paradigmatically, we distinguish between transitive and non-transitive verbs, dynamic and static verbs, finitic and infinitic verbs. The grammatical meaning of the verb base determines the paradigms of conjugation and the sentence structure as a whole. Primary Static verb bases, in contrast to Dynamic verb bases, are in the closed list, i.e., they are quantitatively restricted. The paradigm of the Static verbs may include noun bases without any stemming morpheme. Cf. e.g.: \textbf{у-щIалэ-тэ-мэ} 'if you were a young man' (y is the subject prefix of the 2nd person singular, shIale is the stem of 'young man', -tae is the temporal suffix, -me is the inflectional suffix). This indicates a weak morphological differentiation of nominal and verbal bases. In the synchronic plane, there are bases that are neutral with respect to their belonging to nominal and verbal bases. From such bases are formed forms of nouns and verbs (static and dynamic). Cf. jaegu "game", so-jaegu "I play", ar jaegu-ch "this is a game".\\
Verbs are characterized by affixes of person (subject, direct object, indirect object), number, union, negation, time, inclination, causative, version, possibility, complicity, reciprocity, jointness, reversibility, locative and directive preverbs, modal particles, etc. A distinctive feature of the verb is also the circumstantial forms, functionally equivalent to adjective sentences. These are morphological forms such as \textbf{сы-зэрыкIуэр} "as I go", \textbf{сы-зэрыкIуэ-у} "as soon as I go", \textbf{сы-зэры-кIуэ-рэ} "since I went", \textbf{сы-здэ-кIуэ-р} "where I go", \textbf{сы-щIэ-кIуэ-р} "why I go", \textbf{сы-щы-кIуэ-р} "when I go", \textbf{сы-кIуэ-ху} "while I go".\\
The forms of person, version, dynamism, complicity, reciprocity, jointness, and place of action are expressed by prefixes, e.g.: \textbf{с-о-кIуэ} "I go", \textbf{у-о-кIуэ} "you go", \textbf{сы-ху-о-кIуэ} "I go for him", \textbf{сы-до-кIуэ} "I go together with him", \textbf{ды-зд-о-кIуэ} "we go together", \textbf{ды-зэр-о-лъагъу} "we see each other", \textbf{сы-те-с-щ} "I sit on something".\\
The forms of tense, inclination and conjunction are expressed by suffixes, e.g., sy-kIu-a-sh "I went", sy-kIua-me "if I went", sy-kIue-ri "I went and".\\
Forms of negation and possibility can be expressed in prefixal and suffixal ways, cf. sy-my-kIue "I am not going" (we- is the prefix of negation), sy-kIuer-kyym (-kyym is the suffix of negation), s-hue-hyinsh "I can carry" (hue- is the prefix of possibility), sy-kIue-fyn-sh "I can go" (-fa is the suffix of possibility).\\
The direction of action is expressed in prefixal and prefix-suffixal ways, e.g.: sy-ky-o-kIue "I go here" (kъ(e)- direction prefix), sy-d-o-kIue-y "I go up" (d(e)- direction prefix, -y(s) - direction suffix).\\
Different methods are used to express circumstantial forms - prefixal, suffixal and prefix-suffixal, for example: zer-i-thyr "as he writes" [zer- is a prefix meaning "as"], i-thy-hu "while he writes" (-hu is a suffix meaning "until"), zer-i-thy-u "as soon as he writes" [prefix zer- and suffix -u meaning "as soon as "*].\\
The verb form may be single-morphemic, cf. zhe 'run! "write!" But the verb as a whole is characterized by a high degree of synthesis.\\
The process of fattening a root morpheme with affixes (prefixes and suffixes) is illustrated by the following example:
у-а-къы-д-е-з-гъэ-шэ-жы-ф-а-тэ-къым "I couldn't get him to lead you with them here then."\\

\section{Stem structure}
\section{Noun}
\section{Pronoun}
\section{Adjective}
\section{Numeral}
\section{Ablaut in Nominal Word Formation}
\section{Verb}
\subsection{General characteristics of verb polysynthesis}
The Adyg languages have inherited from the primordial state a verb system with an extremely complex structure and a high degree of synthesis.

The verb word form may be single-morphemic: common Adyg. dy- "to sew", khy- "to mow", zy- "to tsed". The single-morphemic base of the mentioned type varies depending on the surroundings in modern Adyghe languages as CV, C: dy! "shi!" / from dy- "sew" /, so-d "I sew". In such cases, the all-Adygian structure of the base is preserved in Adygean: se-dy "I sew. However, the depth of the verb word-form may be very large.\\
An example, taken from the Kabardian language, gives an idea of the degree and depth of verb synthesis:\\
\textbf{шэ}\\
\textbf{е- шэ -ж}\\
\textbf{з- гъэ- шэ -жы -ф}\\
\textbf{е- з- гъэ- шэ -жы -ф -а -т}\\
\textbf{д- е- з- гъэ- шэ -жы -ф -а -т}\\
\textbf{къIы- д- е- з- гъэ- шэ -жы -ф -а -тэ -къIым}\\
"lead"\\
"he leads him back."\\
"make me lead him back."\\
"I could then make him lead back."\\
"I could then make him lead him back"\\
"I couldn't get him to drive back here with him then."\\

In this "triangle," the root morpheme she "lead" of the last /sixth/ word form is complicated by ten affixal elements - five prefixes and five suffixes. But there are also more extended chains of morphemes in the word paradigm. The latter chain can be extended by including several more affixal morphemes in the paradigm: \textbf{у-а-къIы-ды-д-е-з-гъэ-шы-жы-ф-а-тэ-къIым-и}i "I could not then get him to take you back out of there with them." This paradigm within one word consists of fifteen significant elements /eight prefixes + root morpheme + six suffixes/: y- prefix of direct object of 2nd person singular, a- prefix of indirect object of 3rd person plural, kъIy- directional prefix "here", y- prefix of allied action, e- local prefix, f- prefix of object of 3rd person singular, h- subject prefix of the 1st person singular singular. g-e causative prefix, shy- morpheme of the verb she-n "to lead", -y return suffix, -f possibility suffix, -a past tense suffix, -te suffix "then", -kъIym negative suffix, -i suffix with modal meaning. The high degree of synthesis in a verb is created by the fact that not only the categories of person, number, time and inclination are expressed morphologically, but also negation, affirmation, question, causative, joint action, localization, directional relations, etc.

The structure of the verb, its main word-formation and word-formation categories, their relations, the hierarchy and arrangement of numerous base-forming elements, the distribution of personal affixes in different types of bases, their dependence on other categorical forms, including verb transitivity-non-transitivity, etc., date back to the era of the All-Adyghe language unity. Moreover, the most important material elements (affixes of person, causative, negation, and a number of other formants serving to express forms of word-formation and word-formation) are chronologically derived from the Western Caucasian level.

It is important to stress that the genetically common material features that unite Adyghe verb with the verb of other West-Caucasian languages are inseparable from the structural-typological commonalities that also go back to the West-Caucasian condition. As we shall see, the structural hierarchical features, which have their roots in the epoch of prelanguage unity of the Western Caucasian languages, turned out to be extremely stable in the whole group of languages, indicating a close relationship between genetic and typological features of the verb. In this respect, the verb differs from other classes of words, preserving a number of primordial structural-typological and genetic features that unite all modern Western Caucasian languages. In any case, in terms of the specific weight of structural-typological features, which are attributed to the Western Caucasian condition, the verb has no counterpart among other parts of speech. This refers primarily to the distribution of word-formation and word-formation morphemes, including subject-object indicators.

The main structural types of the verb stem coincide in the Abkhazian-Adygian languages, although the differences between their two main subgroups - the Adygean-Kabardian and Abkhazian-Abazian - are so significant that so far not all consistent sound correspondences between these languages have been established. At the same time, the order of verbal derivational morphemes in the Abkhazian-Adygian languages is so similar that knowing the arrangement of the significant elements of the stem in the Adygian languages, we can build models of the verb stem multimorphemes in the Abkhazian or Abazinian. So, if the type of stem including the root morpheme /K/ and morphemes expressing stimulus /P/, specific localization /L/, joint action /C/ is given, then in the whole group of languages the significant elements of the stem are distributed as follows: S - L - P - C.

The main patterns of distribution of subject-object, derivational prefixal and root morphemes cannot be explained as a result of parallel development of the Western Caucasian languages, but must be attributed to their pra-language condition.\\
\subsection{Transitive and non-transitive verbs}
In the complex system of word-formation and word-formation of a verb, the division into transitive and non-transitive verbs is central. The fact is that the sign of transitivity is present in all words and verb forms.

The categorical (invariant) meaning of transitivity-non-transitivity can be considered as syntagmatic, as it depends on the presence of subject and object in a certain case form of the name and on how they are represented in the verb form. A transitive verb requires a subject in the ergative case and an object in the nominative case: \textbf{Дыгъэм щIыр егъэхуабэ} \glossphonemics{dəʁam ɕʼər jaʁaxʷaːba} - "The sun warms the earth". Here, the subject \textbf{дыгъэм} is in the ergative case and the object \textbf{щIыр} in the nominative case. A non-transitive verb has the subject in the nominative case and the object in the ergative case: The subject is in the nominative case and the object is in the ergative case. The subject is in the nominative case and the object is in the ergative case. Non-transitive (more often) and transitive (less often) may be objectless: \textbf{ЩIалэр йоджэр} \glossphonemics{ɕʼaːɮar jawd⁀ʒar} - "A young man studies" and \textbf{ЩIалэм къекIухь} \glossphonemics{ɕʼaːɮam qajkʷʼəħ} - "A young man walks".

A syntactic construction with a transitive verb is called an ergative verb and with a non-transitive one, a nominative verb.

In the above constructions, the case form of the name (subject and object) depends on the verb, but the name form also predetermines the transitive and non-transitive nature of the verb (especially the objectless one), which is evidence of the analytic relation of the name and the verb.

The order of the subject and object affixes in the verbal form is no less important for transitivity and non-transitivity. Thus, in transitive verbs, the subject affix is preceded by the subject index: \textbf{уы-з-олъагъу} \glossphonemics{wəzawɬaːʁʷ} - "I see you" \textbf{уы-} - the object index in the 2nd person singular and \textbf{-з-} the subject affix in the 1st person singular. Transitive verbs may be bipersonal, three-personal and four-personal. The personal affixes are so arranged: in the two-personal (see above), in the three-personal, the direct object prefix comes first and the indirect object prefix second. The prefix of the direct object comes first, the prefix of the indirect object comes second, and the prefix of the subject comes third: \textbf{сы-къы-п-ху-ишащ} \glossphonemics{səqəpxʷəjʃaːɕ} - "he has brought me to you". The two-person transitive verbs in the reflexive form become monolingual but retain their transitivity: cf. \textbf{абы ар итхьэщIащ} \glossphonemics{aːbə aːr jətħaɕʼaːɕ} "he washed that", \textbf{абы зитхьэщIащ} \glossphonemics{aːbə zəjtħaɕʼaːɕ} "he washed himself" \textbf{абы ар игъэпскIащ} \glossphonemics{aːbə aːr jəʁapskʼaːɕ} "he bathed that"; \textbf{абы зигъэпскIащ} \glossphonemics{aːbə zəjʁapskʼaːɕ} "he bathed himself".

Non-transitive verbs can be both singular and plural: \textbf{Шыр мажэ} - a horse runs - a singular verb; \textbf{ЩIалэр хъыджэбзым lжьэ} - a young man waits for a girl - a two-person verb, etc.

Non-transitive verbs form another syntactic construction, the inverse one, in which the real subject acts as an indirect object and the real object is the grammatical subject: \textbf{Сымаджэм мыIэрысэр хуошх} - The sick manages the apple; \textbf{Студентым тхылъыр иIэщ} - The student has a book. In these sentences, the real subject (\textbf{сымаджэм, студентым}) acts as an indirect object and the real object (\textbf{тхылъыр, мыIэрысэр}) becomes the grammatical subject. Such is the nature of inversion.

The subject-object forms of the verb are a sign of syncretism, but since they represent names that are present in syntactic constructions, they are also analytic in nature.
\subsubsection{Transposition of non-transitive verbs into transitive verbs}
Transitive verbs may be primary (simple) and secondary (derivative). Primary verbs include \textbf{хьын} \glossphonemics{ħən} (\textbf{ехь} \glossphonemics{jaħ}) "to carry", \textbf{дзын} \glossphonemics{d⁀zən} "to throw", and others; secondary verbs are \textbf{кIуэн} \glossphonemics{kʷʼan}, "to go"; \textbf{кIуын} \glossphonemics{kʷʼən}"to pass" and \textbf{гъэкIуэн} \glossphonemics{ʁakʷʼan} "to make go, to send"; \textbf{тхэн} \glossphonemics{txan} "to write"; \textbf{тхын} \glossphonemics{txən} "to write something" and \textbf{гъэтхэн} \glossphonemics{ʁatxan} "to make write something", etc.

The ways of forming secondary transitive verbs are varied:
\begin{xlist}
\ex The causative prefix \textbf{гъэ-} transposes a non-transitive (dynamic and static) verb into a transitive verb: 
	\begin{xlist} 
	\ex \textbf{псэлъэн} \glossphonemics{psaɬan} (\textbf{мэпсалъэ} \glossphonemics{mapsaːɬa}) - "to speak", \textbf{гъэпсэлъэн} \glossphonemics{ʁapsaɬan} (\textbf{егъэпсэлъэ} \glossphonemics{jaʁapsaɬa}) "to make speak, to let repeat";
	\ex \textbf{щытын} \glossphonemics{ɕətən} (\textbf{щытщ} \glossphonemics{cətɕ}) "to stand", becomes \textbf{щы-гъэ-тын} \glossphonemics{ɕəʁatən} (\textbf{щегъэт} \glossphonemics{ɕajʁat}) "to make stand"
	\end{xlist}
\ex Preverb \textbf{къэ} - translates non-transitive into transitive quite actively:

	\begin{xlist}
	\ex \textbf{лъыхъуэн} \glossphonemics{ɬəχʷan} (\textbf{мэлъыхъуэ} \glossphonemics{maɬəχʷa}) "to seek", becomes \textbf{къэлъыхъуэн} \glossphonemics{qaɬəχʷan} (\textbf{къелъыхъуэ} \glossphonemics{qajɬəχʷa}) - "to seek something"
	\end{xlist}
	
\ex The translation of non-transitive into transitive with the preverb \textbf{къэ-} is accompanied by the alternation of the main\textbf{э:ы} in verbs: 
	\begin{xlist}
	\ex \textbf{гупсысэн} \glossphonemics{gʷəpsəsan} (\textbf{мэгупсысэ} \glossphonemics{magʷəpsəsa}) "to think", becomes \textbf{къэгупсысын} \glossphonemics{qagʷəpsəsən} (\textbf{къегупсыс} \glossphonemics{qajgʷəpsəs}) "to invent something"

	\ex \textbf{лэжьэн} \glossphonemics{ɮaʑan} (\textbf{мэлажьэ} \glossphonemics{maɮaːʑa} "to work", becomes \textbf{къэлэжьын} \glossphonemics{qaɮaʑən} (\textbf{къелэжь} \glossphonemics{qajɮaʑ} "to earn something"
	\ex \textbf{дыгъуэн} \glossphonemics{dəʁʷan} (\textbf{мэдыгъуэ} \glossphonemics{madəʁʷa}) - "to steal", becomes \textbf{къэдыгъун} \glossphonemics{qajdəʁʷən} (\textbf{къедыгъу} \glossphonemics{qajdəʁʷ}) "to steal something"
	\end{xlist}

\ex Preverb \textbf{къэ-} with suffix \textbf{-хьы} - accompanied by alternation of basic \textbf{эIы,}, translates non-transitive to transitive:
\begin{xlist} 
\ex \textbf{жэн} \glossphonemics{ʒan } (\textbf{мажэ}  \glossphonemics{maːʒa}) "to run", becomes \textbf{къэжыхьын } \glossphonemics{qaʒəħən} (\textbf{къежыхь}  \glossphonemics{qajʒəħ}) "to run around something"

\ex \textbf{джэдэн} \glossphonemics{d͡ʒadan} (\textbf{мэджэдэ}  \glossphonemics{mad͡ʒada}) "to wander somewhere", becomes \textbf{къэджэдыхьын} \glossphonemics{qad⁀ʒadəħən} (\textbf{къеджэдыхь}  \glossphonemics{qajd͡ʒadəħ}) "to wander without a definite direction".

\ex \textbf{лъэтэн} \glossphonemics{ɬatan} (\textbf{мэлъатэ}  \glossphonemics{maɬaːta}) "to fly", becomes \textbf{къэлъэтыхьын} \glossphonemics{qaɬatəħən} (\textbf{къелъэтыхь}  \glossphonemics{qajɬatəħ}) "to fly around something".

\ex \textbf{вэн} \glossphonemics{van} (\textbf{мавэ}  \glossphonemics{maːva}) "to plow", becomes \textbf{къэвыхьын} \glossphonemics{qavəħən} (\textbf{къевыхь}  \glossphonemics{qajvəħ}) "to plow around something".


\end{xlist}
\ex Preverb \textbf{къэ} + suffix \textbf{-хьы-}: 
\begin{xlist}

\ex \textbf{пщын} \glossphonemics{pɕən} (\textbf{мэпщ} \glossphonemics{mapɕ}) "to crawl", \textbf{къэпщын} \glossphonemics{qapɕən} (\textbf{къопщ} \glossphonemics{qawpɕ}) "to crawl here", becomes \textbf{къэпщыхьын} \glossphonemics{qapɕəħən} (\textbf{къепщыхь} \glossphonemics{qajpɕəħ}) "to crawl without definite direction"

\ex \textbf{плъэн} \glossphonemics{pɬan} (\textbf{маплъэ} \glossphonemics{maːpɬa}) "to look", \textbf{къэплъэн} \glossphonemics{qapɬan} (\textbf{къоплъэ} \glossphonemics{qawpɬa}) "to look here", becomes \textbf{къэплъэхьын} \glossphonemics{qapɬaħən} (\textbf{къеплъэхь} \glossphonemics{qajpɬaħ}) "to look without definite direction"

\end{xlist}

\ex Singular verbs with postverb \textbf{-хьы-} with accompanying alternation of basic \textbf{э/ы}:
	\begin{xlist}
	\ex \textbf{къэфэн} \glossphonemics{qafan} (\textbf{къофэ} \glossphonemics{qawfa}) "to dance", becomes \textbf{къэфыхьын} \glossphonemics{qafəħən} (\textbf{къефыхь} \glossphonemics{qajfəħ}) "to dance around something"
	\end{xlist}

\ex Transitive verbs are formed from non-transitive verbs by means of the affix \textbf{-ы-}:
	\begin{xlist}
	\ex \textbf{кIуэн} \glossphonemics{kʷʼan} (\textbf{макIуэ} \glossphonemics{maːkʷʼa}) "to go", becomes \textbf{кIун} \glossphonemics{kʷʼən} (\textbf{екIу} \glossphonemics{jakʷʼ}) "to pass (a separate distance)"
	\ex \textbf{тхэн} \glossphonemics{txan} (\textbf{матхэ} \glossphonemics{maːtxa}) "to write", becomes \textbf{тхын} \glossphonemics{txən} (\textbf{етх} \glossphonemics{jatx}) "to write something"
	\ex \textbf{дэн} \glossphonemics{dan} (\textbf{мадэ} \glossphonemics{maːda}) "to sew", becomes \textbf{дын} \glossphonemics{dən} (\textbf{ед} \glossphonemics{jad}) "to sew something"
	\ex \textbf{къэпсэлъэн} \glossphonemics{qapsaɬan} (\textbf{къопсалъэ} \glossphonemics{qawpsaɬa}) "to speak", becomes \textbf{къэпсэлъын} \glossphonemics{qapsaɬən} (\textbf{къепсэлъ} \glossphonemics{qajpsaɬ}) "to say something"
	
	\ex \textbf{тхьэщIэн} \glossphonemics{tħaɕʼan} (\textbf{мэтхьэщIэ} \glossphonemics{matħaɕʼa}) "to wash", becomes \textbf{тхьэщIын} \glossphonemics{tħaɕʼən} (\textbf{къетхьэщI} \glossphonemics{qajtħaɕʼ}) "to wash something"
	\end{xlist}
\end{xlist}

Transitive verbs can transpose into transitive verbs in certain forms:

1) in the potency form with the prefix \textbf{хуэ-}: \textbf{шхын} - to eat, to eat - \textbf{хуошх} - can eat, \textbf{сымаджэм мыIэрысэр ешх} (trans. d.) - the patient eats an apple, but \textbf{сымаджэм мыIэрысэр хуошх} (trans. d.) - the patient manages with an apple;

2) in the involuntary form with the preverb \textbf{IэщIэ-}: \textbf{ЩакIуэм дыгъужьыр къиукIащ} \glossphonemics{ɕaːkʷʼam dəʁʷʑər qəjəwət͡ʃʼaːɕ} (Transl. d.) - The hunter shot the wolf - \textbf{ЩакIуэм хьэр IэщIэукIащ} \glossphonemics{ɕaːkʷʼam ħar ʔaɕʼawət͡ʃʼaːɕ} (Neoter. d.) - The hunter involuntarily shot the dog.

3) in the reciprocal form with the prefix \textbf{зэры-} \glossphonemics{zarə-}: \textbf{Абы ар ешэ} \glossphonemics{aːbə aːr jaʃa} - (transitive) - "He will marry her (here)" - \textbf{Ахэр зэрошэхэ} \glossphonemics{aːxar zarawʃaxa} (intransitive) "They will marry".

4) in the object version with the prefix \textbf{фIэ-} \glossphonemics{fʼa-}: some transitive verbs become non-transitive: \textbf{Абы ар ещIэ} \glossphonemics{aːbə aːr jaɕʼa} - He knows that - \textbf{Абы ар къыфIощI} \glossphonemics{aːbə aːr qəfʼawɕʼ} - He seems that; \textbf{Хамэ хьэдэр жей къыпфIощI} (Ps.) \glossphonemics{ħaːma ħadar ʑaj qəpfʼawɕʼ} - A strange dead man seems to be asleep.

\subsection{Dynamic and static verbs}
Dynamic verbs express the process of action, e.g.: se sothe(r) "I write (in general)"; se ar sothe(r) "I write (that)"; se ar soshe(r) "I lead it"; se abi sojhe(r) "I wait for it".
Static verbs express the state, the result of an action: se syshytshch "I stand"; se ar siyeshch "I have that".
Static verbs are also predicative forms of names and pronouns: ar studentshch "that student (is)"; ar dakheshch "that handsome (is)"; studentr arshch "that student (is)".
Morphologically, Dynamic and Static verbs are distinguished by their Present Form. Dynamic verbs in the present tense have the prefix o-(ue-): se s-o-kIue(r) "I go"; se ar s-o-bzy(r) "I cut that"; se s-o-laje(r) "I work".
In non-transitive plural dynamic verbs (I) and in singular non-transitive verbs with preverbs (II), this prefix is present in the forms of all three persons:






It is also a characteristic feature of dynamical verbs that the optional suffix -r is present. Cf. in the present tense se soc1ue/se socIue-r "I go", se ar sotkh/se ar sotkhy-r "I write that", se aby ar isot/se ar aby isoty-r "I give him that".

Static verbs, on the other hand, are characterized by the absence of the prefix o- (ue-) and by the presence of the copula-suffix -shh: se syshyt-sh-sh "I stand"; se sysstudent-sh "I am a student"; se ar si1e-sh "I have that". The suffix -shh is by origin the root suffix of the verb i-schy-schy-sch "that of them is". In the Past Perfect and Future Perfect tenses, dynamic verbs do not differ from static verbs. The forms of the above tenses are formed from participles by adding to them the copula-suffix -sh or the affix -t, e.g., the Past Perfect Form kIuashch 'that went' is formed from the Past Participle kIua 'went' and the copula-suffix -sh, that is, 'went is'. The remote past tense form kIuat "that went (then)" is formed from the same participle form kIua "went" and the past tense suffix -t, i.e. kIuat lit. "(i.e., the past tense suffix -t, i.e. Iuat literally means "went was.

Hence, the Past Perfect and Future Perfect forms of dynamic verbs with the suffix -sh are Static verbs by formation.

The number of primary static verbs is limited: se sy-schy-t-sch "I stand"; se sy-schy-schy-sch "I sit"; se sy-schy-l-l-sch "I lie down"; aby ar i-Iyg-sch "one holds that"; aby ar i-1e-sch "one has that" ("he has that"); I- shy-sh-shch "one (is) of them"; se aby sy-huy-shch "I want that"; aby ar fe-FI-shch "one desires that"; ar aby shy-g-shch "that is put on him".

Static verbs \textbf{щытын} "to stand", \textbf{щысын} "to sit", \textbf{щылъын} "to lie" may have various preverbs of local meaning, e.g:

\begin{xlist}
\ex \textbf{стэчаныр стIолым те-т-щ} \glossphonemics{stat⁀ʃanər stʼawɮəm tajtɕ} "glass stands on the table"
\ex \textbf{хьэр мэкъум те-лъ-щ} \glossphonemics{ħar makʷəm tajɬɕ} "dog sits on the hay"
\ex \textbf{къазыр бжэм Iy-т-щ} \glossphonemics{qaːzər bʒam ʔʷətɕ} "goose stands by the door"; 
\ex \textbf{мыIэрысэр жыгым пы-т-щ} \glossphonemics{məʔarəsar ʒəɣəm pətɕ}  "apple hangs on the tree"
\ex \textbf{щакIуэр мэ-зым хэ-т-щ} \glossphonemics{ɕaːkʷʼar mazəm xatɕ} "the hunter is (lit. stands) in the forest"
\ex \textbf{Iуэху мыублэ блэ хэ-с-щ} \glossphonemics{ʔʷaxʷ məwəbɮa bɮa xasɕ} "There is a snake in the unstarted work"
\ex \textbf{лIыжьыр унэм щIэ-т-щ} \glossphonemics{ɬʼəʑər wənam ɕʼatɕ} "the old man is in the house" (lit. "stands in the house")
\ex \textbf{Сабыр и щIагъ дыщэ щIэ-лъ-щ} \glossphonemics{saːbər jə ɕʼaʁ dəɕa ɕʼaɬɕ} "Under modesty lies gold"
\ex \textbf{жэмыр пщ1антIэм дэ-т-щ} \glossphonemics{ʒamər pɕʼantʼəm datɕ} "the cow stands in the yard".
\end{xlist}

Stative verbs may sometimes have preverbs of place. For example: \textbf{дыщэр къыщыщIахым щылъапIэщ} \glossphonemics{dəɕar qəɕəɕʼaːxəm ɕəɬaːpʼaɕ} "Where gold is mined, there (it is) expensive [is]"; \textbf{Дзыгъуэр и гъуэм щыхахуэщ} \glossphonemics{d⁀zəʁʷar jə ʁʷam ɕəxaːxʷaɕ} "And the mouse in its burrow (there) is brave (is)".

Static verbs, like dynamic verbs, can be both one-personal (ar shytsh "that one stands", ar pshaschaesh "that girl is") and two-personal (abi ar iyesh "that one has that", abi ar iyygsh "that one holds that"), as well as three-personal (schuiyygsh "that one holds that for me").

Static verbs are usually non-transitive, e.g.: ar shysshch "he sits"; ar shytshch "he stands"; aby ar iyesch "he has that"; ar aby huiysch "he wants that"; aby ar fifeysch "he likes that"; aby ar shygysch "he wears that"; aby ar i gugeysch "he seems that"; ar dakhasch "he is beautiful".

In Kabardian-Circassian there is also a transitive static verb: \textbf{абы ар иIыгъщ} \glossphonemics{aːbə aːr jəʔəʁɕ} "he holds that".

Dynamic verbs, on the other hand, may be both transitive (aby ar etkhy(r) "he writes that", aby ar eshe(r) "he leads him") and non-transitive (ar makIue(r) "he goes", ar aby yojye(r) "he waits for him").

Note. Static verbs in causative formation retain the form of stasis, e.g., the transitive verb shch-i-ge-t-shch, "he makes him stand", which retains the form of stasis, indicated by the absence of the dynamic prefix ue-(o-) and the presence of the copula-suffix -sh, characteristic of static verbs.
In some cases, the same stem may be used to form both Static and Dynamic verbs, e.g.: se syshytsh (stat.) 'I stand' - se syshyotyr (dynam.) 'I stand idle'. The verb "I stand idle"; "I sit", "I sit", "I sit"; "I lie", "I lie"; "I lie", "I lie".

From some names, we can also form both static and dynamic verbs, e.g.: pkhashchie "carpenter" - se sepkhashchieesh (stat.) "I carpenter", se sopkha-shhIer (dyn.) "I carpenter"; egyejakIue "teacher" - se syeggejakIueesh (stat. The verb "to teach" is to say "I teach"; bzaje "wicked" is to say "I am wicked" is to say "I become wicked"; dache "beautiful" is to say "I am beautiful" is to say "I become beautiful" is to say "I become beautiful".

Thus, static verbs differ from dynamic verbs in the system of conjugation as well as in the forms of word formation:

\begin{xlist}

\ex 1) Dynamic verbs here have the aorist form. The Aorist is represented by the simple base, without the characteristic of tense, and occurs:

on the one hand, together with the suffix \textbf{-р} and the union particle \textbf{-и}, when it is followed in a sentence by another verb, e.g.: \textbf{Хамэхьэр къихьэри унэхьэр ирихущ} \glossphonemics{xaːmaħar qəjħarəj wənaħar jərəjxʷəɕ} "The strange dog came and chased away the house dog", \textbf{къихьэри} \glossphonemics{qəjħarəj}. On the other hand, the Aorist is found with the copula-suffix \textbf{-щ}, for example:  \textbf{Хамэхьэр къихьэри унэхьэр ирихущ} \glossphonemics{xaːmaħar qəjħarəj wənaħar jərəjxʷəɕ} "The strange dog came and chased away the owner's dog". The Aorist is very rare in this form.

Static verbs cannot form the Aorist. This is common to all Iberian-Caucasian languages.

The Aorist of a dynamic verb coincides with the Present tense of a Static verb. For example, in the sentence: \textbf{сыщытщ, сыщытри сыкъэкIуэжащ} \glossphonemics{səɕətɕ səɕətrəj səqakʷʼaːɕ} "I stood, stood and went" syshytshch "I stood" is a dynamic verb in the Aorist (the present tense is syshotyr "I stand idle") and in se mbdezh syshytshch "I stand here" syshytshch is a static verb in the present tense.

\ex The past imperfect form of dynamic verbs, unlike the form of static verbs, is characterized by the presence of the facultative suffix -r. A dynamic verb in the past imperfective se sykIuet "I went" may have a parallel form with the suffix -r-: se sykIuert "I went"; \textbf{къэбэрдей жылэр Iейуэ гузавэрт} \glossphonemics{qabardaj ʒəɮar ʔejwa gʷəzaːvart} "the Kabardian people was very worried".

The suffix \textbf{-р-} in the past imperfect \textbf{сыкIуэрт} \glossphonemics{səkʷʼart} "I went" is the same optional present tense suffix of dynamic verbs: \textbf{сэ сокIуэ(р)} \glossphonemics{sa sawkʷʼa(r)} "I go".

Static verbs in the past tense do not have this facultative suffix \textbf{-р-}: \textbf{сэ сы-щытт} \glossphonemics{sa səɕətt} "I stood", \textbf{сэ сыщылът} \glossphonemics{sa səɕəɬt} "I lay", \textbf{сэ сыучителт} \glossphonemics{sa səwət⁀ʃəjtajɬt} "I was a teacher".

\ex Primary Static verbs are found only with preverbs, e.g., 
\begin{xlist}
\ex \textbf{Iy-т-щ} \glossphonemics{ʔʷətɕ} "he stands (near something)"; \textbf{Iy-c-щ} \glossphonemics{ʔʷəsɕ} "he sits (near something)"; \textbf{ly-лъ-щ} \glossphonemics{ʔʷəɬɕ} "he lies (near something)"; 
\ex \textbf{те-т-щ} \glossphonemics{tajtɕ} "he stands (on something)"; \textbf{те-с-щ} \glossphonemics{tajsɕ} "he sits (on something)"; \textbf{те-лъ-щ} \glossphonemics{tajɬɕ} "he lies (on something)".
\end{xlist}

Dynamic verbs, on the other hand, have both 

\begin{xlist}
\ex preverbs (\textbf{Iy-дэ-н} \glossphonemics{ʔʷədan} (\textbf{Iуедэ} \glossphonemics{ʔʷajda}) "to sew to something", \textbf{хэ-тхэ-н} \glossphonemics{xatxan} (\textbf{хетхэ} \glossphonemics{xajtxa}) "to write into something", \textbf{1у-хы-н} \glossphonemics{ʔʷəxən} (\textbf{Iуех} \glossphonemics{ʔʷajx}) "to open") 
\ex and no preverbs (\textbf{содэ} \glossphonemics{sawda} "I sew", \textbf{солажьэ} \glossphonemics{sawɮaːʑa} "I work", \textbf{сожэ} \glossphonemics{sawʒa} "I run").
\end{xlist}

\textbf{Note.} Some dynamic verbs that express motion, like the static ones, do not occur without preverbs. For example: 
\begin{xlist}
\ex \textbf{сы-д-о-кI} \glossphonemics{sədawkʼ} "I go out from somewhere", \textbf{сы-бл-о-кI} \glossphonemics{səbɮawkʼ} "I pass by something", \textbf{сы-щI-о-хьэ} \glossphonemics{səɕʼawħa} "I enter something", \textbf{й-о-хьэ} \glossphonemics{jawħa} "he enters something".
\end{xlist}

\ex 4) Primary static verbs are usually characterized by preverbs The primary static verbs are usually characterized by preverbs of local meaning, e.g., ly-t-sh "standing", te-t-sh "standing (on the surface of something)", sh-Ie-t-sh "standing under". With dynamic verbs, on the other hand, both local preverbs and preverbs of motion are used: sy-bl-o-kI "I pass (by)", ph-o-kI "passes (through)", sy-k-o-kIue "go here", sy-n-o-kIue "go there".

\textbf{Note.} Sometimes the preverbs of direction kъ- (kъ-) are also used with static verbs, but together with preverbs of place Iu-, he-, shy-: si gupemkIe kъ- shy-s-shy lit. "The following is an example of this: \textbf{щхьэгъубжэм къыIу-т-щ} \glossphonemics{ɕħaʁʷbʒam qəʔʷətɕ} lit. "standing by the window (here)".

\ex (5) In the modern language, the structure of the present participle of dynamic verbs is the same as that of the participle of static verbs. In both cases the base of the participle coincides with the base of the present verb: \textbf{кIуэ-р} \glossphonemics{kʷʼar} 'going' (the participle from the dynamic verb \textbf{со-кIуэ} \glossphonemics{sawkʷʼa} "I go"), \textbf{щыты-р} \glossphonemics{ɕətər} 'standing' (the participle from the static verb \textbf{сы-щыт-щ} \glossphonemics{səɕətɕ} "I stand"). But historically the forms of the Static and Dynamic Participles have been different. The present participle of dynamic verbs is formed with the help of the suffix \textbf{-рэ}: \textbf{кIуэ-рэ-р}, "going," \textbf{зы-тхы-рэ-р} \glossphonemics{zətxərar}, "writing. And participles from static verbs were formed without any suffix: \textbf{щытыр} "standing" and \textbf{щысыр} "sitting", that is, as they are formed now: \textbf{Псым Iусым икIып1э ещIэ} \glossphonemics{psam ʔʷəsəm jəkʼəpʼa jaɕʼa} "He who lives (sitting) by the river, knows the ford".

The participles of dynamic verbs with the suffix \textbf{-рэ} are preserved in the interrogative forms of the present tense: \textbf{уэ у-кIуэ-рэ?} \glossphonemics{wa wəkʷʼara} "are you going?" (lit. "are you going?"), \textbf{уэ п-тхы-рэ?} \glossphonemics{wa ptxəra} "do you write?" (lit. "are you writing?").

The present tense question forms of static verbs, on the other hand, do not have the -re suffix: ue ushyt? "are you standing? (lit., "are you standing"); üe ustudent? "are you a student?" etc.
\end{xlist}

\subsection{Finite and infinitive verb forms}
\subsection{Face category}
\subsection{Verb paradigms}
\subsubsection{1-personal intransitive verbs}
\paragraph{Preliminary remarks}
The singular dynamic verbs are simple (primary) and derivative (secondary). Cf. e.g.: simple verbs - kIue-n "to go", kab. zhe-n "to run", derivative verbs - k'e-kIue-n "to go here", kIue-fy-n "to manage to go", kab. kIue-zhy-n "to go back".
Singular primary static verbs include, as a rule, a local prefix: shy-sy-n 'to sit', shy-l'y-n 'to lie'. Most of the local preverbs, as well as the affixes of jointness, union and version, translate the stems of homonymic verbs into the stems of bipersonal non-transitive verbs.
The specificity of singular non-transitive verbs is the lack of a coherent and consistent conjugation system. The antithesis of the person forms is realized only in the 1st and 2nd persons. As for the third person, the antithesis of the person forms is either removed by the affix of number. Since single-person non-transitive verbs can only be changed by the person of the subject, the absence of face opposition in the 3rd person indicates a significant destruction of the personal conjugation in the Adyghe languages.
\subsubsection{2-personal intransitive verbs}
\paragraph{Ways to turn singular non-transitive verbs into bifurcated non-transitive verbs}
\begin{xlist}
\ex Conversion, i.e., the formation of a paradigm of two-person non-transitive verbs without special affixes. Cf.: 
	\begin{xlist}
	\ex \textbf{дэIуэн} \glossphonemics{daʔʷan} \textbf{мэ-даIуэ} \glossphonemics{madaːʔʷa} "he listens" becomes  \textbf{й-о-даIуэ} \glossphonemics{jawdaːʔʷa} "he listens to someone
	\ex \textbf{ар ма-лъэ} "he jumps" \glossphonemics{aːr maːɬa} becomes \textbf{ар абы й-о-лъэ} \glossphonemics{aːr aːbə jawɬa} "he jumps over something"
	\ex \textbf{хъуэпсэн} \glossphonemics{χʷapsan} \textbf{ар мэ-хъуапсэ} \glossphonemics{aːr maχʷaːpsa} "he envies" becomes \textbf{ар абы йо-хъуапсэ} \glossphonemics{aːr aːbə jawχʷaːpsa} "he envies him"
	\end{xlist}

\ex Vowel alternation. As a means of transforming singular non-transitive verbs into bifurcated non-transitive verbs, ablaut alternation is widespread, covering a wide variety of bases in their morphemic structure. Cf.: 
	\begin{xlist}
	\ex \textbf{ар мэ-лъаIуэ} "he asks", \textbf{ар абы йо-лъэIу} "he asks him",
	\ex \textbf{ар ма-плъэ} "he looks at someone, something",
	\ex \textbf{ар ма-бгэ} "he curses", \textbf{ар абы йо-бг} "he curses him",
	\ex \textbf{ар мэ-банэ} "he struggles", \textbf{ар абы йо-бэн} "he struggles with him".
	\end{xlist}



\ex The directional suffix -kI(s). Cf.: ma-kIue, "he goes", ar aby yo-kIuekI "he bypasses someone, something", ar ma-ge "he runs", ar aby yo-gekI "he rounds someone, something", ar me-page "he boasts", ar aby yo-pagekI "he treats him boastfully, arrogantly".

\ex Morphemic element -kI(s)+vowel alternation. When combining morphological and morphological means, two types of ablaut alternations are noted: the change of the etymological vowel y- into vowel e- and, conversely, the change of the etymological vowel e- into vowel y-.

\ex Morphemic element -kI(s) + alternation y-e: tIysy-n "to sit", e-tIyse-kIy-n "to sit around something, someone"; pshy-n "to creep", e-pshye-kIy-n "to creep around sth. "Uyshyn 'to trot', e-uyshe-kIyn 'to go, trot towards something'; ukIuryyn 'to fall', eukIureikIyn 'to fall towards something'.

\ex Morphemic element -kI(s) + alternation e-i. Cf. lIe-n "to die", i-lIy-kIy-n "to die of something". The prefix i- in the form i-lIy-kIy-n goes back to the local preverb i- 'in', 'inside'. Cf. also ar ma-tkh'e "he lives to be well-to-do", ar aby ho-tkh'i-kI "he lives to be well-to-do from something".

\ex The suffix -lI(e). Cf. kIue-n "to go", e-kIue-lIe-n "to approach"; zhe-n "to run", e-zhe-lIe-n "to run up"; tIysyn "to sit", e-tIysyn-lIe-n "to sit". In the Present and Aorist, bases ending in vowel e form the extended stage of alternation e-a. Cf.: so-kIua-lIe "I approach", uo-kIua-lIe "you approach", yo-kIua-lIe "he approaches", e-kIua-lIeri... "approached and...". Cf. the extended degree of alternation in forms like e-kIua-lIe "come up", e-kIua-lIeme "if he comes up".

\ex The root element -h(s). Cf.: ar ma-jeh "he runs", ar aby yo-jeh "he runs (escapes) from something"; ar mee-psht "he crawls", ar aby yo-pshy-x "he slides off someone or something".

\ex Root element -h(s)+vowel alternation. Cf.: ar ma-l'e "he jumps", ar aby yo-l'y-h "he jumps from something (down)".

\ex Local preverbs. Cf: shy-sy-n "to sit" gue-sy-n "to sit near, at the side of someone, something", te-sy-n "to sit by someone, something", bg'e-de-sy-n "to sit next to someone".

\ex Local preverbs + alternation of vowels. Cf.: zhe- "to run away", gye-zhy-n, "to run away", bg'ede-zhy-n "to run away from someone-, something-".

\ex Version prefixes. Cf.: kIue-n "to go", hue-kIue-n "to go for sb.", fIe-kIue-n "to go against the wish of sb.

\ex Prefix of union. Cf.: so-keIue "I go," se-do-keIue "I go with him."

Prefix de + suffix -i. Cf.: kIue-n "to go," de-kIue-i-n "to go up to something."

\end{xlist}

The above material shows that in all cases of verb face conversion, a one-person non-transitive verb becomes a two-person non-transitive verb. As will be seen below, some of the methods described are used to convert two-person non-transitive verbs into three-person transitive verbs, two-person transitive verbs into three-person transitive verbs, and three-person transitive verbs into four-person transitive verbs.
\subsubsection{3-personal intransitive verbs}
\paragraph{Types of three-person non-transitive verbs}
Three-person non-transitive verbs are derivatives. They are formed from non-transitive verbs by means of affixes of complicity and version. According to their structure, tributary non-transitive verbs are of several types.

A special type may include verbs formed from two-person non-transitive verbs. Cf. e.g., s-e-pl-pl-asch "I looked at s-, sth.", s-s-b-d-e-pl-pl-asch "I looked at s-, sth. with you", s-s-hu-e-pl-asch "I looked at s-, sth. for you", s-s-p-fI-e-pl-asch "I looked at s-, sth. in spite of you". While the two-person non-transitive verb changes in the persons of the subject and the indirect object, the three-person non-transitive verb changes in the persons of the two indirect objects as well. In addition to the person of the indirect object to which the action is directed, a three-person non-transitive verb also expresses the person of a partner in the action or versioned action.

The other type is made up of verbs that include the affixes accessory and version. These verbs do not express the object to which the action is directed. According to the meaning of the derivation suffixes, verbs of this type express the subject of the action, the accomplice of the action and the person of the versioned action. Cf.: u-a-fIy-de-kIu-a-shch 'you went with him against their will', u-s-fIy-de-kIu-a-shch 'you went with them against my will'. The verbs of that type are based on the infinitive bases of two-person non-transitive verbs, such as bg'edeh-he-he-n 'to come near'. Cf.: uy-s-hu-bg'edy-hy-a-sh "you came up to him for my sake".

The differences between these types of ternary non-transitive verbs concern not only the nature of the derivative base but also the composition of the indirect objects. As we shall see below, the ternary verb types under consideration differ considerably also in the construction of paradigmatic series.
\subsubsection{4-personal intransitive verbs}
\subsubsection{1-personal transitive verbs}
\subsubsection{2-personal transitive verbs}
\subsubsection{3-personal transitive verbs}
\paragraph{Simple three-person transitive verbs}
There are few ternary transitive verbs with simple (non-derivative) bases. These include \textbf{е-ты-н} \glossphonemics{jatən} "to give something, sb. to someone", \textbf{е-щэ-н} \glossphonemics{jaɕan} "to sell something to someone", etc. The prefix \textbf{е-} in these verbs expresses the indirect object of the 3rd person singular. The morphemes of a simple three-person transitive verb are distributed in the following order: direct object prefix + indirect object prefix + subject prefix + root morpheme. Cf. \textbf{у-е-с-т-а-щ} \glossphonemics{wajstaːɕ}, "I gave you to him.

Not all personal forms are formed from simple ternary verbs. For example, the personal form \textbf{фэ-с-т-а-щ} \glossphonemics{fastaːɕ} 'I gave you something' is formed from a verb with a simple stem; the personal form meaning 'he gave you something' is possible only with the direction prefix: \textbf{къы-в-и-та-щ} \glossphonemics{qəvəjtaːɕ} 'he gave you something'. Cf.: \textbf{уэ-с-т-а-щ} \glossphonemics{wastaːɕ} "I gave you something"; \textbf{е-с-т-а-щ} \glossphonemics{jastaːɕ} "I gave him something", but \textbf{къ-у-и-т-а-щ} \glossphonemics{qəwəjtaːɕ} "he gave you something", \textbf{къы-з-и-т-а-щ} \glossphonemics{qəzəjtaːɕ} "he gave me something".

Not all personal forms of three-person transitive verbs are formally marked. The subject in all persons, the 1st and 2nd person direct and indirect object are consistently expressed with positive affixes; the 3rd person direct object is expressed with a null affix. The form of expression of the 3rd person indirect object is very specific. Its peculiarity is determined by the tense form, base structure and morphological position.

All paradigmatic series with a 3rd person direct object are extra-paradigmatic and unmarked. Cf.: \textbf{естащ} \glossphonemics{jastaːɕ} 1) "I gave it to him", 1) "I gave them to him"; \textbf{ептащ} \glossphonemics{japtaːɕ} 1) "you gave it to him"; 2) "you gave them to him"; \textbf{иритащ} \glossphonemics{jərəjtaːɕ} 1) "he gave it to him", 2) "he gave them to him"; \textbf{къызитащ} \glossphonemics{qəzəjtaːɕ} 1) "he gave it to me", 2) "he gave it to me"; \textbf{къуитащ} \glossphonemics{qəwətaːɕ} 1) "he gave it to you", 2) "he gave it to you".

The indirect object of the 3rd person singular is expressed by the null affix in the personal present forms of simple verbs. Cf.: \textbf{сы-ре-т} \glossphonemics{sərajt} "he gives me to him," but \textbf{и-зо-т} \glossphonemics{jəzawt} "I give him to him/them". In the same position, the indirect object of the 3rd person plural is expressed differently, depending on the tense and person of the subject. In the personal form with 1st and 2nd person subject affixes, the 3rd person indirect object of the 3rd person plural, like the 3rd person indirect object of the 3rd person singular, is expressed by zero. This creates unmarked forms such as \textbf{у-зо-т} \glossphonemics{wəzawt} 1) "I give you to him," 2) "I give you to them"; \textbf{фы-зо-т} \glossphonemics{fəzawt} 1) "I give you all to him," 2) "I give you all to them." Moreover, personal unmarked forms like \textbf{узот} \glossphonemics{wəzawt} "I give you to him/them" are phonetically identical with the personal forms of the other paradigmatic series. Cf.: \textbf{узот} \glossphonemics{wəzawt} "I give him (them) you", uzot "we give you him (them)"; \textbf{фыдот} \glossphonemics{fədawt} "we give you all to him/them", \textbf{фыдот} \glossphonemics{fədawt} "we give him/them to you all", where segments y-, fy- include the function of the direct object personal affix in some formations and the function of the indirect object personal affix in other formations.
\paragraph{Paradigms of three-person transitive verbs like yeth-n "to give him whom-, sth.}
Verbs with simple bases form an incomplete paradigm of personal forms. The directional prefix is used to form a complete paradigm. Verbs with the direction prefix are derivatives. There are special derivational affixes used to transform bipersonal verbs into ternary ones. Among them, the suffix -lie is productive for the formation of ternary transitive verbs from correlative bipersonal verbs. Cf. e-she-lIe-n "to bring sb./sth. to sb./sth.", e-dzy-lIe-n "to plant sb./sth. to sb./sth.". The complicit prefixes de-, version fie-, causative g'e- also produce ternary verbs from bipersonal verbs. Cf. dae-hyi-n "to carry someone, something with someone", fieh-hyi-n "to carry someone, something against the will of someone", e-g'eh-hyi-n "to force someone, something to carry someone".
\subsubsection{4-personal transitive verbs}
\paragraph{Types of three-person non-transitive verbs}
\subsection{Tense category}
\subsubsection{Preliminary Remarks}
The temporal forms of the Adyghe languages are analyzed separately not only in descriptive grammars, but also in special studies1. At the same time, not to mention the diachronic aspect of the problem, many questions of the synchronic description of the category of time remain underdeveloped. There is no consensus among researchers even on the question of the number of temporal forms in the Adyghe languages.

In recent descriptive grammars of the Kabardian-Circassian language the concept dominates, according to which two groups of tenses are distinguished. The first group includes tenses expressing the relation of the time of action to the moment of speech, the second group includes tenses showing the relation of the time of action to a certain moment in the past. In accordance with this there are distinguished the present first, the present second, the past first, the past second, the future first, the future second. The morphological indicator of the second group of tenses is considered to be the formant -t2.

The classification of tenses is also based on contrasting the -t-forms with the tenses of another group. At the same time, such a principle of classification remains disputable. The grouping of the tenses according to the above principle presupposes the uniqueness of the -t formative in all the tenses, which is necessary for the formation of the temporal opposition in the indicative by the differential sign - form on -t: form without -t. Meanwhile, the analysis of the material shows that the formative -t is ambiguous in combination with bases of different tenses. Cf.: kIuert "he was going then", but kIuenut "he would like to go". In other words, not all the tenses with the formative -t have an indicative meaning. If forms kIuert, kIuat, kIuegyat are indicative (indicative mood), kIuent, kIuenut have the meaning of conjunctive (subjunctive mood).

According to the classification under consideration, kIuert "he was going then" is a form of the present tense, because it represents an action in its course in the past. Not mentioning the fact that expressing an action (process) in its course (flowing) in the past is a function of the imperfect, from a grammatical point of view the kIuert falls into the paradigmatic series of the past tense. Cf. e.g:
.......

As may be seen, the paradigmatic series (a) differs from the paradigmatic series (b) and (c), which are absolutely identical in this respect, in the form of the 3rd person and the glossation of the 1st and 2nd person affixes. Series (c) coincides with other past tense forms in the glossing of the root morpheme. Cf.: \textbf{сы-кIуэ-рт} \glossphonemics{səkʷʼart} "I went back then", \textbf{сы-кIуэ-щ} \glossphonemics{səkʷʼaɕ} "I went" (Aorist), \textbf{сы-кIуэ-ри} \glossphonemics{səkʷʼarəj} '"I went and ...". (aorist). A strong proof that formations of the kIue(r)t type "he (then) walked (then)" are imperfect is their functional identity with the Adygean imperfect in \textbf{-щтыгъэ} \glossphonemics{-ɕtəʁa}.

The question of classifying forms of tenses is less debatable with respect to the Adygean language, although even here there is no unanimity among specialists.
\subsubsection{Present Tense}
\subsubsection{Future 1}
The characteristic of the base of Future I is the formant \textbf{-н}. The affirmative form of future I is marked with the suffix \textbf{-щ}. Cf.: Adyg. \textbf{сы-кIуэ-н}, kab. \textbf{сы-кIуэ-н-щ} 'I will go', Adyg. \textbf{схьы-н}, kab. \textbf{схьы-н-щ} 'I will carry'. Examples:

\textbf{Мурад щэхуу уэ блэжьынур Мис ныщхьэбэ къэсхутэнщ} \glossphonemics{məwraːd ɕaxʷəw wa bɮaʑənəwr məjs nəɕħaba qasxʷətanɕ} "What you have planned to carry out secretly, I will find out tonight."

In two dialects, Kuban and Beslaneev, the affirmative form lacks the suffix \textbf{-щ}. Future I in Beslenevi dialect (as well as in Adygean) is marked by the formant \textbf{-н} and in the Kuban dialect by the formant \textbf{-нэ}. Cf.: Besl. \textbf{сщIын} \glossphonemics{sɕʼən}, Kub. \textbf{сщIынэ} \glossphonemics{sɕʼəna} "I will do it".

Future I was formed during the era of common Adyghe language unity. The common Adyghe form of the future I is preserved in Adygean and Beslenev dialect. The variant of the suffix future I with the vowel \textbf{э}, typical for the Kuban dialect, is a is an innovation that appeared after dialectal differentiation of Kabardian. The suffix \textbf{-sh} in the affirmative form of dialects is also a new formation.

There is still no satisfactory explanation in the specialized literature for the genesis of the common Adyghe suffix of future I \textbf{-н}. There is no reason to agree with the opinion of N.F. Yakovlev and D.A. Ashkhamaf, according to which the suffix in question genetically ascends to the common Adyghe word \textbf{нэ} "eye", "hole". At the present stage of the study of Adyg languages does not seem possible to solve the question of the origin of common Adyghe formant of future I \textbf{-н}. It may be assumed that the \textbf{-н} form is not finitish, but infinitive, with a meaning of intention, purpose or intention to be. It is in this function that the form is widely used in the modern Adyghe languages. In this connection, attention should also be paid to the Ubykh language suffix \textbf{-н} in the future purpose and intention. Cf. ubykh. \textbf{айнащаутын} \glossphonemics{ajnaɕawtən} "(they) to do".
\subsubsection{Future 2}
Future II is formed from the base of Future I with the suffix \textbf{-у}. The affirmative form of Future II is marked by the suffix \textbf{-щ}. Cf.: \textbf{сы-кIуэ-ну-щ} \glossphonemics{səkʷʼanəwɕ} "I will go", \textbf{с-хьы-ну-щ} \glossphonemics{sħənəwɕ} 'I will carry'.

It is typical that there is no paradigmatic parallelism between future I and future II. Cf. e.g. in Kabardinian the absence of the infinitive (union) form of the future II in the indicative: \textbf{сыкIуэнщи} \glossphonemics{səkʷʼanɕəj} "I will go and", but \textbf{сыкIуэнущи} \glossphonemics{səkʷʼanəwɕəj} "as I will go". In other words, the opposition of the future I to the future II is not realized in all the forms, which also testifies to the appearance of this temporal opposition in a later era of the development of the Adyg languages.

Thus, future II, unlike future I, cannot be attributed to the all-Adyg languages unity. At the same time, the origin of the formants of Future II remains unclear, despite the fact that the temporal form itself developed after the differentiation of the basic language into separate dialects.
\subsubsection{Perfect 1}
The base of the Perfect I is formed with the suffix \textbf{-а}. Cf.: \textbf{зд-а-щ} \glossphonemics{zdaːɕ} "I sewed", \textbf{сыщыс-а-щ} \glossphonemics{səɕəsaːɕ} "I sat".

The perfective suffix \textbf{-а} is an innovation, arisen in the individual development of Adyghe languages on the basis of the original \textbf{-гъэ}. The latter goes back to the common Adyghe language unity.

The original perfective suffix \textbf{-гъэ} can be traced in formations such as \textbf{тхыгъэ} \glossphonemics{txəʁa} "writing" , \textbf{тыгъэ} \glossphonemics{təʁa} "gift", \textbf{бжыгъэ} \glossphonemics{bʒəʁa} "number", \textbf{къэкIыгъэ} \glossphonemics{qat⁀ʃʼəʁa} "plant". The latter are a perfective form of the participle1. The formation of the perfect is accompanied by phonetic changes in the stem. Final etymological vowels \textbf{э}, \textbf{ы} are absorbed by a long vowel a, which serves as a perfective suffix. Cf.: \textbf{зд-а-щ} <- \textbf{*зды-а-щ} "I sewed", \textbf{сы-кIу-а-щ} <- \textbf{*сы-кIуэ-а-щ} "I walked".

The question about the genesis of the perfective suffix \textbf{-гъэ} remains unclear. Neither R. Erkert, who attributed \textbf{-гъэ} to the adverb \textbf{дыгъуасэ} , nor N.F. Yakovlev or D.A. Ashkhamaf, who identified it with the base \textbf{гъэ} 'year' , gave any convincing explanation of the etymology of the suffix in question. Dumezil's close connection of the common Adyghe perfection suffix \textbf{-гъэ} with Ubykh perfection suffix \textbf{-къа} and Abkhazian pluperfect suffix \textbf{-хъа} is not without interest, although there is not enough material for genetic unity of these suffixes yet. Cf. Adyg. \textbf{сыкIуа-гъ}, Ubykh. \textbf{сыкIьа-къа,}, Abkh. \textbf{сцахъейгпI} \glossphonemics{st⁀saχʲejgpʼ} "I went".
\subsubsection{Perfect 2}
Perfect II is formed from Perfect I by means of the suffix \textbf{-т}, indicating that the action took place in a limited time. Cf. \textbf{сыкIуащ} "I went", \textbf{сыкIуат} "I went then".

\textbf{Нартхэр мы щIыпIэм исат, Нарт Сосрыкъуэ и джатэр А зэман жыжьэм щыбзат} \glossphonemics{naːrtxar mə ɕʼəpʼam jəsaːt naːrt sawsrəqʷa jə d⁀ʒaːtar aː zamaːn ʒəʑam ɕəbzaːt} "Narts lived in this land, Sosryko (lit. Sosryko's saber) performed feats of arms here in those distant times"; \textbf{ЩIэсэныгъэм гур щигъэбзэрабзэм ПцIы зыхэмылъ усэхэр бжесIат} \glossphonemics{ɕʼasanəʁam gʷər ɕəjʁabzaraːbzam pt⁀sʼə zəxaməɬ wəsaxar bʒajsʼaːt} "When my heart rejoiced with love, I read you sincere verses".

Perfekt II is a Kabardian new formation.
\subsubsection{Plusquamperfect 1}
The base of pluperfect I includes \textbf{-гъа} \glossphonemics{-ʁaː}. Cf. e.g:
\textbf{Нобэ хуэдэу адыгэкъуэр пшэм пхылъэту щытыгъакъым} \glossphonemics{nawba xʷadaw adəɣaqʷar pʃam pxəɬatəw ɕətəʁaːqəm} "In the olden days, the son of an Adyghe did not soar in the clouds".
The composition \textbf{-гъа} \glossphonemics{-ʁaː} in the pluperfect is unusual. It is usually seen as a single suffix. In this case, the pluuscamperfect I should be historically identified with the perfect I and chronologically be dated to the period of the individual development of the Adyghe languages, which is unlikely.

There is reason to believe that the pluperfect I was formed during the era of common Adyghe language unity. Cab. -Apparently, -g'a consists of two perfective suffixes - old (initial) -gъ (← gъe) and new (Kabardian) -a. The initial suffix for -g'a, as well as for Adyg. -gъa-gъ(e), is -gъe-gъe.

But why of the two perfective suffixes, by means of which pluperfect I is formed, is one presented in the original form, and the other is changed into -a? The order of -gъ-/-а indicates that the formation of the pluperfect I chronologically precedes the change of the suffix -gъe into -a. Otherwise, one would have to admit the possibility of forming the pluperfect by incorporating the suffix -gъe into the perfective: sykIu-à-sh "I went".

In the position before the long vowel a, derived from the second perfective g'e, the etymological voicing of the first perfective suffix g'e is lost. Cf. sykIue-g'a-sch *←* sykIue-g'e-g'e "I went then". The stability of the first suffix, while the second suffix changed to -a, is explained by the desire of the language to preserve the opposition of pluperfect I to perfection I: changing also the first -gъ to -a would lead to the homonymy of pluperfect and perfection, i.e. to the loss of one of these forms.
\subsubsection{Plusquamperfect 2}
\subsubsection{Aorist}
The question of the Aorist in the Adyghe languages is still insufficiently studied. The Aorist has a number of specific features, but not all its forms have been identified, not all constructions, syntactically connected with the Aorist, have been described.

There are different points of view on the question of the Aorist. Not all researchers recognize the existence of the aorist form of the verb. Thus, N. F. Yakovlev and D. A. Ashkhamaf do not distinguish the Aorist in the verb system. Other researchers, while singling out the Aorist as a special temporal form of the verb, differently explain its essence and place in the verb system. The concept of Aorist is understood by researchers as a phenomenon that is genetically unrelated to each other. The aorist form of the verb is considered to be \textbf{къихьэри} "he came in and", \textbf{ирихущ} "he drove out" (cf. \textbf{Хамэхьэ къихьэри унэхьэр ирихущ} "The strange dog came and drove the master dog away", \textbf{къыплъысыт} \glossphonemics{qəpɬəsət} "you received" (cf. \textbf{Хэт и лей къыплъысыт} \glossphonemics{xat jə ɮaj qəpɬəsət} "From whom you suffered").

There is a widespread view in the literature, according to which the Aorist supposedly represents a "pure" base. The position on the identity of the Aorist and the "pure" base of the verb contains some contradictions: first, the synchronic analysis distinguishes not one but several varieties of the Aorist, and second, the Aorist functions as a personal form, not as a verb base.

Of all the forms considered by various authors as aorist, it seems It is possible to classify to the Aorist only the forms of the Irikhushch type 'he drove out' and the infinitive (union) forms of the KIueri type 'he went and'.

The form of the irihushch 'he drove out' is widespread in the poetic speech of oral folklore. Cf. e.g:
\textbf{Уей, хъэзэрыдзэр дзэшхуает, дзэ фIыцIэт, Хы фIыцIэжьым къытехьэщ, къисыкIщ, ПсыжьыкIэкIэ къигъазэщ, къибзыхьщ, Тыхьаер хэку гъунэм къыщищIщ, НартыщIым щыятэщ, щытIысщ} \glossphonemics{waj, χazarədzar dzaʃxʷajat, dza fʼətsʼat, xə fʼətsʼaʑəm qətajħaɕ, qəjsətʃʼɕ, psəʑətʃʼatʃʼa qəjʁaːzaɕ, qəjbzəħɕ, təħaːjar xakʷ ʁʷənam qəɕəjɕʼɕ, nartəɕʼəm ɕəjaːtaɕ, ɕətʼəsɕ} "The Khazar army was large, terrible, it entered the Black Sea, crossed it, moved to the lower reaches of the Kuban, strengthened there, made a massacre in the region, rampaged in the land of the Narts, stopped there."; \textbf{Хы гъунэмкIэ сыщекIуэкIкIэ, ПщIэгъуэлэжьи сыкъыхуэзэщ. А пщIэгъуалэми сыкъэшэсри Хы гъунэмкIи сырежэкIыу Зы джатэжьи къэзгъуэтщ} \glossphonemics{xə ʁʷənamt⁀ʃʼa səɕajkʷʼat⁀ʃʼt⁀ʃʼa, pɕʼaʁʷaɮaʑəj səqəxʷazaɕ. aː pɕʼaʁʷaːɮaməj səqaʃasrəj xə ʁʷənamt⁀ʃʼəj sərajʒat⁀ʃʼəw zə d⁀ʒataʑəj qazʁʷatɕ} "When I was walking along the seashore, I met an old (old) horse of grey color; I sat on it and, riding along the shore, found an old (worn-out) saber."
The discriminated form of the Aorist is often used in the language of poets who continue the traditions of oral folklore. B. Pachev often used this form. Cf. e.g:

\textbf{Гужьеигъуэшхуэм укъыбгъэдашри, Уи анэжь тхьэмыщкIэм укъыхуашэжщ} \glossphonemics{gʷəʑajjəʁʷaʃxʷam wəqəbʁadaːʃrəj, wəj aːnaʑ tħaməɕt⁀ʃam wəqəxʷaːʃaʒɕ} "You were led away from great misfortune and brought to your old mother"; \textbf{КIэрахъуэ фIыцIэр щэ зэтебгъэуащ, Зэ уэгъуэ закъуэти Алихъаныр епсыхыжщ} \glossphonemics{t⁀ʃaraːχʷa fʼət⁀sar ɕa zatajbʁawaːɕ, za waʁʷa zaːqʷatəj aːɮəjχaːnər japsəxəʒɕ} "You shot three times with a black gun, one shot killed Alikhan".

The aorist form under consideration is used in conjunctive constructions in combination with an allied aorist or an allied word. There are several types of constructions that include a connected aorist like sykIuesh "I went".

1) Constructions like adyg. сытхэ-сытхи сыуцужьыгъ \glossphonemics{sətxa sətxəj səwət⁀sʷəʑəʁ}, kab. \textbf{сытхэщ-сытхэри сыувыIэжащ} \glossphonemics{sətxaɕ sətxarəj səwəvəʔaʒaɕ} "I wrote, wrote and stopped (writing)", kyikIuhshch-kyikIuhri kobekIuelIezhashch "he went, walked and returned" include a second (union) aorist. The latter is used not only in the above syntactic environment (about which see below).

2) The construction of the type \textbf{сыкIуэщ аби къэсхьащ} \glossphonemics{səkʷʼaɕ abəj qasħaːɕ} "I went and brought" is most often used in lively colloquial speech. The combination "the aorist of the sykIueshch type + the connecting union abi" is the functional equivalent of the aorist of the sykIueri type "went and". Cf. the synonymic constructions sykIuesh abi k'eshyashch = sykIueri k'eshyashch "I went and brought"; sytIysh abi sthashch = sytIyssri sthashch "I sat down and wrote"; unem shchIyh esh abi tIsasch = unem shchIyh eri tIsasch "he went into the room and sat down".
\subsection{Inflectional category}
\subsection{Negative forms}
\subsection{Interrogative forms}
\subsection{Ways of Verbal Formation}
\subsection{Factitive Verbs}
\subsection{Causative category}
\subsection{Unionality category}
\subsection{Jointness category}
\subsection{reciprocity}
\subsection{Version category}
\subsection{Category potency (possibility)}
\subsection{Category involuntariness}
\subsection{Directional and local prefixes (preverbs)}
\subsubsection{Preliminary remarks}
Kabardian-Circassian is characterized by the formation of verbal bases with preverbs expressing local and directive meanings. As in the other Abkhazian-Adygian languages, Kabardian-Circassian has a large number of local and directive preverbs. But in contrast to Abkhazian and Abaza, preverbs - bases that function in some cases as preverbs, and in other cases as bases - as carriers of verb's main lexical meaning - are less prevalent here.

A peculiarity of the Kabardian-Circassian language is that the expression of place is maximally differentiated, for example: \textbf{тепщэчым и-лъы-н} \glossphonemics{tajpɕat⁀ɕəm jəɬən} "to lie in a plate": \textbf{шкафым дэ-лъы-н} \glossphonemics{ʃkaːfəm daɬən} "to lie in a cupboard"; \textbf{дагъэм хэ-лъын} \glossphonemics{daːʁam xaɬən} "to lie in oil"; \textbf{унэм щIэ-лъы-н} \glossphonemics{wənam ɕʼaɬən} "to lie in a room"; \textbf{шхыIэным кIуэцIы-лъы-н} \glossphonemics{ʃxəʔanəm kʷʼat⁀sʼəɬən} "to lie in a blanket".

These examples show that the position within different objects (\textbf{тепщэч} \glossphonemics{tajpɕat⁀ɕ} "plate", \textbf{шкаф} \glossphonemics{ʃkaːf} "cupboard", \textbf{дагъэ} \glossphonemics{daːʁa} "oil", \textbf{унэ} \glossphonemics{wəna} "room", \textbf{шхыIэн} \glossphonemics{ʃxəʔan} "blanket") is expressed by different preverbs.

Etymologically, preverbs are divided into primary and secondary preverbs. The group of primary preverbs includes preverbs that cannot be connected with concrete bases of autonomous or service words. These include, for instance, preverbs \textbf{дэ-} \glossphonemics{da-}, \textbf{хэ-} \glossphonemics{xa-}, \textbf{и-} \glossphonemics{jə-}, \textbf{т-} \glossphonemics{t-}. The second group of preverbs is made up of preverbs that can be related to units functioning on their own, as noun bases. These include such preverbs as \textbf{блэ-} \glossphonemics{bɮa-}, \textbf{пэ-} \glossphonemics{pa-}, \textbf{лъы-} \glossphonemics{ɬə-}, \textbf{гуэ-} \glossphonemics{gʷa-}, \textbf{кIуэцIы-} \glossphonemics{kʷʼat⁀sʼə-}.

From the standpoint of morphemic structure, preverbs may be divided into simple (non-derivative) and compound. The group of simple preverbs includes all primitive and many secondary preverbs, cf. \textbf{дэ-} \glossphonemics{da-}, \textbf{хэ-} \glossphonemics{xa-}, \textbf{и-} \glossphonemics{jə-}, \textbf{т-} \glossphonemics{t-}, \textbf{блэ-} \glossphonemics{bɮa-}, \textbf{пэ-} \glossphonemics{pa-}, \textbf{лъы-} \glossphonemics{ɬə-}, \textbf{гуэ-} \glossphonemics{gʷa-}. The group of compound preverbs is formed by preverbs of the type \textbf{Iэпы-} \glossphonemics{ʔapə-}, \textbf{щхьэщы-} \glossphonemics{ɕħamə-}, \textbf{бгъэдэ-} \glossphonemics{bʁada-}, \textbf{кIэры-} \glossphonemics{t⁀ʃarə-}, \textbf{пхыры-} \glossphonemics{pxərə-}.

In the distributive sense, that is, in terms of the relationship of the preverb with the root morpheme of the verb, preverbs are also divided into two groups. The local preverbs constitute the third order and the directive preverbs the sixth order (see the table of the arrangement of verbal derivational morphemes on page 246).
\subsubsection{On the basic form of the preverb}
The phonetic composition of stemming morphemes, including preverbs, is characterized by the fact that they begin with a consonant or semivowel phoneme. Depending on the phonomorphological environment, the variant form of the stem morpheme changes, with the vowel element undergoing phonetic changes. Thus, in the word forms \textbf{хэ-дзэ-н} \glossphonemics{xad⁀zan} 'to throw something at something', \textbf{хы-з-о-дзэ} \glossphonemics{xəzawd⁀za} 'I throw something at something', \textbf{х-и-дз-а-щ} 'he threw something at something' the elements \textbf{хэ-} \glossphonemics{xa-}, \textbf{хы-} \glossphonemics{xə-}, \textbf{х-} \glossphonemics{x-} are variants of the same morpheme. Each of them occurs in a setting in which the other variants do not occur. In other words, the elements \textbf{хэ-} \glossphonemics{xa-}, \textbf{хы-} \glossphonemics{xə-}, \textbf{х-} \glossphonemics{x-} are in relation to a complementary distribution. When describing preverbs, it is difficult to list all their variant forms every time, although they are equivalent in terms of synchronic analysis. Therefore, it is customary to take one of the variants as the basic one for the preverb notation. The main form of the variants \textbf{хэ-} \glossphonemics{xa-}, \textbf{хы-} \glossphonemics{xə-}, \textbf{х-} \glossphonemics{x-} is considered to be the variant \textbf{хэ-} \glossphonemics{xa-}. The choice is justified not only by the frequency of its use but also by considerations of a historical character. \textbf{хэ-} \glossphonemics{xa-} is found in the Masdara, and in the Past and Future Forms, from the simple bases; \textbf{хы-} \glossphonemics{xə-}, \textbf{х-} \glossphonemics{x-}, historically go back to \textbf{хэ-} \glossphonemics{xa-}.

The choice of the principal variants of the other prefixes is decided accordingly.
\subsubsection{Directional preverbs}
The preverb \textbf{къэ-} \glossphonemics{qa-} is used with both dynamic and static verbs. This preverb is used to form verb stems that denote the direction to the speaker (from there to here), e.g. \textbf{къэ-кlyэн} "to go here", \textbf{къэ-сын} "to come here", \textbf{къэ-жэн} "to run here", \textbf{къэ-тхэн} "to write here", \textbf{къэ-хьын} "to carry here", \textbf{къэ-джэн} "to shout here".

The preverb \textbf{къэ-} may form a verb stem from the name that denotes the direction of the action to the speaker, e.g. \textbf{Iуэху} \glossphonemics{ʔʷaxʷ} "to do", \textbf{къы-зо-Iуэху} \glossphonemics{qəzawʔʷaxʷ} "I report, convey here", \textbf{къ-о-Iуэху} \glossphonemics{qawʔʷaxʷ} "you report, convey here", \textbf{къ-е-Iуэху} \glossphonemics{qajʔʷaxʷ} "he reports, conveys here".

The main function of the preverb \textbf{къэ-} \glossphonemics{qa-} is not only to express the direction of action. Semantically, the preverb has many meanings.

The preverb \textbf{къэ-} \glossphonemics{qa-} may not express a directional meaning in dynamic verbs. Cf. \textbf{къэ-нэн} \glossphonemics{qanan} "to remain somewhere"; \textbf{къэ-хъун} \glossphonemics{qaχʷən} "to arise, to grow"; \textbf{къэ-укIын} \glossphonemics{qawət⁀ʃʼən} "to kill (animals and birds)"; \textbf{къэ-пцIэн} \glossphonemics{qapt⁀sʼan} "to err in something"; \textbf{къэлъытэн} \glossphonemics{qaɬətan} "to count, take into account"; \textbf{къы-бгъу-ры-щIэн} \glossphonemics{qəbʁʷərəɕʼan} "to bind on the side"; \textbf{къы-бгъу-ры-нэн} \glossphonemics{qəbʁʷərənan} 'to leave beside'; \textbf{къы-гуэ-нэн} \glossphonemics{qəgʷanan} 'to leave at the side, beside'; \textbf{къы-гуэ-кIын} \glossphonemics{qəgʷat⁀ʃʼən} 'to separate from someone, from something'; \textbf{къы-кIэ-ры-хун} \glossphonemics{qət⁀ʃʼarəxʷən} 'to lag behind'; \textbf{къы-кIэ-pы-шын} \glossphonemics{qət⁀ʃʼarəʃən} 'to take away, lead away'.

The preverb \textbf{къэ-} \glossphonemics{qa-} forms the bases of verbs that denote effectiveness and completeness of an action, e.g: \textbf{гуп-сысэн} "to think" - \textbf{къэ-гупсысын} "to think of something"; \textbf{псэ-лъэн} "to speak, to say" - \textbf{къэ-псэлъэн} "to say a word, speech"; \textbf{вэн} "to boil" - \textbf{къэ-вэн} "to boil, to boil"; \textbf{губжьын} "to get angry" - \textbf{къэ-губжьын} "to get angry, to get furious".

With the bases of static verbs, the preverb \textbf{къэ-} \glossphonemics{qa-} has a concrete-local meaning, e.g., \textbf{къы-щысын} \glossphonemics{qəɕəsən} "to sit beside, from aside"; \textbf{къы-щы-тын} \glossphonemics{qəɕətən} "to stand beside, from aside": \textbf{къы-щылъын} \glossphonemics{qəɕəɬən} "to lie beside from aside". Cf. also the dynamic verb \textbf{къэ-тIысын} \glossphonemics{qatʼəsən} "to sit near, nearby".

From the paradigmatic point of view, the 3rd person personal form of verbs with the preverb ke- differs from the corresponding unstated verbs. Cf. \textbf{ар къокIуэ} \glossphonemics{aːr qawkʷʼa} "he goes here", \textbf{ар макIуэ} \glossphonemics{aːr maːkʷʼa} "he goes".

The preverb \textbf{нэ-} \glossphonemics{na-} forms verb bases denoting the direction of action from the speaker (from here to there); \textbf{сы-нэ-плъащ} \glossphonemics{sənapɬaːɕ} 'I looked there'; \textbf{сы-нэ-сащ} 'I arrived there'; \textbf{нэ-с-тхащ} \glossphonemics{nastħaːɕ} 'I wrote there'; \textbf{нэ-с-хьащ} \glossphonemics{nasħaːɕ} 'I carried there'.
\subsubsection{Local Preverbs}
Preverbs de-, i-, and he- are united into one group by their semantics. They belong to the category of simple preverbs, which at the present stage of the development of Adyghe languages cannot be genetically associated with one or another part of speech. Theoretically, however, it is possible to assume that these simple local preverbs ascend to the nominative bases of local meaning. This assumption is supported by the general tendency of transformation of full-valued independent bases with a specific local meaning into local preverbs in the Abkhaz-Adyg languages.
Simple preverbs de-, i-, he- are used with both dynamic and static verb bases. From the bases of dynamic verbs, preverbs de-, i-, he- are used to form derivatives that indicate the direction of action, movement in or from within an object. Static verb stems with these preverbs denote position, being in an object, e.g:
de-: k'alam di-hye-n 'to enter, go into town', k'alam de-sy-n 'to be, be in town', pschIantIem de-kIy-n 'to come out of the yard', pschIantIem de-tyn 'to be, stand in the yard';
He-: psym he-kIe-n "to pour into water", psym he-sy-n "to sit, be in water", yatIem he-kIe-n "to get stuck, get stuck in mud", wesim he-kIe-n "to fall into snow", wesim he-y-n "to stand, be in snow"; mesyim hi-hye-n "to enter, enter the forest", mesyim he-sy-n "to sit, sit in the forest";
i-: zhypym i-lhy-'n "to put in the pocket", bytul'kIem i-kIe-n "to pour into the bottle", bytul'kIem i-l'y-n "to be in the bottle", matem i-kIoute-n "to pour into the basket", matem i-sy-'n "to sit in the basket".
As we can see from the examples, the preverbs de-, he-, i- have the meaning "inside". But it should be noted that these preverbs, though united by their semantics, are not synonymous preverbs. The point is that each of these preverbs is assigned to a particular group of nouns. The presence of one or another preverb with the meaning "within" in the verb stem is conditioned by the noun that enters into syntactic relations with the verb. At the same time, nouns which denote concepts united by some common properties are connected with a certain preverb. Therefore, the usability of a particular preverb is determined by the number of nouns assigned to it.
The circle of nouns requiring the presence of the preverb he- in the verbal stem is the most numerous. The presence of this preverb in the verbal base requires nouns that have:
1) substantive meaning: psy "water", ly "meat", hugu "millet", nartyhu "corn", prunzh "rice", guedz "wheat", myIeryse "apple", khuju "pear", bzhyn "onion", pho "honey", shygyu "salt", phoshygyu "sugar", makhsyme "makhsyma" (national drink), sane "wine", zhyzum "grapes", shchIakkhue "bread", khuyi "cheese", she "milk", gyushchI "iron", dyshe "gold", zhi "air", nehu "light", ues "snow", ueshkh "rain";
2) abstract meaning: gukyeue "grief", gufIegue "joy", gupsysse "thought, thought", dykhyeshkh "laughter", shIyIe "cold", huabe "heat", mamyrige "peace", "silence", pezh "truth", ptsIy "lie";
3) Collective meaning: gup "group", shIalagyuale "youth", zeshkhar "brothers", nybgegukher "friends", biykhar "enemies", dze "army".
The preverbs i- and de- have more specific meanings. The preverb i- is assigned to those nouns that denote objects that have a hole, a recessed place, a hollow space, etc., for example: mutul'kIe "bottle", shak'al'e For example: mutulykIe "bottle", shak'al'e "inkpot", shyg'ul'e "saltcellar", foshyg'ul'e "sugar bowl", jyp "pocket", une "house", mashe "pit", gue "hole, den", gu "arba", mashine "car", khukh "steamboat", khukhlyate "plane", tebe "frying pan", pegun "bucket", tepshach "plate", stachan "glass", mate "basket".

Compared to preverbs ke- and i-, the preverb de- is used less frequently, because the number of nouns assigned to it is not very numerous. Such nouns require the presence of the preverb de- in the verb stem: k'uaje "village", k'ale "town", pschIantIe "yard", nyje "river floodplain, flood meadow", psykhue "bank", ueram "street", pkhuante "trunk", cabinet "cabinet". Word forms such as (i)shIyb, (i)shıbagъ "behind", "behind", "outside", yazihuacu "between" also require the preverb de-, for example: abı and shıbagym sydessh "I sit behind it"; kъaler psytIym ya zechuacum dashyhyahyash "between two rivers they built a city".
Preverb shIe - forms the bases of static and dynamic verbs and brings the meaning "under" into the base of the verb (cf. The verb preverb is used to form the base of a verb with the meaning "bottom": "stone coal is under the ground"; "fruit is put in the cellar"; "a guy is sitting under a tree".
Preverb t(e) - is used with both static and dynamic verbs and indicates that the action is performed on the surface of an object, for example: te-syn "to sit on something", te-l'yin "to lie on something", te-kIyn "to come off the surface of something", te-hyn "to bring something off the surface", te-shen "to do something on the surface of something", te-l'yin "to put on the surface of something".
In contrast to the local preverbs de-, he-, i-, which cannot express the position or action within an object regardless of its external forms (i.e. depend on the semantics of the noun), the preverb t(e)- denotes the position or action on the surface of an object of any form, e.g: psym t-e-l-yin "lie on the water", psym he-l-yin "lie in the water"; matem t-e-l-yin "lie on the basket", matem i-l-yin "lie in the basket"; pkhuantem t-e-l-yin "lie on the trunk", pkhuantem de-l-yin "lie in the trunk".
The preverb shy- is used with the bases of static and dynamic verbs. It is used to form the basic lexical form of the bases of static verbs: shysyn "to sit", shytyn "to stand", shylyn "to lie".
In these verbs, the preverb shy- has no local meaning. This is evidenced by the fact that these static verbs, formed by means of the preverb shy-, are used in a sentence that does not specify a specific local place of the subject: ar zheshkIe kuedre shysh "he sits a long time at night"; se nobe shysh "I sit today (i.e. do not work, rest)".
The loss by the preverb shy- of its local meaning is secondary. Preverb shy- in the past indicated that an action was taking place on the surface of an object. The main argument in favor of this assumption is the fact that this preverb still preserves traces of its etymological meaning. As noted above, the preverb indicates an action on the surface of an object, regardless of its form, i.e., the circle of nouns assigned to this preverb is not quantitatively limited.
However, the exception among the nouns is the word shIy 'earth'. The state on the surface of the earth may be expressed by the base of the static verbs formed with the preverb shy-, cf. shIym shysyn "to sit on the ground", shIym shylyyn "to lie on the ground".
The former meaning of preverb shy- is also preserved in a number of dynamic verbs, for example: shykhun "to fall down from the surface", shydzyn "to dump", shykhuen "to rub against something", shytsIel'yyn "to smear on something", shytIegen "to put on, put on".
The same preverb is represented in the verbs: shchikIyin "to get a reputation", shchyuen "to make a mistake".
Thus, historically, preverb shy- was a local preverb.
Consequently, an action on the surface of an object is expressed in the verb not only by the preverb t-, but also by the preverb shy-. Obviously, both to express action and condition within an object and to express action and condition on its surface, different preverbs have been used in the verb, depending on the form of the object itself. Otherwise, it is difficult to explain why, with the same verb "sit Otherwise, it is difficult to explain why different preverbs are used for the same verb 'sit' (shIym shysh 'sits on the ground', shantym tessh 'sits on a chair').

In this connection, the complex preverb ḥe-ḥy- draws attention. The first part is taken from the word ṇe 'head' and the second part is a preverb ṇy-. In some cases, the preverb shkhe-she is synonymous with preverb shy-, which to some extent sheds light on the original meaning of the latter, for example: bgym shkhe-shekhun, bgym shkhekhun 'to fall down from a mountain, from a cliff'.
The preverb fIe- forms the stative and the dynamic verb bases. Verb stems formed by means of this prefix, denote:
action or state on the surface or terminal part of an object, e.g.: fIel'ın "to lie on top," and pem fIetın "to stand on the front."

position on the surface of an object: blinim fIel'yin "to hang (lit. lie) on the wall", kIapsem fIel'yin "to hang (lit. lie) on a rope", pyshIen "to tie on the surface, on the end part", fIedin "to sew on the surface, on the front part", fIesykhyyn "to dangle, wobble on something".

In addition, the bases of dynamic verbs with the preverb fIe- denote action on the surface or on the end part of an object; action directed to the edge, to the end of an object or from the end of an object: fIelhyn "to put on, put on something", fIekIõyn "to separate from something", fIehõyn "to remove from something".

By its sound, the local preverb fIe- coincides with the version affix fIe-, cf. fIedzyn "to cast off from something" and fIedzyn "to cast against the will of someone"; fIeden "to sew on" and fIeden "to sew against the will of someone-whatever". Obviously, here we are dealing with affix homonyms, which coincided as a result of sound changes.

Preverb dust-. Verbal bases formed with this preverb denote the state and action occurring in the front of something, for example: pytyn "to be, stand in the front"; pysyn "to be, sit in the front"; pygyen "to join from the front", pygyn "to take off from the front"; pyden "to sew on the front".
The preverb py- is found in Abkhaz and Abaza languages with the same meaning, for example: Abkh. a-pylara "to meet, to go towards".

All-Abkhazian-Adygian preverb py- is derived from the noun base pe: Adyg.-kab. pe, Abkh. a-pyntsIa "nose", "front part of something".

By its meaning, the preverb py- is close to the preverb fie-. These preverbs may sometimes replace each other, e.g:
..

..

..

The preverb khifIe-, being a compound preverb, consists of simple preverbs khy- and fIe-. While its constituents are combined with the bases of transitive and non-transitive verbs, the compound preverb khifIe- is combined only with some bases of transitive verbs of motion. The meaning of the compound preverb khifIe- also does not correspond to the sum of the meanings of the simple preverbs khy- and fIe-, cf. khifIe-hue-n 'to push', 'to push', 'to fling' (from the base of the verb hu-n 'to chase'), khifIe-dze-n 'to throw', 'to to toss' (from the base of the verb dzy-n 'to throw'). As may be seen, the formation of the verb by means of the preverb khifIe- is accompanied by the change of the vowel base y to e.
The preverb pery-, which consists of the nominal preverb pe- and the element -ry-. The latter is evidently a phonetic accretion at the junction of the preverb and the verbal base. As will be seen below, this sonorous consonant often appears after the vernacular preverbs. The meaning of the preverb pery- is close to the preverb py-, e.g.: perytyn "to stand before someone or something"; perylhyn "to lie before someone or something"; peryhyn "to approach from the front"; perylheden "to run to the front"; perylhyn "to put from the front".

The preverb paschie-, being a compound preverb, consists of simple preverbs pae- and shchie-. The simultaneous meaning of the preverb does not correspond to the sum of the meanings of its components. The second component - shIe-, whose etymological meaning strongly faded in combination with pe-, has undergone desemantization. The preverb peshie- is combined with both transitive and non-transitive verb bases, e.g: peschIedzyn "to throw away, throw out from under something", peschIekIyin "to withdraw, move away from someone-something", peschIehyin "to remove, remove from under something"; peschIeljeden "to run up, run up to someone-something"; peschIekIueten "to approach, come close to someone-something"; peschIetyn "to stand against someone-something".

The compound preverb under consideration has the phonetic variant pyshIe-, cf. peschIekhuen// pyshIehuen "to be caught", "to meet", peschIeuven// pyshIeuven "to oppose".
Preverb kъue-. Verbal bases with this local preverb denote state and action behind an object, behind something and also direction of movement behind or from behind an object. Examples: k'uel'yyn "to lie behind an object"; k'uesyn "to sit behind an object"; k'uetyn "to stand behind an object"; k'ueshihiyhiyyn "to do something behind, behind an object"; k'uehiyyn "to go behind an object"; k'uekiyyn "to go out from behind an object".

The local preverb kъue- is genetically associated with the postlogue kъuagъ "corner", "the place behind something", cf. une kъuagъym "behind the house", and kъuagъ "behind it, behind it".

The preverb \textbf{лъы-} is derived from the common Adyg noun base \textbf{лъэ-} "trace," "foot," "lower part of something. The preverb \textbf{лъэ-} is used to form the bases of dynamic verbs that denote action, motion, following someone-something: \textbf{лъыщIэхьэн} "to catch up with someone-something", \textbf{лъыкIуэтэн} "to move after someone-something".
The etymological meaning of the local preverb \textbf{лъы-} is preserved in such derivative bases as \textbf{лъытIэгъэн} lit. "to put on legs", \textbf{лъыхын} "to take off the legs". Cf: \textbf{Iыхын} 'to take away, to take from the hands', \textbf{щхьэрытIэгъэн} 'to put on the head'.

From the bases of static verbs, the base \textbf{-гъ} is combined with the preverb \textbf{лъы-}: \textbf{вакъэ слъыгъщ} "I am shoed", lit. "my feet are in shoes."

The preverb kIel-i, being a compound, is composed of two simple preverbs kIe-+l'e-. By its meaning, the preverb kIel- is close to the preverb l-: kIel-ižen "to run after someone or something"; kIel-iõõõyn "to carry after someone or something", kIel-idzyn "to throw after someone or something".

Expression of action and motion following somebody-something by verb bases with one local preverb l- is an archaic phenomenon. The redistribution of the function of one preverb l'y- between two preverbs kIe- and l'y- (kIel'y-) is a later phenomenon. It is proved by the presence of parallel (synonymous) derivatives of the following types: l'yshchIhehin // kIel'yshchIhehin "to catch up with someone-something", l'ykIueten // kIel'ykIueten "to follow someone-something", l'ehezhin // kIel'ezhin "to bear after someone-something".

The common Adyghe language did not know complex preverb kIel'y-, the state of the common Adyghe language in this respect is still preserved by Adyghe language, cf. Adyg. l'y- jen, Kab. kIel'yijen "to shout after sb. something".

Preverb bla- goes back to the noun base, cf. bla "forearm. The preverb blă- is used to form the bases of dynamic verbs that denote motion in the vicinity of someone-something: blălătyn 'to fly past someone-something'; blăhyn 'to carry past someone-something'; blăzhyn 'to run past someone-something'; blăshyn 'to lead past someone-something'.
This preverb is present in the Ubykh language, but has a different meaning, cf. ubykh. sy-bla-s "I sit inside something"; sy-bla-l-l "I lie inside something"; sy-bla-t "I stand inside something".

\subsubsection{Complex Preverbs}
The reciprocal prefix ze- is part of a number of compound affixal morphemes. As has already been noted, complex affixal morphemes in the base of a verb take the place of their constituent elements.

Verbs with the prefix \textbf{зэхэ-} (<\textbf{зэ-}- + local preverb \textbf{хэ-}) express:
\begin{xlist}
	\ex joint position, joint action, e.g.: 
	\begin{xlist} 
		\ex \textbf{зэхэтын} "to stand together", zehegyn "to lie together", 
		\ex \textbf{зэхэлъын} "to lie together", 
		\ex \textbf{зэхэсын} "to sit together", 
		\ex \textbf{зэхэгъэлыбжьэн} "to fry together", 
		\ex \textbf{зэхэпщэн} "to mix together", 
		\ex \textbf{зэхэхьэжэн} "to grind together";
		
	\end{xlist}

	\ex the completeness of the action, e.g.: 
	\begin{xlist}
		\ex \textbf{зэхэцIэлэн} "to stain (wholly)", 
		\ex \textbf{зэхэчэтхъэн} "to tear (all)".
	\end{xlist}
\end{xlist}
A different meaning is introduced by the prefix \textbf{зэхэ-} in verbs like \textbf{зекIуэн} 'to walk', cf. \textbf{зэхэзекIуэн} 'to walk in different directions', \textbf{зэхэзежэн} 'to run in different directions'.

The prefix zechIe- (<ze- + local preverb shIe-), which brings into verbs the meaning of completeness, intensity and joint action, e.g: zeschIephenkIen "to replace"; zeschIekhyuen "to gather, rake"; zeschIekIen "to overgrow"; zeschIedien "to cool"; zeschIelen "to paint"; zeschIeplan "to warm up"; zeschIekhien "to shake up", "to come into motion"; zeschIeshasen "to mount horses (of many)"; zeschIbeghen "to swell (whole)".

The prefix zekhue- (<ze-+version-narrative affix hue-) expresses movement from different directions to one point, e.g.: zekhuesyn "to gather", zekhuekIuen "to go towards one another", zekhueunetIyyn "to head towards one another". The same structure has: zechuetheusihan "to complain to each other", zechuuepsen "to bestow on each other", zechuzezin "to meet each other".

The absence of the 3rd person singular indirect object in the return-version verbs creates a grammatical homonymy, cf. zy-hu-z-o-schI 1) "for him I prepare myself" (lit. "for him do myself"), 2) "for myself do"; zy-hu-b-o-shchI 1) "for him do you prepare" (lit. "for him do yourself"), 2) "for yourself do"; zy-hu-e-shchI 1) "for him do you prepare" (lit. "for him do yourself", 2) "for yourself do."

Such homonymous reversionary forms are formed from a few transitive verb bases.
\subsection{Ablauts in verbal word formation}

\section{Deverbal Formations}
\subsection{Participle}
\subsubsection{Types of participles}
The participle in Kabardino-Circassian is a verb-noun form with the categories of person, number, time, and case. Their classification, which has become traditional in Adygology since the 1960s, is based on the relation between the objects defined by the participle and the actions that the same participle conveys. On this basis the participle is divided into subject, object, instrumental and circumstantial. The latter are in turn subdivided into factual participles of place, time, mode, cause and object.

The subject participles specify the object that serves as the subject of the participle action: \textbf{кIуэр} "one who goes" (from the dynamic verb kIuen 'to go'), \textbf{щысыр} "one who sits" (from the static verb shysyn 'to sit').

Object participles determine the object of the participle action: \textbf{ишэр} "one whom (he) leads" (from the dynamic verb \textbf{шэн} "to lead (sb.)", \textbf{зытетыр} "that on which he stands" (from the static verb \textbf{тетын} "to stand on something").

Instrumental (instrumental) participles point to the instrument or means of the action contained in the participle: \textbf{зэрыкIуэр} "with which (he) rides" (from the verb kIuen "to go"), \textbf{зэрылажьэр} "with which (he) works" (from the verb lazhen "to work"), etc.

The indicative participle of place indicates the place where the action in the participle took place: \textbf{зыдэкIуэр} "where (he) goes" (from kIuen "to go"), \textbf{зыщыс-лъэгъуар} "where I saw him" (from the verb laghun "to see whom or what"), etc.

The indicative participle of time indicates the time of the action contained in the participle: \textbf{щыкIуэр} "when he walks", \textbf{щылажьэр} "when he works", etc.

The Locative Participle of the Action shows how the action in the Participle was carried out: \textbf{зэрыкIуэр} "as he walks", \textbf{зэрылажьэр} "as he works", etc.

The Locative Participle of Reason, of Purpose, indicates the cause or purpose of the action: \textbf{щIэкIуэр} "why he goes", \textbf{щIыщысыр} "why he sits" (from the verb shysyn "to sit"), etc.

The main points of this classification were formulated by A. K. Shagirov, who suggested that the principle of the collateral principle be abandoned when singling out the classes of participles1. Among other approaches, of interest is the classification of participles based on their derivative structure. Its author, Z. I. Kerasheva, divides Adyghe languages into neutral, subject, object, subject-object, circumstantial and non-subjective participles.


\subsubsection{Formation of Participles}
\paragraph{The formation of subject and object participles}

Subject and object participles are derived from both dynamic and static verbs with the prefix \textbf{зы-}(\textbf{зэ-}) or without the special affix.

The prefix \textbf{зы-} is used to form:
\begin{xlist}
\ex  subject participles from transitive verbs: 
	\begin{xlist}
	\ex \textbf{зышэр} "one who leads (him)" (from \textbf{шэн} "to lead (sb.)");
	\end{xlist}

\ex  indirect-object participles from transitive verbs: 
	\begin{xlist}
	\ex \textbf{зы-хуэсшэр} "he for whom I lead him" (from \textbf{хуэшэн} "to lead sb. for sb.");
	\end{xlist}

\ex indirect-object participles from non-transitive verbs: 
	\begin{xlist}
	\ex \textbf{зы-дэкIуэр} "the one with whom he goes" (from \textbf{дэкIуэн} "to go with sb.").
	\end{xlist}
\end{xlist}

Without the prefix \textbf{зы-} are formed:
\begin{xlist}
\ex subject participles from non-transitive verbs: 
	\begin{xlist}
	\ex \textbf{лажьэр} "one who works" (from \textbf{лэжьэн} "to work");
	\end{xlist}

\ex near-object participles from transitive verbs: 
	\begin{xlist}
	\ex \textbf{ишэр} "he whom he leads" (from \textbf{шэн} "to lead (sb.)").
	\end{xlist}
\end{xlist}
In other words, the prefix \textbf{зы-} is present in participles when the relative name of the original verb is in the ergative case. Cf. \textbf{сэ ар зы-хуэсшэр} \glossphonemics{sa aːr zəxʷasʃar} "the one for whom I conduct it" - \textbf{Сэ ар лIыжьым хузошэ} \glossphonemics{sa aːr ɬʼəʑəm xʷəzawʃa} "I conduct it for an old man". The relative word \textbf{лIыжь} \glossphonemics{ɬʼəʑ} "old man" is in the ergative case.

The prefix \textbf{зы-} is absent from the participle when The relative name of the original verb is in the nominative case. Cf.: \textbf{абы ишэр} "the one whom he leads" -\textbf{Абы лIыжьыр ешэ} "He leads an old man". The relative name lIyzh "the old man" is in the nominative case.

The voicing of the prefix \textbf{зы-} may change in some cases. For example, in indirect-object participles constructed from non-transitive verbs with indirect object prefixes \textbf{е-}, \textbf{и-}, such as \textbf{еджэн} 'to read', \textbf{исын} 'to sit in sth. The participle prefix takes the form \textbf{зэ-}: \textbf{зэ-джэр} "that which he reads", \textbf{зэ-псалъэр} "he with whom he speaks" (from the verb e-psal'en "to speak to sb.), \textbf{зэ-даIуэр} "he whom he listens to" (from the verb \textbf{е-дэIуэн} "to hear sb."), etc.

The number of participles formed from any of the verbs depends on the person of the verb - the more persons the verb is, the greater, as a general rule, the number of participles formed from it. There are, however, other factors which affect the formation possibilities of a verb. It is therefore necessary to dwell on this subject in more detail.

\begin{xlist}
\ex From monopersonal intransitive verbs only the subject participles are derived, cf. kIuer, 'he who goes' (from kIuen 'to go') and shysyr, 'he who sits' (from shysyn 'to sit').
\ex From dipersonal intransitive verbs the subject and indirect-object participles are derived.
	\begin{xlist}
	\ex Subject participles are derived f without affixation: ezher "one who waits for him" (from ezhen "to wait for (sb.)"), hue-kIyep "one who goes for him" (from huekIuen "to go (where) for sb."), etc.
	\ex Indirect-object participles from two-person non-transitive verbs are built with the prefix zy- (ze-): zezher "he whom he waits for" (from ezhien "to wait for (whom-thing.)"); ze-uer "he whom he beats" (from euen "to beat sb.").
	\end{xlist}
\ex From tripersonal intransitive verbs the subject and one indirect-object participles are derived.
	\begin{xlist}
	\ex The subject-participle is constructed unaffixatively: dejer "one who waits with him" (from dejen "to wait for sb., together with sb."), hudekIuer "one who goes with him for his sake" (from hude-kiuen "to go with sb. for (for) sb."), etc.
	\ex The indirect-object participle is made with the prefix zy-: zy-dejer "one with whom he waits for him" (from dejen "to wait for sb. with sb."), zy-hudekIuar "one for (for) whom he went with him" (from hudekIuen "to go somewhere with sb. for sb.'s sake"), etc.
	\end{xlist}
	From ternary non-transitive verbs no second indirect-objective participle is formed: from the verb dejen 'to wait for sb. with sb.', for example, it is impossible to construct a participle with the meaning 'one whom he waits with him'.
\end{xlist}

The three-person non-transitive verbs form the subject and one indirect-object participle.


\paragraph{The formation of substantive and instrumental participles}
\subsubsection{The word formation of participles}
\paragraph{Change by person}
\section{Adverb}
\section{Utility words}
\section{Interjections}
\chapter{Syntax}
\end{document}

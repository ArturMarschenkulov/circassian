\documentclass[a4paper,12pt]{book}
\usepackage{fontspec}
\setmainfont{Charis SIL}

\usepackage{tocbasic}
%\usetocstyle{standard}
\usepackage{longtable}
\usepackage{gb4e}
\usepackage{fullpage}
\usepackage{hyperref}
\hypersetup{
    colorlinks = true,
    linkcolor = blue
}

\usepackage{nameref}

\usepackage{multirow}% http://ctan.org/pkg/multirow
\usepackage{multicol}
\usepackage{hhline}% http://ctan.org/pkg/hhline
\newsavebox\ltmcbox

\pagestyle{headings}
\pagestyle{plain}

\newcommand{\1}[1]{\textbf{\emph{#1}}} %at'ik words
\newcommand{\2}[1]{\textbf{[#1]}} %at'ik phonetics
\newcommand{\3}[1]{\fontsize{11pt}{0cm}\textbf{\emph{#1}}} %at'ik morphemes
\newcommand{\4}[1]{\fontsize{10pt}{0cm}\emph{#1}}	%names of morphemes
\newcommand{\5}[1]{\textbf{/#1/}} %at'ik phoneMics //
\newcommand{\6}[1]{\textbf{[#1]}} %at'ik phoneTics []
\newcommand{\7}[1]{\fontsize{12pt}{0cm}\emph{#1}} %emphasis
\newcommand{\8}[1]{\fontsize{12pt}{0cm}`#1'} %English words
\newcommand{\9}[1]{\fontsize{12pt}{0cm}(lit. `#1')} %English words

\newcommand{\glossphonemics}[1]{\textbf{/#1/}} %at'ik phoneMics //

\title{Kabardian Grammar in two volumes (translation)}

\begin{document}

\chapter{Creation of writing}
\section{From the History of Kabardino-Circassian Writing}
In the history of language development, the appearance of written monuments is not always connected with the creation of writing in that language. Some written monuments of the Kabardino-Circassian language belong to travelers, state officials and scholars who had no relation to the development of writing. However, there are also monuments of writing that were made in connection with the attempts to create writing in the Kabardino-Circassian language. But irrespective of the reasons and conditions of their emergence, the existing written monuments are an important material for studying the history of the Adyg languages. These monuments greatly help clarify the linguistic processes in the modern Kabardian-Circassian literary language. In this respect, the earliest materials are also of certain interest, despite their scantiness and extreme imperfection of the techniques of their graphic fixation. Thus, the lexical and textual materials of Evlius Chelebi, N. Witsen, F. I. Strallenberg, I. A. Guldenstedt, P. S. Pallas, G. Yu. Klaproth1 are not only fragmentary, but also suffer from significant flaws, which has caused skeptical attitude to them by specialists. However, it would be wrong to completely ignore them when studying the history of the Kabardino-Circassian literary language. For example, the small lexical material of N. Vitsen reflects a number of archaic features, lost by the modern Kabardino-Circassian language. The records of I. A. Guldenstedt, inaccurately transcribing the Adyghe material with signs of different graphic systems, despite their flaws, shed light on the history of the development of some classes of words. In particular, this material is valuable in reconstructing the history of the development of numerals in the Adyghe languages. The fact is that numerals in the Adyghe languages are characterized by extreme diversity in their word-formation structure. In the living dialects there are different systems of counting - decimal, twentieth and mixed, and there are great differences between the Adyghe dialects and languages in the formation of quantitative numerals. The records of I. A. Guldenstedt helps to identify later innovations and reconstructions of the common Adyghe counting system, which is important for studying the history of the development and formation of the Adyghe literary languages. Therefore, one can hardly agree with P.K. Uslar, who considered completely fruitless the activity of G.Y. Klaproth to collect the linguistic material2. With all their shortcomings, the materials of the above authors are still important for the history of the Adyg languages, that do not have a long written tradition.\\
As for the works of L. Lopatinsky and Sh. Nogmov, they made a significant contribution to the study of the Kabardian-Circassian language in the pre-revolutionary period (see below). Especially valuable for the history of the Kabardino-Circassian literary language are the lexicographical studies of the mentioned authors.\\
Together with the study of the Kabardino-Circassian language, attempts to create a written language were made in the pre-revolutionary period. In the beginning of the 19th century, Russian educators of Adygea people worked on compiling the Kabardian-Circassian alphabet and alphabet.\\
In 1829 I. Gratsilevsky, a teacher at St. Petersburg University, created the Circassian alphabet based on Russian graphics. As S. Petin notes, with the help of Grazilevsky alphabet "students even used to correspond with each other later "1. 1 The alphabet of I. Grazilevsky compiled for soldiers of the Caucasus semi-squadron was not distributed.\\

\chapter{Phonetics and Phonology}
\chapter{Morphology}
\section{General characteristics of the morphological system}
In terms of the complexity of the systems of word-formation and word-formation, the name and the verb form two poles. The name, as compared to the verb, has a relatively simple structure. Nouns have grammatical forms of number, definiteness-undefiniteness, case, possessive and conjunctive. All these grammatical forms, except possessive, are expressed by means of suffixal morphemes. Prefixation is used to express possessiveness. Cf.: \textbf{унэ} "house", \textbf{унэхэр} "house", \textbf{унэр} "house" (definite, known), \textbf{уыни} "and house" (union form); \textbf{шы} "horse", \textbf{сиш} "my horse", \textbf{Iэ} "hand", \textbf{сиIэ} "my hand".\\
The distinction between the main two cases, the Nominative and the Ergative, is related to the nature of the verb base. A transitive verb creates an ergative construction and a non-transitive one creates a nominative construction, for example: \textbf{абы тхылъыр къищэхуащ} "he bought a book" (\textbf{абы} "he" is ergative, \textbf{тхылъыр} "book" is neuter, \textbf{къищэхуащ} is infinitive); \textbf{ар мэлажьэ} "he works" (\textbf{ар} "he" is neuter, \textbf{мэлажьэ} is a non-receptive verb).\\
From the point of view of morphology, adjectives are not very clearly distinguished from nouns. The latter have degrees of comparison - positive, comparative and superlative. Degrees of comparison are expressed in two ways: synthetic (or rather suffixal) and analytic, for example: \textbf{IэфIыщэ} "too sweet", \textbf{нэхъ IэфI} "more sweet", \textbf{нэхъ IэфI дыдэ} "the sweetest".\\
Derivative names are formed by means of affixation, word formation and conversion, for example: \textbf{цIыху-гъэ} "humanity" (suffixation); \textbf{джэдкъаз} "poultry" (\textbf{джэд} "chicken", \textbf{къаз} "goose"), \textbf{псчэ} "cough" from \textbf{псчэ-н} "to cough" (conversion). The most productive of these methods of nominal word formation is word formation. Of the affixal types of word formation in names, only the suffixal one is active. It should be noted that suffixal word formation is mostly characteristic of nouns. As for the word-formation suffixes of adjectives, they are very few, and the most productive of them have not been completely grammatized.\\
In combinations 'noun + noun' and 'noun + adjective', the word-change suffixes are attached only to the postpositional member, e.g.: \textbf{пхъэ унэ-хэ-р} 'wooden houses', \textbf{унэ дахэ-хэ-р} 'beautiful houses', (\textbf{-хэ} is a plural suffix, \textbf{-р} is a neuter suffix).\\
The verb has an exceptionally rich and complex system of word-formation and word-formation. The verb is the clearest of all typological features of the Kabardino-Circassian language.\\
Paradigmatically, we distinguish between transitive and non-transitive verbs, dynamic and static verbs, finitic and infinitic verbs. The grammatical meaning of the verb base determines the paradigms of conjugation and the sentence structure as a whole. Primary Static verb bases, in contrast to Dynamic verb bases, are in the closed list, i.e., they are quantitatively restricted. The paradigm of the Static verbs may include noun bases without any stemming morpheme. Cf. e.g.: \textbf{у-щIалэ-тэ-мэ} 'if you were a young man' (y is the subject prefix of the 2nd person singular, shIale is the stem of 'young man', -tae is the temporal suffix, -me is the inflectional suffix). This indicates a weak morphological differentiation of nominal and verbal bases. In the synchronic plane, there are bases that are neutral with respect to their belonging to nominal and verbal bases. From such bases are formed forms of nouns and verbs (static and dynamic). Cf. jaegu "game", so-jaegu "I play", ar jaegu-ch "this is a game".\\
Verbs are characterized by affixes of person (subject, direct object, indirect object), number, union, negation, time, inclination, causative, version, possibility, complicity, reciprocity, jointness, reversibility, locative and directive preverbs, modal particles, etc. A distinctive feature of the verb is also the circumstantial forms, functionally equivalent to adjective sentences. These are morphological forms such as \textbf{сы-зэрыкIуэр} "as I go", \textbf{сы-зэрыкIуэ-у} "as soon as I go", \textbf{сы-зэры-кIуэ-рэ} "since I went", \textbf{сы-здэ-кIуэ-р} "where I go", \textbf{сы-щIэ-кIуэ-р} "why I go", \textbf{сы-щы-кIуэ-р} "when I go", \textbf{сы-кIуэ-ху} "while I go".\\
The forms of person, version, dynamism, complicity, reciprocity, jointness, and place of action are expressed by prefixes, e.g.: \textbf{с-о-кIуэ} "I go", \textbf{у-о-кIуэ} "you go", \textbf{сы-ху-о-кIуэ} "I go for him", \textbf{сы-до-кIуэ} "I go together with him", \textbf{ды-зд-о-кIуэ} "we go together", \textbf{ды-зэр-о-лъагъу} "we see each other", \textbf{сы-те-с-щ} "I sit on something".\\
The forms of tense, inclination and conjunction are expressed by suffixes, e.g., sy-kIu-a-sh "I went", sy-kIua-me "if I went", sy-kIue-ri "I went and".\\
Forms of negation and possibility can be expressed in prefixal and suffixal ways, cf. sy-my-kIue "I am not going" (we- is the prefix of negation), sy-kIuer-kyym (-kyym is the suffix of negation), s-hue-hyinsh "I can carry" (hue- is the prefix of possibility), sy-kIue-fyn-sh "I can go" (-fa is the suffix of possibility).\\
The direction of action is expressed in prefixal and prefix-suffixal ways, e.g.: sy-ky-o-kIue "I go here" (kъ(e)- direction prefix), sy-d-o-kIue-y "I go up" (d(e)- direction prefix, -y(s) - direction suffix).\\
Different methods are used to express circumstantial forms - prefixal, suffixal and prefix-suffixal, for example: zer-i-thyr "as he writes" [zer- is a prefix meaning "as"], i-thy-hu "while he writes" (-hu is a suffix meaning "until"), zer-i-thy-u "as soon as he writes" [prefix zer- and suffix -u meaning "as soon as "*].\\
The verb form may be single-morphemic, cf. zhe 'run! "write!" But the verb as a whole is characterized by a high degree of synthesis.\\
The process of fattening a root morpheme with affixes (prefixes and suffixes) is illustrated by the following example:
у-а-къы-д-е-з-гъэ-шэ-жы-ф-а-тэ-къым "I couldn't get him to lead you with them here then."\\

\section{Stem structure}
\section{Noun}
\section{Pronoun}
\section{Adjective}
\section{Numeral}
\section{Ablaut in Nominal Word Formation}
\section{Verb}
\subsection{General characteristics of verb polysynthesis}
The Adyg languages have inherited from the primordial state a verb system with an extremely complex structure and a high degree of synthesis.

The verb word form may be single-morphemic: common Adyg. dy- "to sew", khy- "to mow", zy- "to tsed". The single-morphemic base of the mentioned type varies depending on the surroundings in modern Adyghe languages as CV, C: dy! "shi!" / from dy- "sew" /, so-d "I sew". In such cases, the all-Adygian structure of the base is preserved in Adygean: se-dy "I sew. However, the depth of the verb word-form may be very large.\\
An example, taken from the Kabardian language, gives an idea of the degree and depth of verb synthesis:\\
\textbf{шэ}\\
\textbf{е- шэ -ж}\\
\textbf{з- гъэ- шэ -жы -ф}\\
\textbf{е- з- гъэ- шэ -жы -ф -а -т}\\
\textbf{д- е- з- гъэ- шэ -жы -ф -а -т}\\
\textbf{къIы- д- е- з- гъэ- шэ -жы -ф -а -тэ -къIым}\\
"lead"\\
"he leads him back."\\
"make me lead him back."\\
"I could then make him lead back."\\
"I could then make him lead him back"\\
"I couldn't get him to drive back here with him then."\\

In this "triangle," the root morpheme she "lead" of the last /sixth/ word form is complicated by ten affixal elements - five prefixes and five suffixes. But there are also more extended chains of morphemes in the word paradigm. The latter chain can be extended by including several more affixal morphemes in the paradigm: \textbf{у-а-къIы-ды-д-е-з-гъэ-шы-жы-ф-а-тэ-къIым-и}i "I could not then get him to take you back out of there with them." This paradigm within one word consists of fifteen significant elements /eight prefixes + root morpheme + six suffixes/: y- prefix of direct object of 2nd person singular, a- prefix of indirect object of 3rd person plural, kъIy- directional prefix "here", y- prefix of allied action, e- local prefix, f- prefix of object of 3rd person singular, h- subject prefix of the 1st person singular singular. g-e causative prefix, shy- morpheme of the verb she-n "to lead", -y return suffix, -f possibility suffix, -a past tense suffix, -te suffix "then", -kъIym negative suffix, -i suffix with modal meaning. The high degree of synthesis in a verb is created by the fact that not only the categories of person, number, time and inclination are expressed morphologically, but also negation, affirmation, question, causative, joint action, localization, directional relations, etc.

The structure of the verb, its main word-formation and word-formation categories, their relations, the hierarchy and arrangement of numerous base-forming elements, the distribution of personal affixes in different types of bases, their dependence on other categorical forms, including verb transitivity-non-transitivity, etc., date back to the era of the All-Adyghe language unity. Moreover, the most important material elements (affixes of person, causative, negation, and a number of other formants serving to express forms of word-formation and word-formation) are chronologically derived from the Western Caucasian level.

It is important to stress that the genetically common material features that unite Adyghe verb with the verb of other West-Caucasian languages are inseparable from the structural-typological commonalities that also go back to the West-Caucasian condition. As we shall see, the structural hierarchical features, which have their roots in the epoch of prelanguage unity of the Western Caucasian languages, turned out to be extremely stable in the whole group of languages, indicating a close relationship between genetic and typological features of the verb. In this respect, the verb differs from other classes of words, preserving a number of primordial structural-typological and genetic features that unite all modern Western Caucasian languages. In any case, in terms of the specific weight of structural-typological features, which are attributed to the Western Caucasian condition, the verb has no counterpart among other parts of speech. This refers primarily to the distribution of word-formation and word-formation morphemes, including subject-object indicators.

The main structural types of the verb stem coincide in the Abkhazian-Adygian languages, although the differences between their two main subgroups - the Adygean-Kabardian and Abkhazian-Abazian - are so significant that so far not all consistent sound correspondences between these languages have been established. At the same time, the order of verbal derivational morphemes in the Abkhazian-Adygian languages is so similar that knowing the arrangement of the significant elements of the stem in the Adygian languages, we can build models of the verb stem multimorphemes in the Abkhazian or Abazinian. So, if the type of stem including the root morpheme /K/ and morphemes expressing stimulus /P/, specific localization /L/, joint action /C/ is given, then in the whole group of languages the significant elements of the stem are distributed as follows: S - L - P - C.

The main patterns of distribution of subject-object, derivational prefixal and root morphemes cannot be explained as a result of parallel development of the Western Caucasian languages, but must be attributed to their pra-language condition.\\
\subsection{Transitive and non-transitive verbs}
In the complex system of word-formation and word-formation of a verb, the division into transitive and non-transitive verbs is central. The fact is that the sign of transitivity is present in all words and verb forms.

The categorical (invariant) meaning of transitivity-non-transitivity can be considered as syntagmatic, as it depends on the presence of subject and object in a certain case form of the name and on how they are represented in the verb form. A transitive verb requires a subject in the ergative case and an object in the nominative case: \textbf{Дыгъэм щIыр егъэхуабэ} \glossphonemics{dəʁam ɕʼər jaʁaxʷaːba} - "The sun warms the earth". Here, the subject \textbf{дыгъэм} is in the ergative case and the object \textbf{щIыр} in the nominative case. A non-transitive verb has the subject in the nominative case and the object in the ergative case: The subject is in the nominative case and the object is in the ergative case. The subject is in the nominative case and the object is in the ergative case. Non-transitive (more often) and transitive (less often) may be objectless: \textbf{ЩIалэр йоджэр} \glossphonemics{ɕʼaːɮar jawd⁀ʒar} - "A young man studies" and \textbf{ЩIалэм къекIухь} \glossphonemics{ɕʼaːɮam qajkʷʼəħ} - "A young man walks".

A syntactic construction with a transitive verb is called an ergative verb and with a non-transitive one, a nominative verb.

In the above constructions, the case form of the name (subject and object) depends on the verb, but the name form also predetermines the transitive and non-transitive nature of the verb (especially the objectless one), which is evidence of the analytic relation of the name and the verb.

The order of the subject and object affixes in the verbal form is no less important for transitivity and non-transitivity. Thus, in transitive verbs, the subject affix is preceded by the subject index: \textbf{уы-з-олъагъу} \glossphonemics{wəzawɬaːʁʷ} - "I see you" \textbf{уы-} - the object index in the 2nd person singular and \textbf{-з-} the subject affix in the 1st person singular. Transitive verbs may be bipersonal, three-personal and four-personal. The personal affixes are so arranged: in the two-personal (see above), in the three-personal, the direct object prefix comes first and the indirect object prefix second. The prefix of the direct object comes first, the prefix of the indirect object comes second, and the prefix of the subject comes third: \textbf{сы-къы-п-ху-ишащ} \glossphonemics{səqəpxʷəjʃaːɕ} - "he has brought me to you". The two-person transitive verbs in the reflexive form become monolingual but retain their transitivity: cf. \textbf{абы ар итхьэщIащ} \glossphonemics{aːbə aːr jətħaɕʼaːɕ} "he washed that", \textbf{абы зитхьэщIащ} \glossphonemics{aːbə zəjtħaɕʼaːɕ} "he washed himself" \textbf{абы ар игъэпскIащ} \glossphonemics{aːbə aːr jəʁapskʼaːɕ} "he bathed that"; \textbf{абы зигъэпскIащ} \glossphonemics{aːbə zəjʁapskʼaːɕ} "he bathed himself".

Non-transitive verbs can be both singular and plural: \textbf{Шыр мажэ} - a horse runs - a singular verb; \textbf{ЩIалэр хъыджэбзым lжьэ} - a young man waits for a girl - a two-person verb, etc.

Non-transitive verbs form another syntactic construction, the inverse one, in which the real subject acts as an indirect object and the real object is the grammatical subject: \textbf{Сымаджэм мыIэрысэр хуошх} - The sick manages the apple; \textbf{Студентым тхылъыр иIэщ} - The student has a book. In these sentences, the real subject (\textbf{сымаджэм, студентым}) acts as an indirect object and the real object (\textbf{тхылъыр, мыIэрысэр}) becomes the grammatical subject. Such is the nature of inversion.

The subject-object forms of the verb are a sign of syncretism, but since they represent names that are present in syntactic constructions, they are also analytic in nature.
\subsubsection{Transposition of non-transitive verbs into transitive verbs}
Transitive verbs may be primary (simple) and secondary (derivative). Primary verbs include \textbf{хьын} \glossphonemics{ħən} (\textbf{ехь} \glossphonemics{jaħ}) "to carry", \textbf{дзын} \glossphonemics{d⁀zən} "to throw", and others; secondary verbs are \textbf{кIуэн} \glossphonemics{kʷʼan}, "to go"; \textbf{кIуын} \glossphonemics{kʷʼən}"to pass" and \textbf{гъэкIуэн} \glossphonemics{ʁakʷʼan} "to make go, to send"; \textbf{тхэн} \glossphonemics{txan} "to write"; \textbf{тхын} \glossphonemics{txən} "to write something" and \textbf{гъэтхэн} \glossphonemics{ʁatxan} "to make write something", etc.

The ways of forming secondary transitive verbs are varied:
\begin{xlist}
\ex The causative prefix \textbf{гъэ-} transposes a non-transitive (dynamic and static) verb into a transitive verb: 
	\begin{xlist} 
	\ex \textbf{псэлъэн} \glossphonemics{psaɬan} (\textbf{мэпсалъэ} \glossphonemics{mapsaːɬa}) - "to speak", \textbf{гъэпсэлъэн} \glossphonemics{ʁapsaɬan} (\textbf{егъэпсэлъэ} \glossphonemics{jaʁapsaɬa}) "to make speak, to let repeat";
	\ex \textbf{щытын} \glossphonemics{ɕətən} (\textbf{щытщ} \glossphonemics{cətɕ}) "to stand", becomes \textbf{щы-гъэ-тын} \glossphonemics{ɕəʁatən} (\textbf{щегъэт} \glossphonemics{ɕajʁat}) "to make stand"
	\end{xlist}
\ex Preverb \textbf{къэ} - translates non-transitive into transitive quite actively:

	\begin{xlist}
	\ex \textbf{лъыхъуэн} \glossphonemics{ɬəχʷan} (\textbf{мэлъыхъуэ} \glossphonemics{maɬəχʷa}) "to seek", becomes \textbf{къэлъыхъуэн} \glossphonemics{qaɬəχʷan} (\textbf{къелъыхъуэ} \glossphonemics{qajɬəχʷa}) - "to seek something"
	\end{xlist}
	
\ex The translation of non-transitive into transitive with the preverb \textbf{къэ-} is accompanied by the alternation of the main\textbf{э:ы} in verbs: 
	\begin{xlist}
	\ex \textbf{гупсысэн} \glossphonemics{gʷəpsəsan} (\textbf{мэгупсысэ} \glossphonemics{magʷəpsəsa}) "to think", becomes \textbf{къэгупсысын} \glossphonemics{qagʷəpsəsən} (\textbf{къегупсыс} \glossphonemics{qajgʷəpsəs}) "to invent something"

	\ex \textbf{лэжьэн} \glossphonemics{ɮaʑan} (\textbf{мэлажьэ} \glossphonemics{maɮaːʑa} "to work", becomes \textbf{къэлэжьын} \glossphonemics{qaɮaʑən} (\textbf{къелэжь} \glossphonemics{qajɮaʑ} "to earn something"
	\ex \textbf{дыгъуэн} \glossphonemics{dəʁʷan} (\textbf{мэдыгъуэ} \glossphonemics{madəʁʷa}) - "to steal", becomes \textbf{къэдыгъун} \glossphonemics{qajdəʁʷən} (\textbf{къедыгъу} \glossphonemics{qajdəʁʷ}) "to steal something"
	\end{xlist}

\ex Preverb \textbf{къэ-} with suffix \textbf{-хьы} - accompanied by alternation of basic \textbf{эIы,}, translates non-transitive to transitive:
\begin{xlist} 
\ex \textbf{жэн} \glossphonemics{ʒan } (\textbf{мажэ}  \glossphonemics{maːʒa}) "to run", becomes \textbf{къэжыхьын } \glossphonemics{qaʒəħən} (\textbf{къежыхь}  \glossphonemics{qajʒəħ}) "to run around something"

\ex \textbf{джэдэн} \glossphonemics{d͡ʒadan} (\textbf{мэджэдэ}  \glossphonemics{mad͡ʒada}) "to wander somewhere", becomes \textbf{къэджэдыхьын} \glossphonemics{qad⁀ʒadəħən} (\textbf{къеджэдыхь}  \glossphonemics{qajd͡ʒadəħ}) "to wander without a definite direction".
\end{xlist}
\ex Preverb \textbf{къэ} + suffix \textbf{-хьы-}: 
\begin{xlist}

\ex \textbf{пщын} \glossphonemics{pɕən} (\textbf{мэпщ} \glossphonemics{mapɕ}) "to crawl", \textbf{къэпщын} \glossphonemics{qapɕən} (\textbf{къопщ} \glossphonemics{qawpɕ}) "to crawl here", becomes \textbf{къэпщыхьын} \glossphonemics{qapɕəħən} (\textbf{къепщыхь} \glossphonemics{qajpɕəħ}) "to crawl without definite direction"

\ex \textbf{плъэн} \glossphonemics{pɬan} (\textbf{маплъэ} \glossphonemics{maːpɬa}) "to look", \textbf{къэплъэн} \glossphonemics{qapɬan} (\textbf{къоплъэ} \glossphonemics{qawpɬa}) "to look here", becomes \textbf{къэплъэхьын} \glossphonemics{qapɬaħən} (\textbf{къеплъэхь} \glossphonemics{qajpɬaħ}) "to look without definite direction"

\end{xlist}

\ex Singular verbs with postverb \textbf{-хьы-} with accompanying alternation of basic \textbf{э/ы}:
	\begin{xlist}
	\ex \textbf{къэфэн} \glossphonemics{qafan} (\textbf{къофэ} \glossphonemics{qawfa}) "to dance", becomes \textbf{къэфыхьын} \glossphonemics{qafəħən} (\textbf{къефыхь} \glossphonemics{qajfəħ}) "to dance around something"
	\end{xlist}

\ex Transitive verbs are formed from non-transitive verbs by means of the affix \textbf{-ы-}:
	\begin{xlist}
	\ex \textbf{кIуэн} \glossphonemics{kʷʼan} (\textbf{макIуэ} \glossphonemics{maːkʷʼa}) "to go", becomes \textbf{кIун} \glossphonemics{kʷʼən} (\textbf{екIу} \glossphonemics{jakʷʼ}) "to pass (a separate distance)"
	\ex \textbf{тхэн} \glossphonemics{txan} (\textbf{матхэ} \glossphonemics{maːtxa}) "to write", becomes \textbf{тхын} \glossphonemics{txən} (\textbf{етх} \glossphonemics{jatx}) "to write something"
	\ex \textbf{дэн} \glossphonemics{dan} (\textbf{мадэ} \glossphonemics{maːda}) "to sew", becomes \textbf{дын} \glossphonemics{dən} (\textbf{ед} \glossphonemics{jad}) "to sew something"
	\ex \textbf{къэпсэлъэн} \glossphonemics{qapsaɬan} (\textbf{къопсалъэ} \glossphonemics{qawpsaɬa}) "to speak", becomes \textbf{къэпсэлъын} \glossphonemics{qapsaɬən} (\textbf{къепсэлъ} \glossphonemics{qajpsaɬ}) "to say something"
	\end{xlist}
\end{xlist}

Transitive verbs can transpose into transitive verbs in certain forms:

1) in the potency form with the prefix \textbf{хуэ-}: \textbf{шхын} - to eat, to eat - \textbf{хуошх} - can eat, \textbf{сымаджэм мыIэрысэр ешх} (trans. d.) - the patient eats an apple, but \textbf{сымаджэм мыIэрысэр хуошх} (trans. d.) - the patient manages with an apple;

2) in the involuntary form with the preverb \textbf{IэщIэ-}: \textbf{ЩакIуэм дыгъужьыр къиукIащ} \glossphonemics{ɕaːkʷʼam dəʁʷʑər qəjəwət͡ʃʼaːɕ} (Transl. d.) - The hunter shot the wolf - \textbf{ЩакIуэм хьэр IэщIэукIащ} \glossphonemics{ɕaːkʷʼam ħar ʔaɕʼawət͡ʃʼaːɕ} (Neoter. d.) - The hunter involuntarily shot the dog.

3) in the reciprocal form with the prefix \textbf{зэры-} \glossphonemics{zarə-}: \textbf{Абы ар ешэ} \glossphonemics{aːbə aːr jaʃa} - (transitive) - "He will marry her (here)" - \textbf{Ахэр зэрошэхэ} \glossphonemics{aːxar zarawʃaxa} (intransitive) "They will marry".

4) in the object version with the prefix \textbf{фIэ-} \glossphonemics{fʼa-}: some transitive verbs become non-transitive: \textbf{Абы ар ещIэ} \glossphonemics{aːbə aːr jaɕʼa} - He knows that - \textbf{Абы ар къыфIощI} \glossphonemics{aːbə aːr qəfʼawɕʼ} - He seems that; \textbf{Хамэ хьэдэр жей къыпфIощI} (Ps.) \glossphonemics{ħaːma ħadar ʑaj qəpfʼawɕʼ} - A strange dead man seems to be asleep.

\subsection{Dynamic and static verbs}
Dynamic verbs express the process of action, e.g.: se sothe(r) "I write (in general)"; se ar sothe(r) "I write (that)"; se ar soshe(r) "I lead it"; se abi sojhe(r) "I wait for it".
Static verbs express the state, the result of an action: se syshytshch "I stand"; se ar siyeshch "I have that".
Static verbs are also predicative forms of names and pronouns: ar studentshch "that student (is)"; ar dakheshch "that handsome (is)"; studentr arshch "that student (is)".
Morphologically, Dynamic and Static verbs are distinguished by their Present Form. Dynamic verbs in the present tense have the prefix o-(ue-): se s-o-kIue(r) "I go"; se ar s-o-bzy(r) "I cut that"; se s-o-laje(r) "I work".
In non-transitive plural dynamic verbs (I) and in singular non-transitive verbs with preverbs (II), this prefix is present in the forms of all three persons:






It is also a characteristic feature of dynamical verbs that the optional suffix -r is present. Cf. in the present tense se soc1ue/se socIue-r "I go", se ar sotkh/se ar sotkhy-r "I write that", se aby ar isot/se ar aby isoty-r "I give him that".

Static verbs, on the other hand, are characterized by the absence of the prefix o- (ue-) and by the presence of the copula-suffix -shh: se syshyt-sh-sh "I stand"; se sysstudent-sh "I am a student"; se ar si1e-sh "I have that". The suffix -shh is by origin the root suffix of the verb i-schy-schy-sch "that of them is". In the Past Perfect and Future Perfect tenses, dynamic verbs do not differ from static verbs. The forms of the above tenses are formed from participles by adding to them the copula-suffix -sh or the affix -t, e.g., the Past Perfect Form kIuashch 'that went' is formed from the Past Participle kIua 'went' and the copula-suffix -sh, that is, 'went is'. The remote past tense form kIuat "that went (then)" is formed from the same participle form kIua "went" and the past tense suffix -t, i.e. kIuat lit. "(i.e., the past tense suffix -t, i.e. Iuat literally means "went was.

Hence, the Past Perfect and Future Perfect forms of dynamic verbs with the suffix -sh are Static verbs by formation.

The number of primary static verbs is limited: se sy-schy-t-sch "I stand"; se sy-schy-schy-sch "I sit"; se sy-schy-l-l-sch "I lie down"; aby ar i-Iyg-sch "one holds that"; aby ar i-1e-sch "one has that" ("he has that"); I- shy-sh-shch "one (is) of them"; se aby sy-huy-shch "I want that"; aby ar fe-FI-shch "one desires that"; ar aby shy-g-shch "that is put on him".

Static verbs \textbf{щытын} "to stand", \textbf{щысын} "to sit", \textbf{щылъын} "to lie" may have various preverbs of local meaning, e.g:

\begin{xlist}
\ex \textbf{стэчаныр стIолым те-т-щ} \glossphonemics{stat⁀ʃanər stʼawɮəm tajtɕ} "glass stands on the table"
\ex \textbf{хьэр мэкъум те-лъ-щ} \glossphonemics{ħar makʷəm tajɬɕ} "dog sits on the hay"
\ex \textbf{къазыр бжэм Iy-т-щ} \glossphonemics{qaːzər bʒam ʔʷətɕ} "goose stands by the door"; 
\ex \textbf{мыIэрысэр жыгым пы-т-щ} \glossphonemics{məʔarəsar ʒəɣəm pətɕ}  "apple hangs on the tree"
\ex \textbf{щакIуэр мэ-зым хэ-т-щ} \glossphonemics{ɕaːkʷʼar mazəm xatɕ} "the hunter is (lit. stands) in the forest"
\ex \textbf{Iуэху мыублэ блэ хэ-с-щ} \glossphonemics{ʔʷaxʷ məwəbɮa bɮa xasɕ} "There is a snake in the unstarted work"
\ex \textbf{лIыжьыр унэм щIэ-т-щ} \glossphonemics{ɬʼəʑər wənam ɕʼatɕ} "the old man is in the house" (lit. "stands in the house")
\ex \textbf{Сабыр и щIагъ дыщэ щIэ-лъ-щ} \glossphonemics{saːbər jə ɕʼaʁ dəɕa ɕʼaɬɕ} "Under modesty lies gold"
\ex \textbf{жэмыр пщ1антIэм дэ-т-щ} \glossphonemics{ʒamər pɕʼantʼəm datɕ} "the cow stands in the yard".
\end{xlist}

Stative verbs may sometimes have preverbs of place. For example: \textbf{дыщэр къыщыщIахым щылъапIэщ} \glossphonemics{dəɕar qəɕəɕʼaːxəm ɕəɬaːpʼaɕ} "Where gold is mined, there (it is) expensive [is]"; \textbf{Дзыгъуэр и гъуэм щыхахуэщ} \glossphonemics{d⁀zəʁʷar jə ʁʷam ɕəxaːxʷaɕ} "And the mouse in its burrow (there) is brave (is)".

Static verbs, like dynamic verbs, can be both one-personal (ar shytsh "that one stands", ar pshaschaesh "that girl is") and two-personal (abi ar iyesh "that one has that", abi ar iyygsh "that one holds that"), as well as three-personal (schuiyygsh "that one holds that for me").

Static verbs are usually non-transitive, e.g.: ar shysshch "he sits"; ar shytshch "he stands"; aby ar iyesch "he has that"; ar aby huiysch "he wants that"; aby ar fifeysch "he likes that"; aby ar shygysch "he wears that"; aby ar i gugeysch "he seems that"; ar dakhasch "he is beautiful".

In Kabardian-Circassian there is also a transitive static verb: \textbf{абы ар иIыгъщ} \glossphonemics{aːbə aːr jəʔəʁɕ} "he holds that".

Dynamic verbs, on the other hand, may be both transitive (aby ar etkhy(r) "he writes that", aby ar eshe(r) "he leads him") and non-transitive (ar makIue(r) "he goes", ar aby yojye(r) "he waits for him").

Note. Static verbs in causative formation retain the form of stasis, e.g., the transitive verb shch-i-ge-t-shch, "he makes him stand", which retains the form of stasis, indicated by the absence of the dynamic prefix ue-(o-) and the presence of the copula-suffix -sh, characteristic of static verbs.
In some cases, the same stem may be used to form both Static and Dynamic verbs, e.g.: se syshytsh (stat.) 'I stand' - se syshyotyr (dynam.) 'I stand idle'. The verb "I stand idle"; "I sit", "I sit", "I sit"; "I lie", "I lie"; "I lie", "I lie".

From some names, we can also form both static and dynamic verbs, e.g.: pkhashchie "carpenter" - se sepkhashchieesh (stat.) "I carpenter", se sopkha-shhIer (dyn.) "I carpenter"; egyejakIue "teacher" - se syeggejakIueesh (stat. The verb "to teach" is to say "I teach"; bzaje "wicked" is to say "I am wicked" is to say "I become wicked"; dache "beautiful" is to say "I am beautiful" is to say "I become beautiful" is to say "I become beautiful".

Thus, static verbs differ from dynamic verbs in the system of conjugation as well as in the forms of word formation:

\begin{xlist}

\ex 1) Dynamic verbs here have the aorist form. The Aorist is represented by the simple base, without the characteristic of tense, and occurs:

on the one hand, together with the suffix \textbf{-р} and the union particle \textbf{-и}, when it is followed in a sentence by another verb, e.g.: \textbf{Хамэхьэр къихьэри унэхьэр ирихущ} \glossphonemics{xaːmaħar qəjħarəj wənaħar jərəjxʷəɕ} "The strange dog came and chased away the house dog", \textbf{къихьэри} \glossphonemics{qəjħarəj}. On the other hand, the Aorist is found with the copula-suffix \textbf{-щ}, for example:  \textbf{Хамэхьэр къихьэри унэхьэр ирихущ} \glossphonemics{xaːmaħar qəjħarəj wənaħar jərəjxʷəɕ} "The strange dog came and chased away the owner's dog". The Aorist is very rare in this form.

Static verbs cannot form the Aorist. This is common to all Iberian-Caucasian languages.

The Aorist of a dynamic verb coincides with the Present tense of a Static verb. For example, in the sentence: \textbf{сыщытщ, сыщытри сыкъэкIуэжащ} \glossphonemics{səɕətɕ səɕətrəj səqakʷʼaːɕ} "I stood, stood and went" syshytshch "I stood" is a dynamic verb in the Aorist (the present tense is syshotyr "I stand idle") and in se mbdezh syshytshch "I stand here" syshytshch is a static verb in the present tense.

\ex The past imperfect form of dynamic verbs, unlike the form of static verbs, is characterized by the presence of the facultative suffix -r. A dynamic verb in the past imperfective se sykIuet "I went" may have a parallel form with the suffix -r-: se sykIuert "I went"; \textbf{къэбэрдей жылэр Iейуэ гузавэрт} \glossphonemics{qabardaj ʒəɮar ʔejwa gʷəzaːvart} "the Kabardian people was very worried".

The suffix \textbf{-р-} in the past imperfect \textbf{сыкIуэрт} \glossphonemics{səkʷʼart} "I went" is the same optional present tense suffix of dynamic verbs: \textbf{сэ сокIуэ(р)} \glossphonemics{sa sawkʷʼa(r)} "I go".

Static verbs in the past tense do not have this facultative suffix \textbf{-р-}: \textbf{сэ сы-щытт} \glossphonemics{sa səɕətt} "I stood", \textbf{сэ сыщылът} \glossphonemics{sa səɕəɬt} "I lay", \textbf{сэ сыучителт} \glossphonemics{sa səwət⁀ʃəjtajɬt} "I was a teacher".

\ex Primary Static verbs are found only with preverbs, e.g., 
\begin{xlist}
\ex \textbf{Iy-т-щ} \glossphonemics{ʔʷətɕ} "he stands (near something)"; \textbf{Iy-c-щ} \glossphonemics{ʔʷəsɕ} "he sits (near something)"; \textbf{ly-лъ-щ} \glossphonemics{ʔʷəɬɕ} "he lies (near something)"; 
\ex \textbf{те-т-щ} \glossphonemics{tajtɕ} "he stands (on something)"; \textbf{те-с-щ} \glossphonemics{tajsɕ} "he sits (on something)"; \textbf{те-лъ-щ} \glossphonemics{tajɬɕ} "he lies (on something)".
\end{xlist}

Dynamic verbs, on the other hand, have both 

\begin{xlist}
\ex preverbs (\textbf{Iy-дэ-н} \glossphonemics{ʔʷədan} (\textbf{Iуедэ} \glossphonemics{ʔʷajda}) "to sew to something", \textbf{хэ-тхэ-н} \glossphonemics{xatxan} (\textbf{хетхэ} \glossphonemics{xajtxa}) "to write into something", \textbf{1у-хы-н} \glossphonemics{ʔʷəxən} (\textbf{Iуех} \glossphonemics{ʔʷajx}) "to open") 
\ex and no preverbs (\textbf{содэ} \glossphonemics{sawda} "I sew", \textbf{солажьэ} \glossphonemics{sawɮaːʑa} "I work", \textbf{сожэ} \glossphonemics{sawʒa} "I run").
\end{xlist}

\textbf{Note.} Some dynamic verbs that express motion, like the static ones, do not occur without preverbs. For example: 
\begin{xlist}
\ex \textbf{сы-д-о-кI} \glossphonemics{sədawkʼ} "I go out from somewhere", \textbf{сы-бл-о-кI} \glossphonemics{səbɮawkʼ} "I pass by something", \textbf{сы-щI-о-хьэ} \glossphonemics{səɕʼawħa} "I enter something", \textbf{й-о-хьэ} \glossphonemics{jawħa} "he enters something".
\end{xlist}

\ex 4) Primary static verbs are usually characterized by preverbs The primary static verbs are usually characterized by preverbs of local meaning, e.g., ly-t-sh "standing", te-t-sh "standing (on the surface of something)", sh-Ie-t-sh "standing under". With dynamic verbs, on the other hand, both local preverbs and preverbs of motion are used: sy-bl-o-kI "I pass (by)", ph-o-kI "passes (through)", sy-k-o-kIue "go here", sy-n-o-kIue "go there".

\textbf{Note.} Sometimes the preverbs of direction kъ- (kъ-) are also used with static verbs, but together with preverbs of place Iu-, he-, shy-: si gupemkIe kъ- shy-s-shy lit. "The following is an example of this: \textbf{щхьэгъубжэм къыIу-т-щ} \glossphonemics{ɕħaʁʷbʒam qəʔʷətɕ} lit. "standing by the window (here)".

\ex (5) In the modern language, the structure of the present participle of dynamic verbs is the same as that of the participle of static verbs. In both cases the base of the participle coincides with the base of the present verb: \textbf{кIуэ-р} \glossphonemics{kʷʼar} 'going' (the participle from the dynamic verb \textbf{со-кIуэ} \glossphonemics{sawkʷʼa} "I go"), \textbf{щыты-р} \glossphonemics{ɕətər} 'standing' (the participle from the static verb \textbf{сы-щыт-щ} \glossphonemics{səɕətɕ} "I stand"). But historically the forms of the Static and Dynamic Participles have been different. The present participle of dynamic verbs is formed with the help of the suffix \textbf{-рэ}: \textbf{кIуэ-рэ-р}, "going," \textbf{зы-тхы-рэ-р} \glossphonemics{zətxərar}, "writing. And participles from static verbs were formed without any suffix: \textbf{щытыр} "standing" and \textbf{щысыр} "sitting", that is, as they are formed now: \textbf{Псым Iусым икIып1э ещIэ} \glossphonemics{psam ʔʷəsəm jəkʼəpʼa jaɕʼa} "He who lives (sitting) by the river, knows the ford".

The participles of dynamic verbs with the suffix \textbf{-рэ} are preserved in the interrogative forms of the present tense: \textbf{уэ у-кIуэ-рэ?} \glossphonemics{wa wəkʷʼara} "are you going?" (lit. "are you going?"), \textbf{уэ п-тхы-рэ?} \glossphonemics{wa ptxəra} "do you write?" (lit. "are you writing?").

The present tense question forms of static verbs, on the other hand, do not have the -re suffix: ue ushyt? "are you standing? (lit., "are you standing"); üe ustudent? "are you a student?" etc.
\end{xlist}

\subsection{Finite and infinitive verb forms}
\subsection{Face category}
\subsection{Verb paradigms}
\subsection{Tense category}
\subsubsection{Preliminary Remarks}
The temporal forms of the Adyghe languages are analyzed separately not only in descriptive grammars, but also in special studies1. At the same time, not to mention the diachronic aspect of the problem, many questions of the synchronic description of the category of time remain underdeveloped. There is no consensus among researchers even on the question of the number of temporal forms in the Adyghe languages.

In recent descriptive grammars of the Kabardian-Circassian language the concept dominates, according to which two groups of tenses are distinguished. The first group includes tenses expressing the relation of the time of action to the moment of speech, the second group includes tenses showing the relation of the time of action to a certain moment in the past. In accordance with this there are distinguished the present first, the present second, the past first, the past second, the future first, the future second. The morphological indicator of the second group of tenses is considered to be the formant -t2.

The classification of tenses is also based on contrasting the -t-forms with the tenses of another group. At the same time, such a principle of classification remains disputable. The grouping of the tenses according to the above principle presupposes the uniqueness of the -t formative in all the tenses, which is necessary for the formation of the temporal opposition in the indicative by the differential sign - form on -t: form without -t. Meanwhile, the analysis of the material shows that the formative -t is ambiguous in combination with bases of different tenses. Cf.: kIuert "he was going then", but kIuenut "he would like to go". In other words, not all the tenses with the formative -t have an indicative meaning. If forms kIuert, kIuat, kIuegyat are indicative (indicative mood), kIuent, kIuenut have the meaning of conjunctive (subjunctive mood).

According to the classification under consideration, kIuert "he was going then" is a form of the present tense, because it represents an action in its course in the past. Not mentioning the fact that expressing an action (process) in its course (flowing) in the past is a function of the imperfect, from a grammatical point of view the kIuert falls into the paradigmatic series of the past tense. Cf. e.g:
.......

As may be seen, the paradigmatic series (a) differs from the paradigmatic series (b) and (c), which are absolutely identical in this respect, in the form of the 3rd person and the glossation of the 1st and 2nd person affixes. Series (c) coincides with other past tense forms in the glossing of the root morpheme. Cf.: \textbf{сы-кIуэ-рт} \glossphonemics{səkʷʼart} "I went back then", \textbf{сы-кIуэ-щ} \glossphonemics{səkʷʼaɕ} "I went" (Aorist), \textbf{сы-кIуэ-ри} \glossphonemics{səkʷʼarəj} '"I went and ...". (aorist). A strong proof that formations of the kIue(r)t type "he (then) walked (then)" are imperfect is their functional identity with the Adygean imperfect in \textbf{-щтыгъэ} \glossphonemics{-ɕtəʁa}.

The question of classifying forms of tenses is less debatable with respect to the Adygean language, although even here there is no unanimity among specialists.
\subsubsection{Present Tense}
\subsubsection{Future 1}
The characteristic of the base of Future I is the formant \textbf{-н}. The affirmative form of future I is marked with the suffix \textbf{-щ}. Cf.: Adyg. \textbf{сы-кIуэ-н}, kab. \textbf{сы-кIуэ-н-щ} 'I will go', Adyg. \textbf{схьы-н}, kab. \textbf{схьы-н-щ} 'I will carry'. Examples:

\textbf{Мурад щэхуу уэ блэжьынур Мис ныщхьэбэ къэсхутэнщ} \glossphonemics{məwraːd ɕaxʷəw wa bɮaʑənəwr məjs nəɕħaba qasxʷətanɕ} "What you have planned to carry out secretly, I will find out tonight."

In two dialects, Kuban and Beslaneev, the affirmative form lacks the suffix \textbf{-щ}. Future I in Beslenevi dialect (as well as in Adygean) is marked by the formant \textbf{-н} and in the Kuban dialect by the formant \textbf{-нэ}. Cf.: Besl. \textbf{сщIын} \glossphonemics{sɕʼən}, Kub. \textbf{сщIынэ} \glossphonemics{sɕʼəna} "I will do it".

Future I was formed during the era of common Adyghe language unity. The common Adyghe form of the future I is preserved in Adygean and Beslenev dialect. The variant of the suffix future I with the vowel \textbf{э}, typical for the Kuban dialect, is a is an innovation that appeared after dialectal differentiation of Kabardian. The suffix \textbf{-sh} in the affirmative form of dialects is also a new formation.

There is still no satisfactory explanation in the specialized literature for the genesis of the common Adyghe suffix of future I \textbf{-н}. There is no reason to agree with the opinion of N.F. Yakovlev and D.A. Ashkhamaf, according to which the suffix in question genetically ascends to the common Adyghe word \textbf{нэ} "eye", "hole". At the present stage of the study of Adyg languages does not seem possible to solve the question of the origin of common Adyghe formant of future I \textbf{-н}. It may be assumed that the \textbf{-н} form is not finitish, but infinitive, with a meaning of intention, purpose or intention to be. It is in this function that the form is widely used in the modern Adyghe languages. In this connection, attention should also be paid to the Ubykh language suffix \textbf{-н} in the future purpose and intention. Cf. ubykh. \textbf{айнащаутын} \glossphonemics{ajnaɕawtən} "(they) to do".
\subsubsection{Future 2}
Future II is formed from the base of Future I with the suffix \textbf{-у}. The affirmative form of Future II is marked by the suffix \textbf{-щ}. Cf.: \textbf{сы-кIуэ-ну-щ} \glossphonemics{səkʷʼanəwɕ} "I will go", \textbf{с-хьы-ну-щ} \glossphonemics{sħənəwɕ} 'I will carry'.

It is typical that there is no paradigmatic parallelism between future I and future II. Cf. e.g. in Kabardinian the absence of the infinitive (union) form of the future II in the indicative: \textbf{сыкIуэнщи} \glossphonemics{səkʷʼanɕəj} "I will go and", but \textbf{сыкIуэнущи} \glossphonemics{səkʷʼanəwɕəj} "as I will go". In other words, the opposition of the future I to the future II is not realized in all the forms, which also testifies to the appearance of this temporal opposition in a later era of the development of the Adyg languages.

Thus, future II, unlike future I, cannot be attributed to the all-Adyg languages unity. At the same time, the origin of the formants of Future II remains unclear, despite the fact that the temporal form itself developed after the differentiation of the basic language into separate dialects.
\subsubsection{Perfect 1}
The base of the Perfect I is formed with the suffix \textbf{-а}. Cf.: \textbf{зд-а-щ} \glossphonemics{zdaːɕ} "I sewed", \textbf{сыщыс-а-щ} \glossphonemics{səɕəsaːɕ} "I sat".

The perfective suffix \textbf{-а} is an innovation, arisen in the individual development of Adyghe languages on the basis of the original \textbf{-гъэ}. The latter goes back to the common Adyghe language unity.

The original perfective suffix \textbf{-гъэ} can be traced in formations such as \textbf{тхыгъэ} \glossphonemics{txəʁa} "writing" , \textbf{тыгъэ} \glossphonemics{təʁa} "gift", \textbf{бжыгъэ} \glossphonemics{bʒəʁa} "number", \textbf{къэкIыгъэ} \glossphonemics{qat⁀ʃʼəʁa} "plant". The latter are a perfective form of the participle1. The formation of the perfect is accompanied by phonetic changes in the stem. Final etymological vowels \textbf{э}, \textbf{ы} are absorbed by a long vowel a, which serves as a perfective suffix. Cf.: \textbf{зд-а-щ} <- \textbf{*зды-а-щ} "I sewed", \textbf{сы-кIу-а-щ} <- \textbf{*сы-кIуэ-а-щ} "I walked".

The question about the genesis of the perfective suffix \textbf{-гъэ} remains unclear. Neither R. Erkert, who attributed \textbf{-гъэ} to the adverb \textbf{дыгъуасэ} , nor N.F. Yakovlev or D.A. Ashkhamaf, who identified it with the base \textbf{гъэ} 'year' , gave any convincing explanation of the etymology of the suffix in question. Dumezil's close connection of the common Adyghe perfection suffix \textbf{-гъэ} with Ubykh perfection suffix \textbf{-къа} and Abkhazian pluperfect suffix \textbf{-хъа} is not without interest, although there is not enough material for genetic unity of these suffixes yet. Cf. Adyg. \textbf{сыкIуа-гъ}, Ubykh. \textbf{сыкIьа-къа,}, Abkh. \textbf{сцахъейгпI} "I went".
\subsubsection{Perfect 2}
Perfect II is formed from Perfect I by means of the suffix \textbf{-т}, indicating that the action took place in a limited time. Cf. \textbf{сыкIуащ} "I went", \textbf{сыкIуат} "I went then".
\textbf{Нартхэр мы щIыпIэм исат, Нарт Сосрыкъуэ и джатэр А зэман жыжьэм щыбзат} \glossphonemics{naːrtxar mə ɕʼəpʼam jəsaːt naːrt sawsrəqʷa jə d⁀ʒaːtar aː zamaːn ʒəʑam ɕəbzaːt} "Narts lived in this land, Sosryko (lit. Sosryko's saber) performed feats of arms here in those distant times"; \textbf{ЩIэсэныгъэм гур щигъэбзэрабзэм ПцIы зыхэмылъ усэхэр бжесIат} \glossphonemics{ɕʼasanəʁam gʷər ɕəjʁabzaraːbzam pt⁀sʼə zəxaməɬ wəsaxar bʒajsʼaːt} "When my heart rejoiced with love, I read you sincere verses".
Perfekt II is a Kabardian new formation.
\subsubsection{Plusquamperfect 1}
\subsubsection{Plusquamperfect 2}
\subsubsection{Aorist}
\subsection{Inflectional category}
\subsection{Negative forms}
\subsection{Interrogative forms}
\subsection{Ways of Verbal Formation}
\subsection{Factitive Verbs}
\subsection{Causative category}
\subsection{Unionality category}
\subsection{Jointness category}
\subsection{reciprocity}
\subsection{Version category}
\subsection{Category potency (possibility)}
\subsection{Category involuntariness}
\subsection{Directional and local prefixes (preverbs)}
\subsection{Ablauts in verbal word formation}

\section{Deverbal Formations}
\section{Adverb}
\section{Utility words}
\section{Interjections}
\chapter{Syntax}
\end{document}

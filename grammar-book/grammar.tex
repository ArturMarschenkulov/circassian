\documentclass[a4paper,11pt]{book}

% Load necessary packages
\usepackage{fontspec}
\setmainfont{Charis SIL}

% % Set main and monospaced font
% \setmainfont{Times New Roman}
% \setmonofont{Courier New}

% % Set language
% \setdefaultlanguage{english}

\begin{document}

\title{Comprehensive Descriptive Grammar of Eastern Circassian (Kabardian)}
\author{Artur Marschenkulov}
\date{\today}

\maketitle

\tableofcontents


\chapter{Introduction}
\section{Goal}
The goal of this book is to provide a comprehensive description of the Eastern Circassian language, i.e. Kabardian. It is intended to be a living document that will be continuously worked on and updated. The intended target audience are linguists, language learners, and native speakers of Kabardian.

Because of the wide range of audiences, one has to strike a balance between being linguistically precise and being accessible to non-linguists. Thus this book will contain a lot of examples, which might be redundant for linguists, but will be helpful for non-linguists.

The book will be divided into three parts. The first part will be a general introduction to the language, its history, and its speakers. The second part will be a comprehensive description of the grammar of the language. The third part will be a learning guide specifically targeted at language learners, where a big emphasis will be put on expressing certain ideas and explaining certain quirks and concepts in a way that is easy to understand for non-linguists.

\part{Society, Culture and History}
\part{Grammar}
\chapter{Phonology}
\chapter{Morphology}
\chapter{Syntax}
\chapter{Semantics}
\chapter{Pragmatics}
\part{Learning Guide}

\end{document}
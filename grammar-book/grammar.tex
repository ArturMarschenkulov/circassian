
% document which as two columns on one page.
\documentclass[a4paper, 10pt]{book}
% \documentclass[a4paper, twocolumn, 10pt]{book}
\usepackage[margin=3cm]{geometry}

\usepackage{gb4e}
\usepackage{hyperref}
\hypersetup{
    colorlinks = true,
    linkcolor = blue
}

% Load necessary packages
\usepackage{fontspec}
\setmainfont{Charis SIL}

% % Set main and monospaced font
% \setmainfont{Times New Roman}
% \setmonofont{Courier New}

% % Set language
% \setdefaultlanguage{english}

\begin{document}

\title{Comprehensive Grammar of Eastern Circassian (Kabardian)}
\author{Artur Marschenkulov}
\date{\today}


% all my new commands start with `my'
% this creates a command which takes 3 arguments, the first one is the word in the latin script, the second one is the word in the cyrillic script, and the third one is the translation of the word
% \newcommand{\mywordrce}[3]{\textit{#1} -- #2 -- `#3'}
\newcommand{\mywordcre}[3]{\textbf{#1} \textit{#2} `{#3}'}
\newcommand{\mywordcr}[2]{\textbf{#1} \textit{#2}}
\newcommand{\myworde}[1]{`{#1}'}



\maketitle

\tableofcontents


\chapter{Introduction}
% Preface
% \section{Goal and motivation}
This book is intended to be a living document that will be continuously worked on and updated, thus it will be incomplete for a long time. However, I believe that it is better to have something incomplete than nothing at all. I hope that it motivates others to contribute to this book. Note, that I intend it to be mainly an online document, thus calling it a book might be out of place, nonetheless this document will be referred to as a book. 

Because of the nature of this book, it is important to create guidelines as to what this book is about. I decided to write this book with the main overarching goal to make it easier for others to understand Circassian better, be it from a linguistics' or learner's point of view. In my view, it is a big problem that there are so little resources the Circassian language in English, with most good resources being in Russian, Turkish or Circassian itself. While there are still some very good resources in English, they are either incomplete or not very accessible to non-linguists. This is bad, because there is a large Circassian diaspora which doesn't speak Circassian anymore, but wants to relearn the language of their ancestors. Having resources fractured into various languages makes it harder for them to learn Circassian and creates a much bigger barrier than necessary. Since English is the lingua franca of the world, it is the most understandable language for most people.

Because of this, I decided to write a comprehensive grammar. While reading a grammar might not be the most productive way to learn a language, I think in the context of Circassian it is a worthwhile endeavor. The main reason is simply because of how different the language structure is from most other languages (ergative agreement, polypersonality, preverbs, etc). Thus having a comprehensive grammar can be very helpful for linguists and language learners alike to speed up one's understanding.

I intend to write this book in a way that it can be used as a reference for linguists and language learners. First of all, the idea is that the book would have two big parts, the grammar part and the learning part. The grammar part will resemble most other comprehensive grammars, however with the explicit goal to also be a potential learning resource. This would inlcude many examples, maybe even in excessive amount, which might be too much for a  comprehensive grammar, however it might give a better feel for the language. In addition, while the book will try to keep a linguistics appropriate language, an attempt will be made to explain the various concepts. The learning part will be much more focused on simply understanding how to form various phrases and meanings without a big emphasis on the grammatical structure.

This book will be about Eastern Circassian (Kabardian), however it will also contain information about Western Circassian (Adyghe) where it is relevant, as well as Ubykh, Abkhaz and Abaza, as comparing them can create more insight. Maybe there will be even their own parts in the book, but that is not certain yet. A secondary goal is to promote everything which is adjacent to the Circassians in some way.

Besides being a linguistics resource, I also intend to include topics about history, culture and society.

For the most part, this book will mainly compile other resources into one place, such that there is a central place for all the information. However, own research will be included as well.
% % About me
% Another function of this book is compile my Circassian language learning into one source. This means I am not a native speaker, however I am of Circassian descent and thus I have a personal connection to this language. This will probably assure that work on this project will never be completely halted, at worst only paused. Also keep in mind, I'm not a linguist, however I had big interest in linguistics since my childhood.

% I cannot guarantee that everything will be correct. If you find any mistakes, please contact me.


\part{Society, Culture and History}

\part{Grammar}
\part{Phonology}
\part{Morphology}

\chapter{Nouns}
\section{Case}
Kabardian has 4 cases, absolutive (traditionally also called nominative) (\mywordcr{-р}{-r}), ergative (\mywordcr{-м}{-m}), instrumental (\mywordcr{-кIэ}{-tʃʼa}) and adverbial (\mywordcr{-уэ}{-wa}). The case markers are not part of the stem and are only suffixed when the noun is used in a sentence.

Those 4 cases are divided into primary and secondary cases. The primary cases are only used if the noun is definite, while they are absent if the noun is indefinite (basically \myworde{the man} vs \myworde{a man}). The secondary cases are morphological build upon the primary cases. This can be seen in definite nouns, where the instrumental case suffix (\mywordcr{-кIэ}{-tʃʼa}) is added to the ergative suffx (\mywordcr{-м}{-m}) resulting in \mywordcr{-мкIэ}{-mtʃʼa} and the adverbial case suffix (\mywordcr{-уэ}{-wa}) is added to the nominative suffix (\mywordcr{-р}{-r}) resulting in \mywordcr{-рaуэ}{-raːwa}.

\begin{table}[h]
\centering
\begin{tabular}{l|l|l|l|l}
& Absolutive & Ergative & Instr. & Adv. \\
\hline
Indefinite & \mywordcr{-∅}{-∅} & \mywordcr{-∅}{-∅} & \mywordcr{-кIэ}{-tʃʼa} & \mywordcr{-уэ}{-wa}/\mywordcr{-у}{-wə} \\
Definite & \mywordcr{-р}{-r} & \mywordcr{-м}{-m} & \mywordcr{-мкIэ}{-mtʃʼa} & \mywordcr{-рауэ}{-raːwa} \\
\end{tabular}
\end{table}


Below are a few examples:
\begin{exe}
\ex
\begin{xlist}
    \ex \mywordcre{пхъащIэм пхъэбгъухэр куэбжэу ищIащ}{pχaːɕʼam pχabʁʷxar kʷabʒawə jəɕʼaːɕ}{the carpenter made a gate out of planks} (lit. \myworde{the carpenter made planks like a gate})
\end{xlist}
\end{exe}

There is also a special interaction with the plural forms of a noun, as in their infinite form they only have the adverbial form, while in their definite form they have all 4 cases. This is inline with the fact that marking the plural is optional and thus also inherently definite.

\begin{table}[h]
\centering
\begin{tabular}{l|l|l|l|l}
& Absolutive & Ergative & Instr. & Adv. \\
\hline
Indefinite & not used & not used & not used & \mywordcr{-хэу}{-xawə} \\
Definite & \mywordcr{-хэр}{-xar} & \mywordcr{-хэм}{-xam} & \mywordcr{-хэмкIэ}{-xamtʃʼa} & \mywordcr{-хэрауэ}{-xaraːwa} \\
\end{tabular}
\end{table}

However, the plural indefinite form is used in a vocative sense, which is used to address someone. This is also the only case where the plural indefinite form is used, e.g. .

\begin{exe}
\ex
\begin{xlist}
    \ex \mywordcre{щIалэхэ, къызэдэIуэ}{ɕʼaːɮaxa qəzadaʔʷa}{boys, listen to me}
    \ex \mywordcre{щIалэхэ, унэмкIэ фынеблагъэт}{ɕʼaːɮaxa wənamtʃʼa fənajbɮaːʁat}{boys, visit the house}
\end{xlist}
\end{exe}


Similar to the plural, other word forms also may require case marking to be grammatical. However note, that it is more accurate to say that not the noun, but the noun phrase is marked for the case.


\begin{exe}
\ex
\begin{xlist}
    \ex \mywordcre{сосрыкъуэ нартхэм мафIэ къазэрыхуихьар}{sawsrəqʷa naːrtxam maːfʼa qaːzarəxʷəjħaːr}{how Sosruko brought fire to the Narts}
    \ex \mywordcre{хъуэжэ псым къызэрикIыжар}{χʷaːʒa psəm qəzərəjtʃʼəʒaːr}{how Khaja returned from the river}
\end{xlist}
\end{exe}


\subsection{Absolutive}
The absolutive case is marked by the suffix \mywordcr{-р}{-r}. As is usual for most ergative languages, the absolutive is used as the subject of an intransitive verb and as the direct object of an transitive verb.


It denotes the subject of an intransitive verb.
\begin{exe}
\ex
\begin{xlist}
    \ex \mywordcre{\underline{хъыджэбзыр} мэбауэ}{\underline{χədʒabzər} mabaːwa}{\underline{the girl} breaths}
    \ex \mywordcre{\underline{фызыр} мэзым йоплъ}{\underline{fəzər} mazəm jawpɬ}{\underline{the woman} looks at the moon}
\end{xlist}
\end{exe}

It denotes the direct object of a transitive verb.
\begin{exe}
\ex
\begin{xlist}
    \ex \mywordcre{сэ къэслъыхъуа \underline{лIыр} къэзгъуэтащ}{sa qasɬəχʷaː \underline{ɬʼər} qazʁʷataːɕ}{I have found \underline{the man} whom I was looking for}
\end{xlist}
\end{exe}

It denotes the nominal predicate in equative sentences.
\begin{exe}
\ex
\begin{xlist}
    \ex \mywordcre{сэ тыкуэным згъэкIуар \underline{си къуэшырщ}}{sa təkʷanəm zʁakʷʼaːr \underline{səj qʷaʃərɕ}}{whom I have sent to the store \underline{is my brother}}
    \ex \mywordcre{а хъыджэбз дахэр си нысэрщ}{aː χədʒabz daːxar səj nəsarɕ}{this pretty girl is my bride}
    \ex \mywordcre{си къуэр \underline{щакIуэрщ}}{səj qʷar \underline{ɕaːkʷʼarɕ}}{my brother \underline{is the hunter}}
\end{xlist}
\end{exe}


\subsection{Ergative}

If used with transitive verbs it denotes the subject.
\begin{exe}
\ex
\begin{xlist}
    \ex \mywordcre{\underline{лIым} дыгъужь илъэгъуащ}{\underline{ɬʼəm} dəʁʷəʑ jəɬaʁʷaːɕ}{\underline{the man} saw a wolf}
\end{xlist}
\end{exe}

It also denotes the indirect object of intransitive and transitive verbs (mainly because the indirect object is governed by the preverb).

\begin{exe}
\ex
\begin{xlist}
    \ex \mywordcre{хьэр \underline{щIалэм} еплъащ}{ħar \underline{ɕʼaːɮam} japɬaːɕ}{the dog looked at \underline{the boy}}
\end{xlist}
\end{exe}

\begin{exe}
\ex
\begin{xlist}
    \ex \mywordcre{щIалэм мыIэрысэ \underline{тхьэмадэм} иритащ}{ɕʼaːɮam məʔarəsa \underline{tħamaːdam} jərətaːɕ}{the boy gave \underline{the Tkhamada} an applied}
\end{xlist}
\end{exe}

It can also be used to create locative adverbs.

\begin{exe}
\ex
\begin{xlist}
    \ex \mywordcre{уэ \underline{къалэм} письмо птха?}{wa \underline{qaːɮam} pəjsmaw ptxaː}{did you write the letter \underline{to the city}?}
    \ex \mywordcre{\underline{къуажэм} усшэнщ}{\underline{qʷaːʒam} wəsʃanɕ}{I'll lead you \underline{to the village}}
\end{xlist}
\end{exe}

It can also be used to create temporals adverbs.

\begin{exe}
\ex
\begin{xlist}
    \ex \mywordcre{\underline{жэщым} хъэр мэбанэ}{\underline{ʒaɕəm} ħar mabaːna}{the dog barks \underline{at night}}
\end{xlist}
\end{exe}

It is also used in possive constructions to denote the possessor.
\begin{exe}
\ex
\begin{xlist}
    \ex \mywordcre{фызым и нэр}{fəzən jə nar}{the woman's eyes}
\end{xlist}
\end{exe}


\subsection{Instrumental}


It can denote an instrumental meaning.
\begin{exe}
\ex
\begin{xlist}
    \ex \mywordcre{сабийр \underline{къалэмкIэ} матхэ}{saːbəjr qaːɮamtʃʼa maːtxa}{the child writes \underline{with the pen}}
    \ex \mywordcre{дэ тенджызым \underline{кхъухьлъатэкIэ} дылъэтащ}{da tajndʒəzəm \underline{qχʷəħɬaːtatʃʼa} dəɬataːɕ}{we flew to the sea \underline{with a plane}}
\end{xlist}
\end{exe}


It can denote the direction.

\begin{exe}
\ex
\begin{xlist}
    \ex \mywordcre{бгым тет \underline{унэмкIэ} маплъэ}{bɣəm tajt \underline{wənamtʃʼa} maːpɬa}{he looks \underline{towards the house} standing on a hill}
\end{xlist}
\end{exe}

\subsection{Adverbial}



\chapter{Pronouns}
\section{Personal Pronouns}
Kabardian has only four personal pronouns. They mark the combination of singular or plural, and first or second person. There is no distinction between inclusive and exclusive first person plural. The third person is marked by demonstrative pronouns.

A peculiarity is that personal pronouns do not differentiate between ergative and absolutive case. One of the reasons might be that the ergative and absolutive case markers also express definiteness, which is redundant with the personal pronouns as they are inherently definite. This can be seen by the example below:
\begin{exe}
\ex
\begin{xlist}
    \ex \mywordcre{сэ сокIуэ}{sa sawkʷʼa}{I go}
    \ex \mywordcre{уэ уощI}{wa wawɕʼ}{you do}
    \ex \mywordcre{сэ уэ узохь}{sa wa wəzawħ}{I carry you}
    \ex \mywordcre{уэ сэ къызыбот}{wa sa qəzəbawt}{you give me Y}
\end{xlist}
\end{exe}

\begin{table}[h]
\centering
\begin{tabular}{l|l|l|l}
& Erg-Abs & Instr. & Adv. \\
\hline
1. sing. & \mywordcr{сэ}{sa} & \mywordcr{сэркIэ}{sartʃʼa} & \mywordcr{сэрауэ}{saraːwa} \\
2. sing. & \mywordcr{уэ}{wa} & \mywordcr{уэркIэ}{wartʃʼa} & \mywordcr{уэрауэ}{waraːwa} \\
\hline
1. pl. & \mywordcr{дэ}{da} & \mywordcr{дэркIэ}{dartʃʼa} & \mywordcr{дэрауэ}{daraːwa} \\
2. pl & \mywordcr{фэ}{fa} & \mywordcr{фэркIэ}{fartʃʼa} & \mywordcr{фэрауэ}{faraːwa} \\
\end{tabular}
\end{table}

There exist also extended forms of the respective personal pronouns, characterized by the addition of the suffix \mywordcr{-р}{-r} or \mywordcr{-рa}{-raː}. The instrumental and adverbial case forms use the extended versions.

This happens in predicative use, e.g. \mywordcre{сэращ}{saraːɕ}{I am}.

This happens if it is followed by a postposition, e.g. \mywordcre{сэр папщIэ}{sar paːpɕʼa}{for me}.

This happens if used in a duplicated form, e.g. \mywordcre{сэр-сэру}{sar-sarəw}{me myself}.

\section{Possessive Pronouns}
\section{Demonstrative Pronouns}
\section{Interrogative Pronouns}

\chapter{Adjective}
\section{Qualitative Adjectives}


\begin{exe}
\ex
\begin{xlist}
    \ex color:
    \begin{xlist}
        \ex \mywordcre{плъыжь}{pɬəʑ}{red}
        \ex \mywordcre{хужь}{xʷəʑ}{white}
        \ex \mywordcre{щхъуэ}{ɕχʷa}{gray}
        \ex \mywordcre{щIыху}{ɕʼəxʷ}{blue}
        \ex \mywordcre{щхъуантIэ}{ɕχʷaːntʼa}{green}
    \end{xlist}
    \ex fell/hair color:
    \begin{xlist}
        \ex \mywordcre{гъуэ}{ʁʷa}{red}
        \ex \mywordcre{къуэху}{qʷaxʷ}{brown-gray}
        \ex \mywordcre{брул}{bʁəwɮ}{chaly}
        \ex \mywordcre{тхъуэ}{tχʷa}{bulany}
    \end{xlist}
    \ex spatial and temporal qualities:
    \begin{xlist}
        \ex \mywordcre{ин}{jən}{big}
        \ex \mywordcre{бгъуэ}{bʁʷa}{wide}
        \ex \mywordcre{жыжьэ}{ʒəʑa}{far}
        \ex \mywordcre{лъахъшэ}{ɬaːχʃa}{low}
        \ex \mywordcre{куу}{kʷəw}{deep}
    \end{xlist}
    \ex properties and qualities of things directly perceived by the senses:
    \begin{xlist}
        \ex \mywordcre{IэфI}{ʔʷəfʼ}{sweet}
        \ex \mywordcre{сыр}{sər}{bitter}
        \ex \mywordcre{щабэ}{ɕaːbə}{soft}
        \ex \mywordcre{хъурeй}{χʷəraj}{round}
        \ex \mywordcre{псынщIэ}{psənɕʼa}{light}
        \ex \mywordcre{хуабэ}{χʷaːbə}{warm}
    \end{xlist}
    \ex physical qualities of people and animals:
    \begin{xlist}
        \ex \mywordcre{щIалэ}{ɕʼaːɮa}{young}
        \ex \mywordcre{уэд}{wad}{thin}
        \ex \mywordcre{дэгу}{dagʷ}{deaf}
        \ex \mywordcre{хуахуэ}{xʷaːxʷa}{brave}
    \end{xlist}
    \ex qualities of character and mental disposition:
    \begin{xlist}
        \ex \mywordcre{бзаджэ}{bzaːdʒa}{evil, cunning}
        \ex \mywordcre{пагэ}{paːɣa}{proud}
        \ex \mywordcre{нэпсей}{napsaj}{greedy}
    \end{xlist}
\end{xlist}
\end{exe}


\section{Relative Adjectives}
All relative adjectives are derived in Kabardian. Relative adjectives, unlike qualitative adjectives, do not change by degrees of comparison, do not form the form of evaluations.
Native relative adjectives can express:
\begin{exe}
\ex
\begin{xlist}
    \ex relation of time:
    \begin{xlist}
        \ex \mywordcre{нобэрей}{nawbaraj}{today} (cf. \mywordcre{нобэ}{nawba}{today})
        \ex \mywordcre{дыгъуэпшыхьырей}{dəʁʷapʃəħəraj}{yesterday, evening} (cf. \mywordcre{дыгъуэпшыхь}{dəʁʷapʃəħ}{yesterday evening})
        \ex \mywordcre{нэгъабэрей}{naʁaːbaraj}{last year} (cf. \mywordcre{нэгъабэ}{naʁaːba}{last year})
        \ex \mywordcre{зымахуэрей}{zəmaːxʷaraj}{referring to some past day}
        \ex \mywordcre{ещанэгъэрей}{jaɕaːnaʁaraj}{referring to the third year}
    \end{xlist}
    \ex relation to place:
    \begin{xlist}
        \ex \mywordcre{ищхьэ}{jəɕħa}{upper} (cf. \mywordcre{щхьэ}{ɕħa}{head})
        \ex \mywordcre{ипэ/япэ}{jəpa/jaːpa}{front} (cf. \mywordcre{пэ}{pa}{nose})
        \ex \mywordcre{икIэ}{jətʃʼa}{last} (cf. \mywordcre{кIэ}{tʃʼa}{tail})
        \ex \mywordcre{модрэй}{mawdraj}{other}
    \end{xlist}
    \ex numerical relations:
    \begin{xlist}
        \ex \mywordcre{защIэ}{zaːɕʼa}{single} (cf. \mywordcre{зы}{zə}{one})
        \ex \mywordcre{тIуащIэ}{tʼwaːɕʼa}{double} (cf. \mywordcre{тIу}{tʼəw}{two})
    \end{xlist}
\end{xlist}
\end{exe}

\chapter{Numeral}
\section{Cardinal Numerals}
\begin{table}[h]
\centering
\begin{tabular}{l|l}
1 & \mywordcr{зы}{zə} \\
2 & \mywordcr{тIу}{tʼəw} \\
3 & \mywordcr{щы}{ɕə} \\
4 & \mywordcr{плIы}{pɬʼə} \\
5 & \mywordcr{тху}{txʷə} \\
6 & \mywordcr{хы}{xə} \\
7 & \mywordcr{блы}{bɮə} \\
8 & \mywordcr{и}{jə} \\
9 & \mywordcr{бгъу}{bʁʷə} \\
10 & \mywordcr{пщIы}{pɕʼə} \\
100 & \mywordcr{щэ}{ɕa}
\end{tabular}
\end{table}

Numbers from 11 to 19 are formed in a special way, by putting \mywordcr{-кIу-}{-kʷʼ-}, between the ten and the units.
\begin{table}[h]
\centering
\begin{tabular}{l|l}
    11 & \mywordcr{пщIыкIуз}{pɕʼəkʷʼəz} \\
    12 & \mywordcr{пщIыкIутI}{pɕʼəkʷʼətʼ} \\
    13 & \mywordcr{пщIыкIущ}{pɕʼəkʷʼəɕ} \\
    14 & \mywordcr{пщIыкIуплI}{pɕʼəkʷʼəpɬʼ} \\
    15 & \mywordcr{пщIыкIутху}{pɕʼəkʷʼətxʷ} \\
    16 & \mywordcr{пщIыкIух}{pɕʼəkʷʼəx} \\
    17 & \mywordcr{пщIыкIубл}{pɕʼəkʷʼəbɮ} \\
    18 & \mywordcr{пщIыкIуий}{pɕʼəkʷʼəj} \\
    19 & \mywordcr{пщIыкIубгъу}{pɕʼəkʷʼəbʁʷ} \\
\end{tabular}
\end{table}

Numbers which represent tens are formed by suffixing \mywordcr{-щI}{-ɕʼ} to the units (except for 10).
\begin{table}[h]
\centering
\begin{tabular}{l|l}
    20 & \mywordcr{тIощI}{tʼawɕʼ} \\
    30 & \mywordcr{щэщI}{ɕaɕʼ} \\
    40 & \mywordcr{плIыщI}{pɬʼəɕʼ} \\
    50 & \mywordcr{тхущI}{txʷəɕʼ}, \mywordcr{щэныкъуэ}{ɕanəqʷa} \\
    60 & \mywordcr{хыщI}{xəɕʼ} \\
    70 & \mywordcr{блыщI}{bɮəɕʼ} \\
    80 & \mywordcr{ищI}{jəɕʼ}, \mywordcr{пщIей}{pɕʼaj} \\
    90 & \mywordcr{бгъущI}{bʁʷəɕʼ} \\
\end{tabular}
\end{table}

The word for 80 has another version except its expected form, that is \mywordcr{пщIей}{pɕʼaj}. \mywordcr{щэныкъуэ}{ɕanəqʷa} (cf. \mywordcre{щэ}{ɕa}{hundred} + \mywordcre{ныкъуэ}{nəqʷa}{half}) is used less frequently than \mywordcr{тхущI}{txʷəɕʼ}.

Kabardian has still vestiges of a vegesimal system, which is a base-20 numeral system.

\begin{table}[h]
\centering
\begin{tabular}{l|l}
    20 & \mywordcr{тIощI}{tʼawɕʼ} \\
    30 & \mywordcr{щэщI}{ɕaɕʼ}, \mywordcr{тIощIрэ пщIырэ}{tʼawɕʼra pɕʼəra} \\
    40 & \mywordcr{плIыщI}{pɬʼəɕʼ}, \mywordcr{тIощIитI}{tʼawɕʼəjtʼ} \\
    50 & \mywordcr{тхущI}{txʷəɕʼ} \\
    60 & \mywordcr{хыщI}{xəɕʼ}, \mywordcr{тIощIищ}{tʼawɕʼəjɕ} \\
    70 & \mywordcr{блыщI}{bɮəɕʼ} \\
    80 & \mywordcr{ищI}{jəɕʼ} \\
    90 & \mywordcr{бгъущI}{bʁʷəɕʼ} \\
\end{tabular}
\end{table}

\chapter{Verb}
The verb is the 
\section{Transitivity}
The main overarching category for a verb is transitivity. A verb can be transitive (\mywordcr{лъэIэс}{ɬaʔas}) and intransitive (\mywordcr{лъэмыIэс}{ɬaməʔas}). The main difference is that a base intransitive verb governs only over 1 argument, the subject in the absolutive/nominative case (\mywordcr{-р}{-r}), while a base transitive verb governs over 2 arguments, the subject in the ergative case (\mywordcr{-м}{-m}) and the direct object in the absolutive/nominative case (\mywordcr{-р}{-r}).

Below the intransitive verb \mywordcre{плъэн}{pɬan}{to look} and transitive verb \mywordcre{лъагъун}{ɬaːʁʷən}{to see Y} are used to demonstrate that. With \mywordcre{лIы}{ɬʼə}{man} man and \mywordcre{жыг}{ʒəɣ}{tree}.

\begin{exe}
    \ex
    \begin{xlist}
    \item \mywordcre{лIыр плъэнщ}{ɬʼər pɬanɕ}{the man will look}
    \item \mywordcre{лIым жыгыр илъагъунщ}{ɬʼəm ʒəɣər jəɬaːʁʷənɕ}{the man will see the tree}
\end{xlist}
\end{exe}

However, in both cases, those base verbs can get indirect objects in the ergative case (\mywordcr{-м}{-m}), usually by deploying preverbs. A frequent example is the use of the preverb \mywordcr{е-}{ja-} with intransitive verbs, which is a very generic way to add an indirect object. In many ways it is equivalent to \myworde{to} or \myworde{at} or the French \myworde{à}. Thus leading to the verb \mywordcr{еплъын}{japɬən} (not that \mywordcr{э}{a} became \mywordcr{ы}{ə}).


\begin{exe}
\ex
\begin{xlist}
    \item \mywordcre{лIыр жыгым еплъынщ}{ɬʼər ʒəɣəm japɬənɕ}{the man will look at the tree}
    \item \mywordcre{лIым жыгыр илъэгъунщ}{ɬʼəm ʒəɣər jəɬaʁʷənɕ}{the man will see the tree}
\end{xlist}
\end{exe}

As one can see, intransitive and transitive verbs can have 2 arguments and if that is the case the cases are switched. While with an intransitive with 2 arguments the subject is in the absolutive/nominative with an (indirect) object in the ergative, the transitive verb has it the other way around, the subject is in the ergative case while the (direct) object is in the absolutive/nominative. 

Important to note, that while morphologically there is only one ergative case, it is useful to differentiate between the 'pure' ergative and the oblique case. The 'pure' usage would be only in regards to the use as subject, while the oblique usage would be everything else. More on that later.

\section{Potential Form}
There are two potential forms which modify the verb in a such a way that it expresses a potential. One form is formed through the prefix \mywordcr{хуэ-}{xʷa-} while the the other is formed through the suffix \mywordcr{-ф}{-f}. The latter can only be used with transitive verbs (as it makes them intransitive) and the latter can be used by all verbs.

\subsection{Suffix {\mywordcr{-ф}{-f}}}
\begin{exe}
    \ex
    \begin{xlist}
    \item \mywordcre{бзу мэлъэтэ\underline{ф}}{bzəw maɬata\underline{f}}{birds \underline{can} fly}
    \item \mywordcre{сыжей\underline{ф}къым}{səʒaj\underline{f}qəm}{I \underline{can}'t sleep}
    \item \mywordcre{сыжей\underline{фы}нутэкъым}{səʒaj\underline{fə}nəwtaqəm}{I \underline{can}'t sleep}
\end{xlist}
\end{exe}

\subsection{Prefix \mywordcr{хуэ-}{xʷa-}}
The prefix potential form is morphologically more complex. It only works with transitive verbs, since it works be `deleting' the subject in the ergative and transfers the subject role to the argument of the prefix \mywordcr{хуэ-}{xʷa-}.

\begin{exe}
    \ex
    \begin{xlist}
    \item \mywordcre{с\underline{ху}ошх}{s\underline{xʷ}awʃx}{I \underline{can}'t eat it}
    \item \mywordcre{мыщэ сэ сы\underline{ху}ошх}{məɕa sa sə\underline{xʷ}awʃx}{the bear \underline{can} eat me}
    \item \mywordcre{бзу цIыкIур къыт\underline{хуэ}гъуэтакъым}{bzəw tsʼəkʷʼər qət\underline{xʷa}ʁʷataːqəm}{we couldn't find the small bird}
    % \item \mywordcre{сыжеи\underline{фы}нутэкъым}{səʒajə\underline{fə}nəwtaqəm}{I \underline{can}'t sleep}
\end{xlist}
\end{exe}

It seems posssible that this prefix and the benefactive \mywordcr{xуэ-}{xʷa-} are related.
\begin{exe}
    \ex
    \begin{xlist}
    \item \mywordcre{X сэ сишхащ}{}{X ate me}
    \item \mywordcre{X сэ Z сыхуишхащ}{}{X ate me for Z}
    \item \mywordcre{сэ Z сыхуэшхащ}{}{ate me for Z} or \myworde{I was eaten for Z}
    \item \mywordcre{сэ Z сыхуэшхащ}{}{Z was able to eat me}
\end{xlist}
\end{exe}


\section{Participles}
Kabardian has a rich participle morphology. In fact, Kabardian is a very participle heavy language and one can even argue that many verb forms, which are usually not regarded as participles, are in fact participles.

The various participle types are divided into whether they represent an argument of a verb, where in this case they take on the slot of that argument, or whether they represent something else, usually more adverbial in nature, like place, time, reason, manner, etc.

Other than not being able to change the grammatical category of mood, participle can mark for everything else what normal verbs can. 

\subsection{Absolutive Participle}
The absolutive participle denotes the absolutive argument of a verb. If that verb is intransitive, it refers to the subject, if it is transitive it refers to the direct object. Absolutive participles are marked by a null morpheme. 

Intransitive verbs:
\begin{exe}
\ex
\begin{xlist}
    \ex \mywordcre{кIуэр}{kʷʼar}{one, who goes} (cf. \mywordcre{кIуэн}{kʷʼaн}{to go})
    \ex \mywordcre{жэр}{ʒar}{one, who runs} (cf. \mywordcre{жэн}{ʒan}{to run})
\end{xlist}
\end{exe}

Intransitive verbs with preverbs: 
\begin{exe}
\ex
\begin{xlist}
    \ex \mywordcre{ежьэр}{jaʑar}{one who waits for Y} (cf. \mywordcre{ежьэн}{jaʑan}{to wait for Y})  
    \ex \mywordcre{еплъыр}{japɬər}{one who looks at Y} (cf. \mywordcre{еплъын}{japɬən}{to look at Y})    
    \ex \mywordcre{едэIуэр}{jadaʔʷar}{one who listens to Y} (cf. \mywordcre{едэIуэн}{jadaʔʷan}{to listen to Y})
\end{xlist}
\end{exe}

Transitive verbs:
\begin{exe}
\ex
\begin{xlist}
    \ex \mywordcre{илъэгъур}{jəɬaʁʷər}{one, whom X sees} (cf. \mywordcre{лъагъун}{ɬaːʁʷən}{to see Y})
    \ex \mywordcre{ишэр}{jəʃar}{one, whom X leads} (cf. \mywordcre{шэн}{ʃan}{to lead Y})
    \ex \mywordcre{итыр}{jətər}{that, what X gives} (cf. \mywordcre{тын}{tən}{to give Y})
\end{xlist}
\end{exe}

Transitive verbs with preverb:
\begin{exe}
\ex
\begin{xlist}
    \item \mywordcre{зэхихыр}{zaxəjxər}{one, whom X hears} (cf. \mywordcre{зэхэхын}{zaxaxən}{to hear Y})
    \item \mywordcre{жыпIар}{ʒəpʔaːr}{that, what you said} (cf. \mywordcre{жыIэн}{ʒəʔan}{to say Y})
    \item \mywordcre{иритыр}{jərəjtər}{that, what X gives to Z} (cf. \mywordcre{етын}{jatən}{to give Y to Z})
\end{xlist}
\end{exe}

\subsection{Ergative Participle}
The ergative participle denotes the ergative argument of a verb. This participle is only present in transitive verbs and refers to the subject. It is marked by \mywordcr{зы-}{zə-}.

Transitive verbs:
\begin{exe}
\ex
\begin{xlist}
    \item \mywordcre{\underline{зы}лъэгъур}{\underline{zə}ɬaʁʷər}{one, who sees Y} (cf. \mywordcre{лъагъун}{ɬaːʁʷən}{to see Y})
    \item \mywordcre{\underline{зы}шэр}{\underline{zə}ʃar}{one, who leads Y} (cf. \mywordcre{шэн}{ʃan}{to lead Y})
    \item \mywordcre{\underline{зы}тыр}{\underline{zə}tər}{one, who gives Y} (cf. \mywordcre{тын}{tən}{to give Y})
\end{xlist}
\end{exe}

Transitive verbs with preverb:
\begin{exe}
    \ex
    \begin{xlist}
    \item \mywordcre{зэхэ\underline{зы}хыр}{zaxa\underline{zə}xər}{one, who hears Y} (cf. \mywordcre{зэхэхын}{zaxaxən}{to hear Y})
    \item \mywordcre{жы\underline{зы}Iар}{ʒə\underline{zə}ʔaːr}{one, who said Y} (cf. \mywordcre{жыIэн}{ʒəʔan}{to say Y})
    \item \mywordcre{е\underline{зы}тыр}{ja\underline{zə}tər}{one, who gives Y to Z} (cf. \mywordcre{етын}{jatən}{to give Y to Z})
\end{xlist}
\end{exe}

\subsection{Oblique Participle}
The oblique participle denotes the oblique argument of a verb. This participle is present in every verb which has an oblique argument, usually only possible by having a preverb. It is marked by \mywordcr{зы-}{zə-}. One can argue that it is simply the ergative participle, but simply applied on a preverb, however it may be useful to differentiate them, because oblique participles don't denote the subject of a verb (the ergative participle can only denote the subject) in addition they exist for transitive and intransitive verbs (the ergative participle only exists for transitive verbs). 

Intransitive verbs with preverbs: 
\begin{exe}
    \ex
    \begin{xlist}
    \item \mywordcre{\underline{з}эжьэр}{\underline{z}aʑar}{one, whom X waits for} (cf. \mywordcre{ежьэн}{jaʑan}{to wait for Y})
    \item \mywordcre{\underline{з}эплъыр}{\underline{z}apɬər}{one, who X looks at} (cf. \mywordcre{еплъын}{japɬən}{to look at Y})
    \item \mywordcre{\underline{з}эдэIуэр}{\underline{z}adaʔʷar}{one, who X listens to} (cf. \mywordcre{едэIуэн}{jadaʔʷan}{to listen to Y})
    \item \mywordcre{сы\underline{зы}тесыр}{sə\underline{zə}tajsər}{that, what I sit on} (cf. \mywordcre{тесын}{tajsən}{to sit on Y})
\end{xlist}
\end{exe}

Transitive verbs with preverb:
\begin{exe}
    \ex
    \begin{xlist} 
    \item \mywordcre{\underline{зы}ритыр}{\underline{zə}rəjtər}{one, to whom X gives Y} (cf. \mywordcre{етын}{jatən}{to give Y to Z})
\end{xlist}
\end{exe}


Sentence Examples:
\begin{exe}
    \ex
    \begin{xlist}   
    \item \mywordcre{дыгъуасэ хъыджэбз сы\underline{зы}хуэзар дахэщ}{dəʁʷaːsa χədʒabz sə\underline{zə}xʷazaːr daːxaɕ}{the girl, whom I met yesterday is pretty} (cf. \mywordcre{хуэзэн}{xʷazan}{to meet Y})
\end{xlist}
\end{exe}

\subsection{Temporal Participle \mywordcr{щы-}{ɕə-}}
This participle denotes time as well as location, depending on the context. It is marked by щы-.
\begin{exe}
\ex
\begin{xlist} 
    \item \mywordcre{\underline{щы}лажьэр}{\underline{ɕə}ɮaːʑar}{when X works} (cf. \mywordcre{лэжьэн}{ɮaʑan}{to work})
\end{xlist}
\end{exe}

Sentence Examples:
\begin{exe}
    \ex
    \begin{xlist}   
    \item \mywordcre{ар \underline{щы}лажьэр унэрщ}{aːr \underline{ɕə}ɮaːʑar wənarɕ}{he works at home} (lit. \myworde{where he works is home})
    \item \mywordcre{ар \underline{щы}лажьэр сощIэ}{aːr \underline{ɕə}ɮaːʑar sawɕʼa}{I know where/when he works}
    \item \mywordcre{уэ укъы\underline{щы}кIуэжам щыгъуэ сэ унэм сыщыIакъым}{wa wəqə\underline{ɕə}kʷʼaʒaːm ɕəʁʷa sa wənam səɕəʔaːqəm}{I wasn't home when you arrived}
    \item \mywordcre{сы\underline{щ}илъэгъум, ар жащ}{sə\underline{ɕə}jɬaʁʷəm, ar ʒaːɕ}{When he saw me, he ran away}
    % \item \mywordcre{}{}{}
\end{xlist}
\end{exe}

\subsection{Adverbial Participle \mywordcr{зэры-}{zarə-}}
The manner participle denotes the manner. It is marked by \mywordcr{зэры-}{zarə-}.

\begin{exe}
\ex
\begin{xlist}
    \ex \mywordcre{\underline{зэры}лажьэр}{\underline{zarə}ɮaːʑar}{how, X works} (cf. \mywordcre{лажьэн}{ɮaːʑan}{to work})
    \ex \mywordcre{\underline{зэр}илъэгъур}{\underline{zar}əjɬaʁʷər}{how, X sees} (cf. \mywordcre{лъагъун}{ɬaːʁʷən}{to see Y})
\end{xlist}
\end{exe}

This is frequently used in complement clauses as a generic complementizer.
\begin{exe}
\ex
\begin{xlist}
    \ex \mywordcre{укъы\underline{зэры}фэр слъэгъуащ}{wəqə\underline{zarə}far sɬaʁʷaːɕ}{I saw, \underline{that} you danced}
    \ex \mywordcre{анэм ещIэ и кIуэр къы\underline{зэры}кIуэжынур}{aːnam jaɕʼa jə kʷʼar qə\underline{zarə}kʷʼaʒənəwr}{mother knows, \underline{that} her son will return (home)}
    \ex \mywordcre{щIалэм хъыбар сигъэщIащ фы\underline{зэры}сымаджар}
    {ɕʼaːɮam χəbaːr səjʁaɕʼaːɕ fə\underline{zarə}səmaːdʒaːr}
    {the boy let me know, \underline{that} you were sick}
    \ex \mywordcre{сэ сщыгъупщэнкъым у\underline{зэры}лэжьэнур}{sa sɕəʁʷəpɕanqəm wə\underline{zarə}ɮaʑanəwr}{I won't forget, \underline{that} you'll work}
    % \ex \mywordcre{}{}{}
\end{xlist}
\end{exe}

% \chapter{Syntax}
% \chapter{Semantics}
% \chapter{Pragmatics}

\chapter{Postpositions}

\chapter{Vocabulary}
This chapter will explore the vocabulary.
\section{Semantic Categories}
\subsection{Kinship}
This category is about family.

    
\begin{table}[ht]    
    \caption{Kinship Terms}\
    \begin{tabular}{ l | c | r }
        Term & Translation & Notes \\
        \hline
        \mywordcr{адэ}{aːda} & \myworde{father} &  \\
        \mywordcr{анэ}{aːnа} & \myworde{mother} &  \\
        \mywordcr{дадэ}{daːdа} & \myworde{grandfather} &  \\
        \mywordcr{нанэ}{nаːnа} & \myworde{grandmother} &  \\
        \mywordcr{адэшхуэ}{aːdаʃxʷa} & \myworde{grandfather} & more formal \\
        \mywordcr{анэшхуэ}{аːnаʃxʷa} & \myworde{grandmother} & more formal \\
        \hline
        \mywordcr{бын}{bən} & \myworde{child} &  \\
        \mywordcr{къуэ}{qʷa} & \myworde{son} &  \\
        \mywordcr{пхъу}{pχʷə} & \myworde{daughter} &  \\
        \hline
        \mywordcr{къуэш}{qʷaʃ} & \myworde{brother} & said by males \\
        \mywordcr{дэлъху}{daɬxʷ} & \myworde{brother} & said by females \\
        \mywordcr{шыпхъу}{ʃəpχʷ} & \myworde{sister} &  \\
        \hline
        \mywordcr{зэтIолъхуэныкъуэ}{zatʼawɬxʷanəqʷa} & \myworde{twins} &  \\

    \end{tabular}
\end{table}



The terms \mywordcr{адэ}{aːdа} and \mywordcr{анэ}{аːnа} denote \myworde{father} and \myworde{mother}, respectively. 

The generation above, the parents' parents, is denoted by \mywordcre{адэшхуэ}{aːdaʃxʷa}{grandfather} and \mywordcre{анэшхуэ}{аːnаʃxʷa}{grandmother}, respectively, derived with the suffix \mywordcr{-шхуэ}{-ʃxʷa} which denotes biggness, thus literally \myworde{big father} and \myworde{big mother}. There is no differentiation made between the maternal and paternal grandparents. On the other hand, \mywordcr{дадэ}{daːdа} and \mywordcr{нанэ}{nаːnа} are the more endearing forms of the former and are also used when referring to them.

The terms describing the direct offspring are \mywordcre{бын}{bən}{child}, \mywordcre{къуэ}{qʷa}{son} and \mywordcre{пхъу}{pχʷə}{daughter}. The term \mywordcre{пхъу}{pχʷə}{daughter} had likely \myworde{woman} as its main meaning, as it is frequently combined with other words to refer to females, some of them will be seen below.

The terms describing siblings are \mywordcr{къуэш}{qʷaʃ}, \mywordcr{дэлъху}{daɬxʷ} for \myworde{brother} and \mywordcre{шыпхъу}{ʃəpχʷ}{sister}. Females always refer to their brother as \mywordcr{дэлъху}{daɬxʷ} (roughly \myworde{one, who is born with}), while males as \mywordcr{къуэш}{qʷaʃ}.
The term \mywordcre{къуэш}{qʷaʃ}{brother} appears to be a compound word of \mywordcre{къуэ}{qʷa}{son} and \mywordcr{шы}{ʃə} which is an archaic way to refer to \myworde{brother}, as it is still used in Western Circassian \mywordcre{шы}{ʃə}{brother}. The term \mywordcre{шыпхъу}{ʃəpχʷ}{sister} is a compound word of \mywordcre{шы}{ʃə}{brother} and \mywordcre{пхъу}{pχʷə}{daughter}. This and some other uses suggest that the original meaning of \mywordcr{шы}{ʃə} was closer to \myworde{relative}, \myworde{kin} or \myworde{sibling}.


\begin{table}[ht]    
    \caption{Kinship Terms}\
    \begin{tabular}{ l | c | r }
        Term & Translation & Notes \\
        \hline
        \mywordcr{бынырылъху}{bənərəɬxʷ} & \myworde{child's offspring} &  \\
        \mywordcr{къуэрылъху}{qʷarəɬxʷ} & \myworde{son's offspring} &  \\
        \mywordcr{пхъурылъху}{pχʷərəɬxʷ} & \myworde{daufhter's offspring} &  \\
        \hline
        \mywordcr{къуэшырылъху}{qʷaʃərəɬxʷ} & \myworde{brother's offspring} &  \\
        %\mywordcr{дэлъхурылъху}{daɬxʷərəɬxʷ} & \myworde{} &  \\
        \mywordcr{шыпхъурылъху}{ʃəpχʷərəɬxʷ} & \myworde{sister's offspring} &  \\
        \hline

    \end{tabular}
\end{table}

The suffix \mywordcr{-рылъху}{-rəɬxʷ} denotes the offspring of the base noun.


\begin{table}[h]    
    \caption{Kinship Terms}\
    \begin{tabular}{ l | c | r }
        Term & Translation & Notes \\
        \hline       
        \mywordcr{тхьэмадэ}{tħamaːda} & \myworde{husband's father} & \\
        \mywordcr{гуащэ}{gʷaːɕa} & \myworde{husband's mother} & \\
        \mywordcr{пщыкъуэ}{pɕəqʷa} & \myworde{husband's brother} & \\
        \mywordcr{пщыпхъу}{pɕəpχʷ} & \myworde{husband's sister} & \\
        \hline
        \mywordcr{щыкъу адэ}{ɕəqʷ aːda} & \myworde{wife's father} & \\
        \mywordcr{щыкъу анэ}{ɕəqʷ aːnа} & \myworde{wife's mother} & \\
        \mywordcr{щыкъу щIалэ}{ɕəqʷ ɕʼaːɬa} & \myworde{wife's son} & \\
        \mywordcr{щыкъу хъыджэбз}{ɕəqʷ χədʒabz} & \myworde{wife's daughter} & \\
        \hline
        \mywordcr{фызабэ}{fəzaːba} & \myworde{widow} & \\
        \mywordcr{лIыгъуабэ}{ɬʼəʁʷaːba} & \myworde{widower} & \\
        \mywordcr{зэиншэ}{zajənʃa} & \myworde{orphan} & \\
        \mywordcr{ибэ}{jəba} & \myworde{orphan} & \\
    \end{tabular}
    \end{table}


\part{Syntax}
% This part talks about the syntax.
\chapter{Relative Clause}
\chapter{Adverbial Clause}
\chapter{Complement Clause}

\chapter{Comparing the Circassian Dialects}
This chapter is about comparing the Circassian dialects. The main focus will be on the Baksan and Temirgoy dialects, the two dialects on which the standardized literary languages, Kabardian (East Circassian) and Adyghe (Western Circassian), are based on. The others will be also mentioned but not in as much detail. This chapter is a good way to quickly get a grasp on Adyghe if one already knows Kabardian (and possibly vice versa), since
\begin{table}[h]    
\caption{Kinship Terms}
\begin{tabular}{ l | c | c | c }
    Meaning & Kabardian (Baksan) & Basleney & Adyghe (Termingoy) \\
    \hline
    chicken & \mywordcr{джэд}{dʒad} & \mywordcr{гяд}{gʲad} & \mywordcr{чэты}{tʃatə} \\
    cat & \mywordcr{джэду}{dʒadəw} & \mywordcr{гяду}{gʲadəw} & \\
\end{tabular}
\end{table}

\part{Learning Guide}
\chapter{Learning}

\end{document}
\documentclass[a4paper,12pt]{book}
\usepackage{fontspec}
\setmainfont{Charis SIL}

\usepackage{tocbasic}
%\usetocstyle{standard}
\usepackage{longtable}
\usepackage{gb4e}
\usepackage{fullpage}
\usepackage{hyperref}
\hypersetup{
    colorlinks = true,
    linkcolor = blue
}

\usepackage{nameref}

\usepackage{multirow}% http://ctan.org/pkg/multirow
\usepackage{multicol}
\usepackage{hhline}% http://ctan.org/pkg/hhline
\newsavebox\ltmcbox

\pagestyle{headings}
\pagestyle{plain}

\newcommand{\1}[1]{\textbf{\emph{#1}}} %at'ik words
\newcommand{\2}[1]{\textbf{[#1]}} %at'ik phonetics
\newcommand{\3}[1]{\fontsize{11pt}{0cm}\textbf{\emph{#1}}} %at'ik morphemes
\newcommand{\4}[1]{\fontsize{10pt}{0cm}\emph{#1}}	%names of morphemes
\newcommand{\5}[1]{\textbf{/#1/}} %at'ik phoneMics //
\newcommand{\6}[1]{\textbf{[#1]}} %at'ik phoneTics []
\newcommand{\7}[1]{\fontsize{12pt}{0cm}\emph{#1}} %emphasis
\newcommand{\8}[1]{\fontsize{12pt}{0cm}`#1'} %English words
\newcommand{\9}[1]{\fontsize{12pt}{0cm}(lit. `#1')} %English words

\newcommand{\glossphonemics}[1]{\textbf{/#1/}} %at'ik phoneMics //


\title{Test}
\begin{document}

\frontmatter
\maketitle\newpage
\setcounter{secnumdepth}{4}
\setcounter{tocdepth}{4}

\newpage
\mainmatter

\chapter{Introduction}
\chapter{Orthography}
This document tries to use a more or less consistent transliteration (romanization) of the Kabardian alphabet which is mainly a Cyrillic one. However, from time to time, Cyrillic texts will also appear, especially if its a copy and paste of larger texts.

The main goal of the transliteration is to be phonemically consistent and not phonetically.
\chapter{Roots}
\glossphonemics{.} - \textbf{.}
\glossphonemics{ɮ} - \textbf{v: to burn}
\glossphonemics{ɮa} - \textbf{v: to color X}
\glossphonemics{.} - \textbf{.}
\chapter{Verb}ɮ
\section{Intransitive monovalent}
\textbf{kʼʷa-n} - "to go"\\
\textbf{ʒa-n} - "to run"\\
\textbf{d⁀ʒagʷ-ən} - "to play"\\


\textbf{sawd⁀ʒagʷ} - "I play"\\
\textbf{wawd⁀ʒagʷ} - "you play"\\
\textbf{mad⁀ʒagʷ} - "he plays"\\
\textbf{dawd⁀ʒagʷ} - "we play"\\
\textbf{fawd⁀ʒagʷ} - "you play"\\
\textbf{mad⁀ʒagʷ} - "they play"\\


\section{Intransitive divalent}
\section{Transitive monovalent}
\section{Transitive divalent}
\chapter{Texts}
\subsection{The Parable 1}
This is the parable from the youtube chanell ILoveLanguages!. https://www.youtube.com/watch?v=OgAi-H3Cd6Q

лIы гуэрым къуытI иIэт, - жиIащ хьисэ.

нэхъыщIэм адэм мыпхуэдэу жриIащ: "ди адэ, уи щIэиным щыщу кыслъысын хуейр къызэт". апхуэдэу щыхъум, адэм мылъкур зэшхэм яхуигуэшащ.


\begin{exe}
\ex
\1{ЛӀы гуэрым къуитӀ иӀэт, – жиӀащ Хьисэ.}\\
\2{ɬʼə gʷarəm qʷəjtʼ jəʔat, – ʒəjʔaːɕ ħəjsa}\\
\3{ЛӀы гуэр-ым къу-и-тӀ и-Ӏэ-т, – ж-и-Ӏ-а-щ Хьисэ}\\
\4{man some-ERG son-CONNECT-two 3-have-past}\
\trans \8{}

\ex
\1{нэхъыщIэм адэм мыпхуэдэу жриIащ: "ди адэ, уи щIэиным щыщу кыслъысын хуейр къызэт". }\\
\2{naχəɕʼam aːdam məpxʷadaw ʒrəjʔaːɕ dəj aːda qəj ɕʼajənəm ɕəɕəw qəsɬəsən xʷejr qəzat}\\
\3{нэхъыщIэм адэм мыпхуэдэу жриIащ}\\
\4{younger_one-ERG father-ERG that_way }\
\trans \8{}

\ex
\1{апхуэдэу щыхъум, адэм мылъкур зэшхэм яхуигуэшащ.}\\
\2{aːpxʷadaw ɕəχʷəm aːdam məɬkʷər zaʃxam jaːxʷəjgʷaʃaːɕ}\\
\3{нэхъыщIэм адэм мыпхуэдэу жриIащ}\\
\4{younger_one-ERG father-ERG that_way }\
\trans \8{}
\end{exe}



\subsection{The Parable 2}
ЛӀы гуэрым къуитӀ иӀэт, – жиӀащ Хьисэ. 

Къуэ нэхъыщӀэм и адэм жриӀащ: «Си адэ, мылъкум щыщу къыслъысыну Ӏыхьэр къызэт». Арати, адэм мылъкур яхуигуэшащ.

Зыкъом дэкӀа нэужь, къуэ нэхъыщӀэм иӀэ псори зэщӀикъуэри, хамэ къэрал гуэрым кӀуащ. И мылъкур а къэралъым щхьэпсыншагъэкӀэ щигъэкӀуэдащ. 

Псори фӀэкӀуэдауэ, а къэралым гъаблэшхуэ къыщыхъури, щӀалэр тхьэмыщкӀэ дыдэ хъуащ. 

Апхуэдэу щыхъум, а къэралым щыпсэу лӀы гуэрым деж кӀуэри, хуэлӀыщӀэну гурыӀуащ. А лӀым ар игъэкӀуащ и кхъуэхэр губгъуэм щригъэгъэхъуну.

Кхъуэхэм я Ӏусым щыщ и ныбэ из ищӀыну ехъуапсэрт лӀыщӀэм, ауэ а Ӏусым щыщ зыми къритакъым

\begin{exe}
\ex
\1{ЛӀы гуэрым къуитӀ иӀэт, – жиӀащ Хьисэ.}\\
\2{ɬʼə gʷarəm qʷəjtʼ jəʔat, – ʒəjʔaːɕ ħəjsa}\\
\3{ЛӀы гуэр-ым къу-и-тӀ и-Ӏэ-т, – ж-и-Ӏ-а-щ Хьисэ}\\
\4{man some-ERG son-CONNECT-two 3-have-past}\
\trans \8{}

\ex
\1{Къуэ нэхъыщӀэм и адэм жриӀащ: «Си адэ, мылъкум щыщу къыслъысыну Ӏыхьэр къызэт». Арати, адэм мылъкур яхуигуэшащ.}\\
\2{qʷa naχəɕʼam jə aːdam ʒrəjʔaːɕ: «səj aːda, məɬkʷəm ɕəɕəw qəsɬəsənəw ʔəħar qəzat». aːraːtəj, aːdam məɬkʷər jaːxʷəjgʷaʃaːɕ.}\\
\3{}\\
\4{}\
\trans \8{}


\ex
\1{}\\
\2{}\\
\3{}\\
\4{}\
\trans \8{}
\end{exe}


\subsection{Adyghe Nise}

Адыгэ пшынэм щlыгур къыпэджэмэ, \\
уоирэ, уорирэ, уойра,\\
Бжэlупэ джэгум къуажэр зэхишэмэ,\\
Хъуэхъубжьэ усэр псэм зэхидзыжымэ,\\
Адыгэ нысэ унэ идошэрэ, \\
уоирэ, уорирэ, уойра.\\
\\
Ей жи, адыгэ лъахэ,\\
Ей жи, ди хабзэ дахэ,\\
Дарий lэщхьэху, къамэ lэпщэху,\\
Нэ фlыцlэ нэху, джэгу умыпсэху,\\
Адыгэ нысэ унэ идошэрэ, уоирэ, уорирэ, уойра.\\
\\\\
Хьэзыр lупэхуу ди щауэ уардэхэм,\\
уоирэ, уорирэ, уойра,\\
Ди лъэпкъ гурыщlэр я уэредадэмэ,\\
Дыгъэр зэхъуапсэр я дыщэидэмэ,\\
Адыгэ нысэ унэ идошэрэ,\\
уоирэ, уорирэ, уойра.\\
\\\\
Ей жи, адыгэ лъахэ,\\
Ей жи, ди хабзэ дахэ,\\
Дарий lэщхьэху, къамэ lэпщэху,\\
Нэ фlыцlэ нэху, джэгу умыпсэху,\\
Адыгэ нысэ унэ идошэрэ,\\
уоирэ, уорирэ, уойра.\\

Гъатхэм и дахэр уэрам къыдихьэмэ,\\
уоирэ, уорирэ, уойра,\\
Бжьыхьэм и бэвыр жьэгум къыдилъхьэмэ,\\
Къалъхуа мазэщlэр и нэмыс щыпкъэмэ,\\
Адыгэ нысэ унэ идошэрэ,\\
уоирэ, уорирэ, уойра.\\
\\
Ей жи, адыгэ лъахэ,\\
Ей жи, ди хабзэ дахэ,\\
Дарий lэщхьэху, къамэ lэпщэху,\\
Нэ фlыцlэ нэху, джэгу умыпсэху,\\
Адыгэ нысэ унэ идошэрэ,\\
уоирэ, уорирэ, уойра.\\

И фоужь матэр дадэ икъутэмэ,\\
уоирэ, уорирэ, уойра,\\
Пшэплъыфэ lупэм хуахьыр фоущхьэмэ,\\
Фlыгъуэ къепхъыхыр уэшхыу къыттешхэмэ,\\
Адыгэ нысэ унэ идошэрэ,\\
уоирэ, уорирэ, уойра.\\

Ей жи, адыгэ лъахэ,\\
Ей жи, ди хабзэ дахэ,\\
Дарий lэщхьэху, къамэ lэпщэху,\\
Нэ фlыцlэ нэху, джэгу умыпсэху,\\
Адыгэ нысэ унэ идошэрэ,\\
уоирэ, уорирэ, уойра.\\



\begin{exe}
\ex
\1{Адыгэ пшынэм щlыгур къыпэджэмэ}\\
\2{aːdəɣa pʃənam ɕʼəɡʷər qəpad͡ʒama}\\
\3{Адыгэ пшынэ-м щlыгу-р къы-пэджэ-мэ}\\
\4{Circassian accordion-ERG earth-ABS greet-if}\
\trans \8{If the Earth greets the Circassian accordion}

\ex
\1{Бжэlупэ джэгум къуажэр зэхишэмэ}\\
\2{bʒaʔʷəpa d͡ʒagʷəm qʷaːʒar zaxjəʃama}\\
\3{Бжэlупэ джэгу-м къуажэ-р зэ-х-и-шэ-мэ}\\
\4{front\_door\_yard wedding-ERG village-ABS gather}\
\trans \8{If the wedding at the gate gathers the village}


\ex
\1{Хъуэхъубжьэ усэр псэм зэпидзыжымэ}\\
\2{χʷaχʷbʑa wəsar psam zapjəd͡zəʒəma}\\
\3{Хъуэхъубжьэ усэ-р псэ-м зэ-п-и-дзы-жы-мэ}\\
\4{toast rhyme-ABS soul-ERG}\
\trans \8{If the soul replies to the toast rhymes}


\ex
\1{Адыгэ нысэ унэ идошэрэ}\\
\2{aːdəɣa nəsa wəna jədawʃara}\\
\3{Адыгэ нысэ унэ и-д-о-шэ-рэ}\\
\4{Circassian bride house }\
\trans \8{we bring home the Circassian bride}

\end{exe}


\begin{exe}
\ex
\1{Ей жи, адыгэ лъахэ}\\
\2{ej ʒjə aːdəɣa ɬaːxa}\\
\3{}\\
\4{}\
\trans \8{Ey Circassian homeland}
\ex
\1{Ей жи, ди хабзэ дахэ}\\
\2{ej ʒjə djə xaːbza daːxa}\\
\3{}\\
\4{}\
\trans \8{Ey our beautiful Habze}
\ex
\1{Дарий lэщхьэху, къамэ lэпщэху}\\
\2{daːrjə ʔaɕħaxʷ qaːma ʔapɕaxʷ}\\
\3{}\\
\4{satin _ dagger }\
\trans \8{Ey Circassian people}
\ex
\1{Нэ фlыцlэ нэху, джэгу умыпсэху}\\
\2{na fʼət͡sʼa naxʷ d͡ʒagʷ wəməpsaxʷ}\\
\3{Нэ фlыцlэ нэху, джэгу у-мы-псэху}\\
\4{eye black bright, play you-not-stop}\
\trans \8{bright black eyes, don't stop playing}

\end{exe}

\begin{exe}
\ex
\1{Хьэзыр lупэхуу ди щауэ уардэхэмэ}\\
\2{ħazər ʔʷəpaxʷəw djə ɕaːwa waːrdaxama}\\
\3{}\\
\4{}\
\trans \8{beautifully dressed young men}
\ex
\1{Ди лъэпкъ гурыщlэр я уэредадэмэ}\\
\2{djə ɬapq gʷərəɕʼar jaː warejdaːdama}\\
\3{Ди лъэпкъ гурыщlэ-р я уэредадэмэ}\\
\4{our nation true-ERG their wedding\_song}\
\trans \8{If it is the song of my nation}
\ex
\1{Дыгъэр зэхъуапсэр я дыщэидэмэ}\\
\2{dəʁar zaχʷaːpsar jaː dəɕajədama}\\
\3{Дыгъэ-р зэхъуапсэ-р я дыщэидэ-мэ}\\
\4{}\
\trans \8{the gold engraving that the sun envies}
\ex
\1{Адыгэ нысэ унэ идошэрэ}\\
\2{}\\
\3{}\\
\4{}\
\trans \8{}

\end{exe}

\begin{exe}
\ex
\1{Гъатхэм и дахэр уэрам къыдихьэмэ}\\
\2{ʁaːtxam jə daːxar waraːm qədəjħama}\\
\3{Гъатхэ-м и дахэ-р уэра-м къы-д-и-хьэ-мэ}\\
\4{spring-ERG its beautiful-ABS street-ERG}\
\trans \8{}
\ex
\1{Бжьыхьэм и бэвыр жьэгум къыдилъхьэмэ}\\
\2{bʑəħam jə bavər ʑagʷəm qədəjɬħama}\\
\3{Бжьыхьэ-м и бэв-ыр жьэгу-м къы-д-и-лъхьэ-мэ}\\
\4{fall its plentiful hearth}\
\trans \8{if the fall puts its plentifullness between the hearth}
\ex
\1{Къалъхуа мазэщlэр и нэмыс щыпкъэмэ}\\
\2{qaːɬxʷa maːzaɕʼar jə naməs ɕəpqama}\\
\3{Къалъхуа мазэщlэ-р и нэмыс щыпкъэ-мэ}\\
\4{born new_moon-ERG its honor excellent}\
\trans \8{}
\ex
\1{Адыгэ нысэ унэ идошэрэ}\\
\2{}\\
\3{}\\
\4{}\
\trans \8{}

\end{exe}




И фоужь матэр дадэ икъутэмэ,\\
уоирэ, уорирэ, уойра,\\
Пшэплъыфэ lупэм хуахьыр фоущхьэмэ,\\
Фlыгъуэ къепхъыхыр уэшхыу къыттешхэмэ,\\
Адыгэ нысэ унэ идошэрэ,\\
уоирэ, уорирэ, уойра.\\

\begin{exe}
\ex
\1{И фоужь матэр дадэ икъутэмэ}\\
\2{jə fawwəʑ maːtar daːda jəqʷətama}\\
\3{И ф-о-ужь матэ-р дадэ и-къутэ-мэ}\\
\4{}\
\trans \8{}
\ex
\1{Пшэплъыфэ lупэм хуахьыр фоущхьэмэ}\\
\2{pʃapɬəfa ʔʷəpam xʷaːħər fawwəɕħama}\\
\3{Пшэплъыфэ lупэ-м хуах-ьыр ф-о-ущхьэмэ}\\
\4{}\
\trans \8{if the fall puts its plentifullness between the hearth}
\ex
\1{Фlыгъуэ къепхъыхыр уэшхыу къыттешхэмэ}\\
\2{qaːɬxʷa maːzaɕʼar jə naməs ɕəpqama}\\
\3{Къалъхуа мазэщlэ-р и нэмыс щыпкъэ-мэ}\\
\4{born new_moon-ERG its honor excellent}\
\trans \8{}
\ex
\1{Адыгэ нысэ унэ идошэрэ}\\
\2{}\\
\3{}\\
\4{}\
\trans \8{}

\end{exe}




\begin{exe}
\ex
\1{ghatsws'i ghatswgi iktzharta}\\
\2{}\\
\3{ghatsw-s'i ghatsw-gi iktzharta}\\
\4{}\
\trans \8{18 wolves have (just) attacked us} (lit. eleven wolves and seven wolves attacked us)
\end{exe}


\end{document}